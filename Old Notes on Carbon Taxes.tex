%2multibyte Version: 5.50.0.2960 CodePage: 65001
%\input{tcilatex}
%\input{tcilatex}


\documentclass[notitlepage,12pt]{article}
%%%%%%%%%%%%%%%%%%%%%%%%%%%%%%%%%%%%%%%%%%%%%%%%%%%%%%%%%%%%%%%%%%%%%%%%%%%%%%%%%%%%%%%%%%%%%%%%%%%%%%%%%%%%%%%%%%%%%%%%%%%%%%%%%%%%%%%%%%%%%%%%%%%%%%%%%%%%%%%%%%%%%%%%%%%%%%%%%%%%%%%%%%%%%%%%%%%%%%%%%%%%%%%%%%%%%%%%%%%%%%%%%%%%%%%%%%%%%%%%%%%%%%%%%%%%
\usepackage{amsfonts}
\usepackage{amsmath}

\setcounter{MaxMatrixCols}{10}
%TCIDATA{OutputFilter=LATEX.DLL}
%TCIDATA{Version=5.00.0.2552}
%TCIDATA{Codepage=65001}
%TCIDATA{<META NAME="SaveForMode" CONTENT="1">}
%TCIDATA{Created=Tuesday, November 13, 2012 12:43:14}
%TCIDATA{LastRevised=Friday, August 19, 2016 00:30:15}
%TCIDATA{<META NAME="GraphicsSave" CONTENT="32">}
%TCIDATA{<META NAME="DocumentShell" CONTENT="Standard LaTeX\Standard LaTeX Article">}
%TCIDATA{Language=American English}
%TCIDATA{CSTFile=40 LaTeX article.cst}

\newtheorem{theorem}{Theorem}
\newtheorem{acknowledgement}[theorem]{Acknowledgement}
\newtheorem{algorithm}[theorem]{Algorithm}
\newtheorem{axiom}[theorem]{Axiom}
\newtheorem{case}[theorem]{Case}
\newtheorem{claim}[theorem]{Claim}
\newtheorem{conclusion}[theorem]{Conclusion}
\newtheorem{condition}[theorem]{Condition}
\newtheorem{conjecture}[theorem]{Conjecture}
\newtheorem{corollary}[theorem]{Corollary}
\newtheorem{criterion}[theorem]{Criterion}
\newtheorem{definition}[theorem]{Definition}
\newtheorem{example}[theorem]{Example}
\newtheorem{exercise}[theorem]{Exercise}
\newtheorem{lemma}[theorem]{Lemma}
\newtheorem{notation}[theorem]{Notation}
\newtheorem{problem}[theorem]{Problem}
\newtheorem{proposition}[theorem]{Proposition}
\newtheorem{remark}[theorem]{Remark}
\newtheorem{solution}[theorem]{Solution}
\newtheorem{summary}[theorem]{Summary}
\newenvironment{proof}[1][Proof]{\noindent\textbf{#1.} }{\ \rule{0.5em}{0.5em}}

\begin{document}

\title{Notes on Competitiveness and Carbon Taxes}
\author{Kortum and Weisbach \\
%EndAName
Yale and University of Chicago}
\maketitle

\begin{abstract}
Blah blah blah
\end{abstract}

\section{Introduction}

Our goal is to develop analytic models to study carbon carbon taxes in an
international setting. We begin in section 2 by extending results from our
earlier work, Elliott et. al. (2010). This model allows us to evaluate a
consumption and extraction tax. We then turn to a more sophisticated model
in section 3, which includes a manufacturing sector. In this model we can
study a production tax together with a consumption and extraction tax. Our
main question is what combination of taxes is unilaterally optimal for a
country, given a goal for reducing global carbon emissions.

\section{Simple Model}

There are two countries, home (henceforth $\mathcal{H}$) and foreign
(henceforth $\mathcal{F}$), endowed with fossil fuel deposits ($E$ and $%
E^{\ast }$) and labor ($L$ and $L^{\ast }$, measured in efficiency units).
Note that variables related to $\mathcal{F}$ are denoted with a $^{\ast }$.
Fuel deposits are extracted using labor to produce energy (the $e$-good) and
carbon emissions. Each country can also produce another good (the $l$-good)
using only labor. Consumers have preferences over the $e$-good and the $l$%
-good. Both are costlessly traded. Nonetheless, after-tax prices of the $e$%
-good may differ across countries due to carbon taxes. The law of one price
holds for the $l$-good.

\subsection{Basic Setup}

We parameterize the model with a Cobb-Douglas production functions for
extraction of the $e$-good and constant elasticity of substitution
preferences over the two goods. We allow the parameters to differ by country.

\subsubsection{Production}

Labor's share in the extraction of energy is $\beta $:%
\begin{equation}
Q_{e}=\left( L_{e}/\beta \right) ^{\beta }E^{1-\beta }.  \label{e production}
\end{equation}%
(We derive this production function from more primitive assumptions about
extraction in the Appendix.) Production of the $l$-good in a given country
is given by:%
\begin{equation*}
Q_{l}=L_{l}.
\end{equation*}%
Since we measure labor in efficiency units, this formulation can capture
differences in $l$-sector productivity across countries.

Consider the problem of a representative $e$-good extraction firm in $%
\mathcal{H}$ that owns energy deposits $E$. Facing a price for output of $%
p_{e}$ and a cost of labor $w$, it solves:%
\begin{equation*}
\max_{l_{e}}\left\{ p_{e}q_{e}-wl_{e}\right\} ,
\end{equation*}%
subject to the production function:%
\begin{equation*}
q_{e}=\left( l_{e}/\beta \right) ^{\beta }E^{1-\beta }.
\end{equation*}%
The solution is to hire labor:%
\begin{equation*}
L_{e}=\beta \left( \frac{p_{e}}{w}\right) ^{1/(1-\beta )}E
\end{equation*}%
and to extract a quantity:%
\begin{equation*}
Q_{e}=\left( \frac{p_{e}}{w}\right) ^{\beta /(1-\beta )}E.
\end{equation*}%
The elasticity of energy supply is $\varepsilon _{S}=\beta /(1-\beta )$.

The problem is the same in $\mathcal{F}$, replacing $\beta $ with $\beta
^{\ast }$, $w$ with $w^{\ast }$, $E$ with $E^{\ast }$, and taking into
account any differences in the price faced by producers there. The solution
in $\mathcal{F}$ is $L_{e}^{\ast }$ and $Q_{e}^{\ast }$.

\subsubsection{Consumption}

Preferences are parameterized as:%
\begin{equation}
U(C_{e},C_{l})=\left( \alpha ^{1/\sigma }C_{e}^{(\sigma -1)/\sigma }+\left(
1-\alpha \right) ^{1/\sigma }C_{l}^{(\sigma -1)/\sigma }\right) ^{\sigma
/(\sigma -1)}.  \label{preferences}
\end{equation}%
The parameter $\sigma $ is the elasticity of substitution in consumption
between the $e$-good and the $l$-good while $\alpha $ governs the share of
spending on the $e$-good. In the special case of $\sigma =1$ preferences
simplify to the Cobb-Douglas case considered in Elliott et. al. (2010):%
\begin{equation*}
U=\left( C_{e}\right) ^{\alpha }\left( C_{l}\right) ^{1-\alpha }.
\end{equation*}%
For $\sigma =0$ we get Leontief preferences:%
\begin{equation*}
U=\min \left\{ \frac{C_{e}}{\alpha },\frac{C_{l}}{1-\alpha }\right\} .
\end{equation*}%
By analogy to the parameter $\beta \in \lbrack 0,1]$ on the supply side, we
can define $\rho =\sigma /(\sigma +1)$ on the demand side, a parameter which
also stays in the unit interval.

Consumer income $Y$ comes from labor $wL$, from rents on energy deposits $rE$%
, and from any tax revenue $T$, which is rebated lump-sum. Consider the
problem of a representative consumer in $\mathcal{H}$. Taking income $Y$ and
prices $p_{e}$ and $p_{l}$ as given, the consumer solves:%
\begin{equation*}
\max_{c_{e},c_{l}}\left\{ U(c_{e},c_{l})\right\} ,
\end{equation*}%
subject to the budget constraint:%
\begin{equation*}
Y=p_{e}c_{e}+p_{l}c_{l}.
\end{equation*}%
The solution is:%
\begin{equation*}
C_{e}=\frac{\alpha }{p_{e}}\left( \frac{p_{e}}{p}\right) ^{-(\sigma -1)}Y,
\end{equation*}%
and%
\begin{equation*}
C_{l}=\frac{1-\alpha }{p_{l}}\left( \frac{p_{l}}{p}\right) ^{-(\sigma -1)}Y,
\end{equation*}%
with the price index given by:%
\begin{equation}
p=\left( \alpha p_{e}^{-(\sigma -1)}+(1-\alpha )p_{l}^{-(\sigma -1)}\right)
^{-1/(\sigma -1)}.  \label{aggregate price index}
\end{equation}

The problem is the same in $\mathcal{F}$, replacing $\sigma $ with $\sigma
^{\ast }$, $\alpha $ with $\alpha ^{\ast }$, $Y$ with $Y^{\ast }$, and
taking into account any differences in the price $p_{e}$ faced by consumers
there. The solutions in $\mathcal{F}$ are $C_{e}^{\ast }$ and $C_{l}^{\ast }$%
, along with $p^{\ast }$.

\subsubsection{Identities and Constraints}

Labor is perfectly mobile between producing the $e$-good $L_{e}$ and the $l$%
-good, $L_{l}$:%
\begin{equation*}
L=L_{e}+L_{l}.
\end{equation*}%
Labor and deposits are immobile across countries. The resource constraints
are:%
\begin{equation*}
Q_{e}^{W}=Q_{e}+Q_{e}^{\ast }=C_{e}+C_{e}^{\ast },
\end{equation*}%
and%
\begin{equation*}
Q_{l}^{W}=Q_{l}+Q_{l}^{\ast }=C_{l}+C_{l}^{\ast }.
\end{equation*}%
Emissions arise from extracting and consuming energy so global carbon
emissions are equal to $Q_{e}^{W}$.

\subsubsection{Equilibrium}

We consider a competitive equilibrium, consisting of wages and prices such
that consumers maximize utility, producers maximize profit, and markets
clear. It is convenient to take $\mathcal{F}$'s wage $w^{\ast }=1$ as the
numeraire. Furthermore, we will restrict parameters so that the equilibrium
is one in which both countries produce some of the $l$-good. In that
situation wages are the same in both countries so that $p_{l}=w=w^{\ast }=1$%
. (We can later derive the conditions under which both countries do, in
fact, produce the $l$-good.)

\subsection{Carbon Taxes}

We introduce both extraction taxes and consumption taxes on the $e$-good. We
let $t_{e}$ denote the level of an ad-valorem extraction tax in $\mathcal{H}$
(and $t_{e}^{\ast }$ in $\mathcal{F}$).\footnote{%
In Elliott et. al. (2010), we referred to this extraction tax as a
production tax.} We let $t_{c}$ denote the level of an ad-valorem
consumption tax in $\mathcal{H}$ (and $t_{c}^{\ast }$ in $\mathcal{F}$).

\subsubsection{Taxes on Extraction}

With an extraction tax $t_{e}$, the producer of the $e$-good in $\mathcal{H}$
faces a net price of $p_{e}/(1+t_{e})$, so that:%
\begin{equation*}
Q_{e}=\left( \frac{p_{e}}{1+t_{e}}\right) ^{\beta /(1-\beta )}E.
\end{equation*}%
(Note that we have imposed the condition that $w=1$.)

After-tax revenue in the sector is:%
\begin{equation*}
R_{e}=\frac{p_{e}}{1+t_{e}}Q_{e}=\left( \frac{p_{e}}{1+t_{e}}\right)
^{1/(1-\beta )}E,
\end{equation*}%
which is shared between wages payed to labor:%
\begin{equation*}
wL_{e}=L_{e}=\beta R_{e}
\end{equation*}%
and rents payed to owners of energy deposits:%
\begin{equation*}
rE=\left( 1-\beta \right) R_{e}.
\end{equation*}%
Total pre-tax revenue is:%
\begin{equation*}
p_{e}Q_{e}=\left( 1+t_{e}\right) R_{e}=\left( 1+t_{e}\right) \left( \frac{%
p_{e}}{1+t_{e}}\right) ^{1/(1-\beta )}E,
\end{equation*}%
split between after-tax revenue $R_{e}$ and extraction tax revenue:%
\begin{equation*}
T_{e}=\tau R_{e}=t_{e}\left( \frac{p_{e}}{1+t_{e}}\right) ^{1/(1-\beta )}E.
\end{equation*}

\subsubsection{Taxes on Consumption}

With a consumption tax $t_{c}$, consumers face an after-tax price for the $e$%
-good of $(1+t_{c})p_{e}$, hence after-tax spending is:%
\begin{equation*}
(1+t_{c})p_{e}C_{e}=\alpha \left( \frac{(1+t_{c})p_{e}}{p}\right) ^{-(\sigma
-1)}Y,
\end{equation*}%
with the price of a consumption bundle given by:%
\begin{equation}
p=\left( \alpha \left( (1+t_{c})p_{e}\right) ^{-(\sigma -1)}+(1-\alpha
)\right) ^{-1/(\sigma -1)}.  \label{price index with tax}
\end{equation}%
(Note that we have imposed the condition that $p_{l}=1$.)

Total spending on the $e$-good can be expressed as:%
\begin{equation}
(1+t_{c})p_{e}C_{e}=\frac{\alpha \left( (1+t_{c})p_{e}\right) ^{-(\sigma -1)}%
}{\alpha \left( (1+t_{c})p_{e}\right) ^{-(\sigma -1)}+1-\alpha }Y,
\label{e consumption}
\end{equation}%
split between pre-tax consumption spending $p_{e}C_{e}$ and tax revenue:%
\begin{equation*}
T_{c}=t_{c}p_{e}C_{e}=\frac{t_{c}}{1+t_{c}}\frac{\alpha \left(
(1+t_{c})p_{e}\right) ^{-(\sigma -1)}}{\alpha \left( (1+t_{c})p_{e}\right)
^{-(\sigma -1)}+1-\alpha }Y.
\end{equation*}

\subsubsection{Income}

So far we have simply taken a country's GDP $Y$ (and $Y^{\ast }$ in $%
\mathcal{F}$) as given. Adding together labor income, rents on the $E$
factor, and tax revenue from the two carbon taxes, income in $\mathcal{H}$
is:%
\begin{equation*}
Y=wL+rE+T_{e}+T_{c}=L+\left( 1-\beta +t_{e}\right) R_{e}+t_{c}p_{e}C_{e}.
\end{equation*}

Certain shares of GDP prove useful in simplifying and calibrating the model.
We denote after-tax spending on energy consumption as a share of GDP by:%
\begin{equation*}
\pi _{c}=\frac{\left( 1+t_{c}\right) p_{e}C_{e}}{Y}.
\end{equation*}%
Similarly, we denote pre-tax revenue of the energy extraction sector as a
share of GDP is:%
\begin{equation*}
\pi _{e}=\frac{p_{e}Q_{e}}{Y}.
\end{equation*}%
Returns to the $L$ factor as a share of GDP are:%
\begin{equation*}
\pi _{L}=\frac{wL}{Y}=\frac{L}{Y}.
\end{equation*}

These shares are connected in several ways. First, to satisfy our assumption
that both countries produce the $l$-good we need $L_{e}<L$, hence:%
\begin{equation*}
\frac{\beta }{1+t_{e}}\pi _{e}<\pi _{L}.
\end{equation*}%
Second, for the income identity we need:%
\begin{equation*}
1-\pi _{L}-\frac{1-\beta }{1+t_{e}}\pi _{e}=\frac{t_{e}}{1+t_{e}}\pi _{e}+%
\frac{t_{c}}{1+t_{c}}\pi _{c},
\end{equation*}%
hence:%
\begin{equation*}
1-\pi _{L}=\frac{1-\beta +t_{e}}{1+t_{e}}\pi _{e}+\frac{t_{c}}{1+t_{c}}\pi
_{c}
\end{equation*}%
The analogs of these two expressions must also hold in $\mathcal{F}$.

\subsubsection{Equilibrium with Taxes}

We can calculate the competitive equilibrium with taxes by finding the world
energy price $p_{e}$ that clears the energy market. By Walras law the $l$%
-good market will also clear. Equivalently, we can find the price that
equate pre-tax consumption spending with pre-tax revenue of the energy
sector:%
\begin{equation}
p_{e}\left( Q_{e}+Q_{e}^{\ast }\right) =p_{e}\left( C_{e}+C_{e}^{\ast
}\right) .  \label{market clearing}
\end{equation}%
Using our share expressions, market clearing becomes:%
\begin{equation*}
\pi _{e}Y+\pi _{e}^{\ast }Y^{\ast }=\frac{\pi _{c}Y}{1+t_{c}}+\frac{\pi
_{c}^{\ast }Y^{\ast }}{1+t_{c}^{\ast }}.
\end{equation*}%
Thus, at the equilibrium price we have an additional restriction on these
shares:%
\begin{equation*}
\pi _{c}^{\ast }=\left( \frac{1+t_{c}^{\ast }}{Y^{\ast }}\right) \left( \pi
_{e}Y+\pi _{e}^{\ast }Y^{\ast }-\frac{\pi _{c}Y}{1+t_{c}}\right) .
\end{equation*}

In general a solution needs to be computed numerically. But before setting
aside the analytics, we can show some results about taxation and can solve
the model in special cases.

\subsubsection{Tax Equivalence Results}

We consider two results.

\paragraph{Uniform Taxes with Identical Preferences}

First we show that if demand parameters are identical across countries the
effect on global emissions of a uniform carbon tax is the same whether it is
applied to extraction or consumption. There is an income effect, however,
which is why we need to impose identical demand parameters. If the tax is
applied to consumption it leaves the country that is a net importer of the $%
e $-good with higher income, while if applied to extraction it leaves that
country with lower income.

To demonstrate this result, start with a uniform consumption tax $t_{c}$ on
the $e$-good and no extraction tax. Imposing $\alpha =\alpha ^{\ast }$ and $%
\sigma =\sigma ^{\ast }$, the global equilibrium conditions can be written
as:%
\begin{equation*}
\frac{\alpha }{1+t_{c}}\frac{\left( (1+t_{c})p_{e}\right) ^{-(\sigma -1)}}{%
\alpha \left( (1+t_{c})p_{e}\right) ^{-(\sigma -1)}+(1-\alpha )}%
Y^{W}=p_{e}^{1/(1-\beta )}E+p_{e}^{1/(1-\beta ^{\ast })}E^{\ast },
\end{equation*}%
where world income can be expressed as:%
\begin{eqnarray*}
Y^{W} &=&L^{W}+\left( 1-\beta \right) p_{e}^{1/(1-\beta )}E+\left( 1-\beta
^{\ast }\right) p_{e}^{1/(1-\beta ^{\ast })}E^{\ast }+tp_{e}C_{e}^{W} \\
&=&L^{W}+\left( 1-\beta +t_{c}\right) p_{e}^{1/(1-\beta )}E+\left( 1-\beta
^{\ast }+t_{c}\right) p_{e}^{1/(1-\beta ^{\ast })}E^{\ast }
\end{eqnarray*}

Let's conjecture that the equilibrium price with an extraction tax is $\bar{p%
}_{e}=(1+t_{c})p_{e}$. Substituting this new price into the conditions
above, we get:%
\begin{equation*}
\frac{\alpha }{1+t_{c}}\frac{\bar{p}_{e}^{-(\sigma -1)}}{\alpha \bar{p}%
_{e}^{-(\sigma -1)}+(1-\alpha )}Y^{W}=\left( \frac{\bar{p}_{e}}{1+t_{c}}%
\right) ^{1/(1-\beta )}E+\left( \frac{\bar{p}_{e}}{1+t_{c}}\right)
^{1/(1-\beta ^{\ast })}E^{\ast },
\end{equation*}%
with%
\begin{equation*}
Y^{W}=L^{W}+\left( 1-\beta +t_{c}\right) \left( \frac{\bar{p}_{e}}{1+t_{c}}%
\right) ^{1/(1-\beta )}E+\left( 1-\beta ^{\ast }+t_{c}\right) \left( \frac{%
\bar{p}_{e}}{1+t_{c}}\right) ^{1/(1-\beta ^{\ast })}E^{\ast },
\end{equation*}%
exactly the global equilibrium conditions for a uniform extraction tax $%
t_{e}=t_{c}$ on extraction of the $e$-good and no consumption tax.

While the equilibrium price is higher with the extraction tax, global energy
supply is the same in either case:%
\begin{eqnarray*}
Q_{e}^{W} &=&p_{e}^{\beta /(1-\beta )}E+\left( 1-\beta ^{\ast }\right)
p_{e}^{\beta ^{\ast }/(1-\beta ^{\ast })}E^{\ast } \\
&=&\left( \frac{\bar{p}_{e}}{1+t_{c}}\right) ^{\beta /(1-\beta )}E+\left( 
\frac{\bar{p}_{e}}{1+t_{c}}\right) ^{\beta ^{\ast }/(1-\beta ^{\ast
})}E^{\ast }.
\end{eqnarray*}%
The two policies achieve the same goal in reducing global emissions, but
they do not lead to the same global distribution of income. In the case of a
uniform consumption tax, income in $\mathcal{H}$ is:%
\begin{eqnarray*}
Y &=&L+\left( 1-\beta \right) p_{e}^{1/(1-\beta )}E+tp_{e}C_{e} \\
&=&L+\left( 1-\beta +t_{c}\right) p_{e}^{1/(1-\beta )}E+t_{c}\left(
p_{e}C_{e}-p_{e}Q_{e}\right) .
\end{eqnarray*}%
In the case of a uniform extraction tax, it is:%
\begin{equation*}
Y=L+\left( 1-\beta +t_{c}\right) \left( \frac{\bar{p}_{e}}{1+t_{c}}\right)
^{1/(1-\beta )}E.
\end{equation*}%
Thus, if $\mathcal{H}$ is a net importer of the $e$-good, it will have
higher income under a uniform consumption tax. Since a consumer's after-tax
price of energy is the same in either case, welfare is higher in $\mathcal{H}
$ under a consumption tax if it is a net importer of the $e$-good.

\paragraph{Home Taxes and Border Adjustments}

Now consider the same comparison, but with a tax only in $\mathcal{H}$. We
can drop the restriction of identical preferences, since all the action is
now in $\mathcal{H}$. First, consider a tax $t_{c}$ on consumption of the $e$%
-good. At an equilibrium price $p_{e}$, the quantity of energy extracted in $%
\mathcal{H}$ is:%
\begin{equation*}
Q_{e}=p_{e}^{\beta /(1-\beta )}E
\end{equation*}%
and consumption spending is:%
\begin{equation*}
p_{e}C_{e}=\frac{1}{1+t_{c}}\frac{\left( (1+t_{c})p_{e}\right) ^{-(\sigma
-1)}}{\left( (1+t_{c})p_{e}\right) ^{-(\sigma -1)}+(1-\alpha )/\alpha }Y,
\end{equation*}%
where%
\begin{equation*}
Y=L+\left( 1-\beta \right) p_{e}^{1/(1-\beta )}E+\frac{t_{c}}{1+t_{c}}\frac{%
\left( (1+t_{c})p_{e}\right) ^{-(\sigma -1)}}{\left( (1+t_{c})p_{e}\right)
^{-(\sigma -1)}+(1-\alpha )/\alpha }Y.
\end{equation*}%
Extraction in $\mathcal{F}$ is:%
\begin{equation*}
Q_{e}^{\ast }=p_{e}^{\beta ^{\ast }/(1-\beta ^{\ast })}E^{\ast }
\end{equation*}%
and consumption spending in $\mathcal{F}$ is:%
\begin{equation*}
p_{e}C_{e}^{\ast }=\frac{p_{e}^{-(\sigma ^{\ast }-1)}}{p_{e}^{-(\sigma
^{\ast }-1)}+(1-\alpha ^{\ast })/\alpha ^{\ast }}Y^{\ast },
\end{equation*}%
where%
\begin{equation*}
Y^{\ast }=L^{\ast }+\left( 1-\beta ^{\ast }\right) p_{e}^{1/(1-\beta ^{\ast
})}.
\end{equation*}%
Let's try to reinterpret these conditions as an extraction tax $t_{e}=t_{c}$
with border tax adjustments (and no consumption tax). Letting $\bar{p}%
_{e}=(1+t_{c})p_{e}$ be the equilibrium price in $\mathcal{H}$ under the
extraction tax interpretation, we have:%
\begin{equation*}
Q_{e}=\left( \frac{\bar{p}_{e}}{1+t_{c}}\right) ^{\beta /(1-\beta )}E
\end{equation*}%
and%
\begin{equation*}
\bar{p}_{e}C_{e}=\alpha \frac{\bar{p}_{e}^{-(\sigma -1)}}{\alpha \bar{p}%
_{e}^{-(\sigma -1)}+(1-\alpha )}Y,
\end{equation*}%
where%
\begin{eqnarray*}
Y &=&L+\left( 1-\beta \right) \left( \frac{\bar{p}_{e}}{1+t_{c}}\right)
^{1/(1-\beta )}E+\frac{t_{c}}{1+t_{c}}\alpha \frac{\bar{p}_{e}^{-(\sigma -1)}%
}{\alpha \bar{p}_{e}^{-(\sigma -1)}+(1-\alpha )}Y \\
&=&L+\left( 1-\beta +t_{c}\right) \left( \frac{\bar{p}_{e}}{1+t_{c}}\right)
^{1/(1-\beta )}E+\frac{t_{c}}{1+t_{c}}\left( \bar{p}_{e}C_{e}-\bar{p}%
_{e}Q_{e}\right) \\
&=&L+\left( 1-\beta +t_{c}\right) \left( \frac{\bar{p}_{e}}{1+t_{c}}\right)
^{1/(1-\beta )}E+t_{c}\left( p_{e}C_{e}-p_{e}Q_{e}\right) .
\end{eqnarray*}%
The last term of the last expression shows the revenue (or cost) of the
border tax adjustment, at rate $t$. Thus, the border tax adjustment is
either a tariff on $\mathcal{H}$'s net imports of the $e$-good or a tax
rebate on $\mathcal{H}$'s net exports of the $e$-good. The border taxes are
collected on net imports or paid on net exports of the $e$-good valued at
the original equilibrium price $p_{e}$. The price of the $e$-good in $%
\mathcal{F}$ is $p_{e}=\bar{p}_{e}/(1+t)$, which is lower than the price in $%
\mathcal{H}$ due to the border tax adjustment. It is the price $\mathcal{F}$
would face if $\mathcal{H}$ had imposed a consumption tax.

\subsubsection{Special Cases}

To get analytical tractability, we consider the case of common parameters
and unit elastic demand, as in Elliott et. al. (2010). Suppose $\beta =\beta
^{\ast }$, $\sigma =\sigma ^{\ast }=1$, $\alpha =\alpha ^{\ast }$, $%
t_{e}=t_{e}^{\ast }\geq 0$, and $t_{c}=t_{c}^{\ast }=0$. The right-hand side
of (\ref{market clearing}) becomes:%
\begin{equation*}
p_{e}Q_{e}^{W}=\left( 1+t_{e}\right) ^{-\beta /(1-\beta )}p_{e}^{1/(1-\beta
)}E^{W}.
\end{equation*}%
The left-hand side is:%
\begin{equation*}
p_{e}C_{e}^{W}=\frac{\alpha }{1+t_{c}-\alpha t_{c}}\left( L^{W}+\left(
1-\beta +t_{e}\right) \left( \frac{p_{e}}{1+t_{e}}\right) ^{1/(1-\beta
)}E^{W}\right) .
\end{equation*}%
Equating the two, we obtain the equilibrium price:%
\begin{equation*}
\frac{p_{e}}{1+t_{e}}=\left( \frac{\alpha }{\left( 1+t_{e}\right) \left(
1+t_{c}\right) \left( 1-\alpha \right) +\alpha \beta }\frac{L^{W}}{E^{W}}%
\right) ^{1-\beta }.
\end{equation*}%
Thus, world energy supply is:%
\begin{equation*}
Q_{e}^{W}=\left( \frac{\alpha }{\left( 1+t_{e}\right) \left( 1+t_{c}\right)
\left( 1-\alpha \right) +\alpha \beta }\right) ^{\beta }\left( L^{W}\right)
^{\beta }\left( E^{W}\right) ^{1-\beta }.
\end{equation*}%
In the case of $t_{c}=0$, we match the result in Elliott et. al. (2010). We
can also verify in this setting that a consumption tax can acheive the same
outcome as a extraction tax at the same ad valorem rate. In general the two
taxes augment each other to lower world energy supply.

\subsection{Solving the Model}

To solve the model in a realistic setting, we calibrate it to aggregate
income $Y$ (and $Y^{\ast }$ in $\mathcal{F}$) together with the shares (in
GDP) of energy extraction and consumption, $\pi _{e}$ and $\pi _{c}$ (and $%
\pi _{e}^{\ast }$ and $\pi _{c}^{\ast }$ in $\mathcal{F}$). By calibrating
to these values, we will not need to know $\alpha $, $L$, or $E$ (nor their
analogs in $\mathcal{F}$).

We then consider how the equilibrium values of endogenous variables change
with respect to a change in carbon taxes. All such changes are denoted with
a \textquotedblleft hat\textquotedblright , indicating the proportional
change from the baseline (e.g. for a variable whose baseline value is $x$,
if its value becomes $x^{\prime }$ when carbon taxes are changed, then we
denote $\hat{x}=x^{\prime }/x$).

\subsubsection{Equilibrium Energy Price Change}

First consider total pre-tax revenue in $\mathcal{H}$'s energy sector after
a change in taxes:%
\begin{eqnarray*}
\pi _{e}^{\prime }Y^{\prime } &=&p_{e}^{\prime }Q_{e}^{\prime }=\left(
1+t_{e}^{\prime }\right) ^{-\beta /(1-\beta )}p_{e}^{\prime 1/(1-\beta )}E \\
&=&\pi _{e}Y\left( \frac{1+t_{e}^{\prime }}{1+t_{e}}\right) ^{-\beta
/(1-\beta )}\hat{p}_{e}^{1/(1-\beta )}.
\end{eqnarray*}%
It follows that the change in pre-tax energy revenue is:%
\begin{equation}
\hat{\pi}_{e}\hat{Y}=\left( \frac{1+t_{e}^{\prime }}{1+t_{e}}\right)
^{-\beta /(1-\beta )}\hat{p}_{e}^{1/(1-\beta )}.  \label{pi_y_hat}
\end{equation}%
(The same equation works for $\mathcal{F}$, with *'s in the appropriate
places.)

The change in the share of after-tax consumption spending on the $e$-good is:%
\begin{equation}
\hat{\pi}_{c}=\frac{\left( \frac{1+t_{c}^{\prime }}{1+t_{c}}\right)
^{-(\sigma -1)}\hat{p}_{e}^{-(\sigma -1)}}{\pi _{c}\left( \frac{%
1+t_{c}^{\prime }}{1+t_{c}}\right) ^{-(\sigma -1)}\hat{p}_{e}^{-(\sigma
-1)}+1-\pi _{c}}.  \label{pi_c_hat}
\end{equation}%
(Again, there is an analogous equation for $\mathcal{F}$.)

After a change in taxes, income is:%
\begin{eqnarray*}
Y^{\prime } &=&L+\left( 1-\beta \right) R_{e}^{\prime }+t_{e}^{\prime
}R_{e}^{\prime }+t_{c}^{\prime }p_{e}^{\prime }C_{e}^{\prime } \\
&=&L+\frac{1-\beta }{1+t_{e}^{\prime }}\pi _{e}^{\prime }Y^{\prime }+\frac{%
t_{e}^{\prime }}{1+t_{e}^{\prime }}\pi _{e}^{\prime }Y^{\prime }+\frac{%
t_{c}^{\prime }}{1+t_{c}^{\prime }}\pi _{c}^{\prime }Y^{\prime }.
\end{eqnarray*}%
Thus:%
\begin{equation}
\hat{Y}=\pi _{L}+\frac{1-\beta +t_{e}^{\prime }}{1+t_{e}^{\prime }}\pi _{e}%
\hat{\pi}_{e}\hat{Y}+\frac{t_{c}^{\prime }}{1+t_{c}^{\prime }}\pi _{c}\hat{%
\pi}_{c}\hat{Y},  \label{Y_hat2}
\end{equation}%
(Again, there is an analogous equation for the change in income in $\mathcal{%
F}$.)

Finally, the goods market clearing condition is:%
\begin{equation*}
p_{e}^{\prime }Q_{e}^{\prime }+p_{e}^{\prime }Q_{e}^{\ast \prime
}=p_{e}^{\prime }C_{e}^{\prime }+p_{e}^{\prime }C_{e}^{\ast \prime },
\end{equation*}%
or%
\begin{equation}
\pi _{e}Y\hat{\pi}_{e}\hat{Y}+\pi _{e}^{\ast }Y^{\ast }\hat{\pi}_{e}^{\ast }%
\hat{Y}^{\ast }=\frac{\pi _{c}Y\hat{\pi}_{c}\hat{Y}}{1+t_{c}^{\prime }}+%
\frac{\pi _{c}^{\ast }Y^{\ast }\hat{\pi}_{c}^{\ast }\hat{Y}^{\ast }}{%
1+t_{c}^{\ast \prime }}  \label{market clearing hat}
\end{equation}%
(This equation applies for the world, so there is no analog in $\mathcal{F}$%
.)

Given initial and final levels of carbon taxes in $\mathcal{H}$ ($t_{c}$, $%
t_{c}^{\prime }$, $t_{e}$, $t_{e}^{\prime }$) and $\mathcal{F}$, we can
solve the system (including the analogs in $\mathcal{F}$) consisting of (\ref%
{pi_y_hat}), (\ref{pi_c_hat}), (\ref{Y_hat2}), and (\ref{market clearing hat}%
) for $\hat{\pi}_{e}$, $\hat{\pi}_{e}^{\ast }$, $\hat{\pi}_{c}$, $\hat{\pi}%
_{c}^{\ast }$, $\hat{Y}$, $\hat{Y}^{\ast }$, and the change in the
equilibrium world price of energy $\hat{p}_{e}$. We need only the baseline
data on $Y$, $\pi _{c}$, $\pi _{L}$, and $\pi _{e}$ (along with $Y^{\ast }$, 
$\pi _{c}^{\ast }$, $\pi _{L}^{\ast }$, and $\pi _{e}^{\ast }$ in $\mathcal{F%
}$) together with parameters $\beta $ and $\sigma $ ($\beta ^{\ast }$ and $%
\sigma ^{\ast }$ in $\mathcal{F}$).

\subsubsection{Carbon Leakage}

Having determined the equilibrium change in the world energy price, we can
calculate the change in world energy production. For cases in which only $%
\mathcal{H}$ imposes a tax, we can compute several forms of carbon leakage.
For an increase in $\mathcal{H}$'s extraction tax, extraction leakage is the
increase in $\mathcal{F}$'s extraction of energy as a share of $\mathcal{H}$%
's reduction in extraction of energy:%
\begin{equation*}
l_{Q}=\frac{\left( Q_{e}^{\ast \prime }-Q_{e}^{\ast }\right) }{\left(
Q_{e}-Q_{e}^{\prime }\right) }=\left( \frac{\hat{Q}_{e}^{\ast }-1}{1-\hat{Q}%
_{e}}\right) \frac{\pi _{e}^{\ast }Y^{\ast }}{\pi _{e}Y},
\end{equation*}%
where%
\begin{equation*}
\hat{Q}_{e}=\frac{\hat{\pi}_{e}\hat{Y}}{\hat{p}_{e}},
\end{equation*}%
and 
\begin{equation*}
\hat{Q}_{e}^{\ast }=\frac{\hat{\pi}_{e}^{\ast }\hat{Y}^{\ast }}{\hat{p}_{e}}.
\end{equation*}%
For an increase in $\mathcal{H}$'s consumption tax, consumption leakage is
the increase in $\mathcal{F}$'s energy consumption as a share of $\mathcal{H}
$'s reduction in energy consumption:%
\begin{equation*}
l_{C}=\frac{\left( C_{e}^{\ast \prime }-C_{e}^{\ast }\right) }{\left(
C_{e}-C_{e}^{\prime }\right) }=\left( \frac{\hat{C}_{e}^{\ast }-1}{1-\hat{C}%
_{e}}\right) \frac{\pi _{c}^{\ast }Y^{\ast }/(1+t_{c}^{\ast })}{\pi
_{c}Y/(1+t_{c})},
\end{equation*}%
where%
\begin{equation*}
\hat{C}_{e}=\frac{\hat{\pi}_{c}\hat{Y}}{\hat{p}_{e}\frac{1+t_{c}^{\prime }}{%
1+t_{c}}},
\end{equation*}%
and 
\begin{equation*}
\hat{C}_{e}^{\ast }=\frac{\hat{\pi}_{c}^{\ast }\hat{Y}^{\ast }}{\hat{p}_{e}%
\frac{1+t_{c}^{\ast \prime }}{1+t_{c}^{\ast }}}.
\end{equation*}

\subsection{Tax Policy to Achieve a Global Emission Goal}

We want to compare tax policies in $\mathcal{H}$ that achieve a set goal of
emission reduction for the world. To keep this exercise as simple as
possible, we will assume that $\sigma =\sigma ^{\ast }$, $\beta =\beta
^{\ast }$, and $t_{e}=t_{c}=0$ (i.e. we'll start from a world without taxes).

\subsubsection{Supply Side}

Let $\hat{Q}_{e}^{W}=G<1$ be the goal set for reducing world emissions.
Looking at energy extraction, for any $t_{e}^{\prime }\geq 0$, the change in
energy price that would achieve that goal satisfies:%
\begin{eqnarray*}
G &=&\frac{Q_{e}\hat{Q}_{e}+Q_{e}^{\ast }\hat{Q}_{e}^{\ast }}{Q_{e}^{W}} \\
&=&\frac{Q_{e}\left( 1+t_{e}^{\prime }\right) ^{-\beta /(1-\beta )}\hat{p}%
_{e}^{\beta /(1-\beta )}+Q_{e}^{\ast }\hat{p}_{e}^{\beta /(1-\beta )}}{%
Q_{e}^{W}} \\
&=&\omega _{e}\left( 1+t_{e}^{\prime }\right) ^{-\beta /(1-\beta )}\hat{p}%
_{e}^{\beta /(1-\beta )}+\left( 1-\omega _{e}\right) \hat{p}_{e}^{\beta
/(1-\beta )},
\end{eqnarray*}%
where%
\begin{equation*}
\omega _{e}=\frac{\pi _{e}Y}{\pi _{e}Y+\pi _{e}^{\ast }Y^{\ast }}=\frac{%
p_{e}Q_{e}}{p_{e}Q_{e}^{W}}
\end{equation*}%
is $\mathcal{H}$'s share of world energy extraction. Given $G$ (which will
be held fixed in this problem), we can express the required change in the
energy price, as dictated by the supply side $\hat{p}_{e}^{S}$, as a
function of the extraction tax:%
\begin{equation}
\hat{p}_{e}^{S}(t_{e}^{\prime })=\left( \frac{G}{\omega _{e}\left(
1+t_{e}^{\prime }\right) ^{-\beta /(1-\beta )}+\left( 1-\omega _{e}\right) }%
\right) ^{\left( 1-\beta \right) /\beta }.  \label{energy price change}
\end{equation}

Note that the required change in energy prices moves in the same direction
as $G$. Given the tax rate on extraction, a more ambitious goal for reducing
emissions (a smaller $G$) requires that producers face a steeper decline in
the energy price. Our focus is on how the extraction tax alters incentives
for producing energy. We see from (\ref{energy price change}) that, given $G$%
, a higher extraction tax in $\mathcal{H}$ means energy prices fall by less,
or even rise. As the tax rate rises from $0$ to $\infty $, the energy price
change increases from $\hat{p}_{e}^{S}(0)=G^{(1-\beta )/\beta }$ to $\hat{p}%
_{e}^{S}(\infty )=\left( 1-\omega _{e}\right) ^{-(1-\beta )/\beta }\hat{p}%
_{e}^{S}(0)$. The extent of this rise is governed by the fraction of world
energy extraction $\omega _{e}$ initially taking place in $\mathcal{H}$, as
that determines the size of the tax base.

\subsubsection{Demand Side}

Turning to the consumption side, we want to calculate the change in the
energy price that, along with a tax on energy consumption $t_{c}^{\prime }$
(in $\mathcal{H}$), reduces world consumption by the factor $G$. The decline
in global consumption can be expressed as:%
\begin{equation*}
G=\hat{C}_{e}^{W}=\frac{C_{e}\hat{C}_{e}+C_{e}^{\ast }\hat{C}_{e}^{\ast }}{%
C_{e}^{W}}=\omega _{c}\hat{C}_{e}+\left( 1-\omega _{c}\right) \hat{C}%
_{e}^{\ast },
\end{equation*}%
where%
\begin{equation*}
\omega _{c}=\frac{\pi _{c}Y}{\pi _{c}Y+\pi _{c}^{\ast }Y^{\ast }}=\frac{%
p_{e}C_{e}}{p_{e}C_{e}^{W}}.
\end{equation*}%
We can write:%
\begin{equation}
\hat{C}_{e}^{W}(t_{c}^{\prime },t_{e}^{\prime },\hat{p}_{e})=\omega _{c}%
\frac{\hat{\pi}_{c}(t_{c}^{\prime },\hat{p}_{e})}{\left( 1+t_{c}^{\prime
}\right) \hat{p}_{e}}\hat{Y}(t_{c}^{\prime },t_{e}^{\prime },\hat{p}%
_{e})+\left( 1-\omega _{c}\right) \frac{\hat{\pi}_{c}^{\ast }(\hat{p}_{e})}{%
\hat{p}_{e}}\hat{Y}^{\ast }(\hat{p}_{e}),  \label{demand side}
\end{equation}%
where $\hat{\pi}_{c}$ is given by (\ref{pi_c_hat}) with $t_{c}=0$:%
\begin{equation*}
\hat{\pi}_{c}(t_{c}^{\prime },\hat{p}_{e})=\frac{\left( 1+t_{c}^{\prime
}\right) ^{-(\sigma -1)}\hat{p}_{e}^{-(\sigma -1)}}{\pi _{c}\left(
1+t_{c}^{\prime }\right) ^{-(\sigma -1)}\hat{p}_{e}^{-(\sigma -1)}+1-\pi _{c}%
}
\end{equation*}%
and%
\begin{equation*}
\hat{\pi}_{c}^{\ast }(\hat{p}_{e})=\frac{\hat{p}_{e}^{-(\sigma -1)}}{\pi
_{c}^{\ast }\hat{p}_{e}^{-(\sigma -1)}+1-\pi _{c}^{\ast }}.
\end{equation*}%
We now turn to the terms for the change in income.

Adjusting to lower global emissions generates income changes in both
countries. Changes in the energy price affect revenue in the energy sector.
Furthermore, in $\mathcal{H}$ there is income generated by tax revenue. From
(\ref{Y_hat2}) and (\ref{pi_y_hat}), the change in $\mathcal{H}$'s income is:%
\begin{equation*}
\hat{Y}=\pi _{L}+\frac{1-\beta +t_{e}^{\prime }}{1+t_{e}^{\prime }}\pi
_{e}\left( 1+t_{e}^{\prime }\right) ^{-\beta /(1-\beta )}\hat{p}%
_{e}^{1/(1-\beta )}+\frac{t_{c}^{\prime }}{1+t_{c}^{\prime }}\pi _{c}\hat{\pi%
}_{c}\hat{Y},
\end{equation*}%
or%
\begin{equation}
\hat{Y}(t_{c}^{\prime },t_{e}^{\prime },\hat{p}_{e})=\frac{1+t_{c}^{\prime }%
}{1+t_{c}^{\prime }-t_{c}^{\prime }\pi _{c}\hat{\pi}_{c}(t_{c}^{\prime },%
\hat{p}_{e})}\left( \pi _{L}+\frac{1-\beta +t_{e}^{\prime }}{1+t_{e}^{\prime
}}\pi _{e}\left( 1+t_{e}^{\prime }\right) ^{-\beta /(1-\beta )}\hat{p}%
_{e}^{1/(1-\beta )}\right) .  \label{Yhat tax}
\end{equation}%
Since carbon taxes do not generate any revenue abroad, the change in $%
\mathcal{F}$'s income is simply:%
\begin{equation}
\hat{Y}^{\ast }(\hat{p}_{e})=\pi _{L}^{\ast }+\left( 1-\beta \right) \pi
_{e}^{\ast }\hat{p}_{e}^{1/(1-\beta )}.  \label{Ystarhat tax}
\end{equation}

\subsubsection{Welfare}

Welfare in $\mathcal{H}$ is given by:%
\begin{equation*}
W=\frac{Y}{p},
\end{equation*}%
where we use equation (\ref{price index with tax}) for the aggregate price
level, $p$. The change in welfare, when taxes are introduced, is:%
\begin{equation*}
\hat{W}=\frac{W^{\prime }}{W}=\frac{\hat{Y}}{\hat{p}}.
\end{equation*}

We already have an expression (\ref{Yhat tax}) for $\hat{Y}$, so we just
need an expression for the change in the aggregate price level to evaluate
welfare. From (\ref{price index with tax}), the change in the price level
will be:%
\begin{equation*}
\hat{p}=\left( \frac{\alpha \left( (1+t_{c}^{\prime })p_{e}^{\prime }\right)
^{-(\sigma -1)}+(1-\alpha )}{\alpha p_{e}^{-(\sigma -1)}+1-\alpha }\right)
^{-1/(\sigma -1)},
\end{equation*}%
which simplifies to: 
\begin{equation*}
\hat{p}=\left( \pi _{c}\left( \left( 1+t_{c}^{\prime }\right) \hat{p}%
_{e}\right) ^{-(\sigma -1)}+(1-\pi _{c})\right) ^{-1/(\sigma -1)}.
\end{equation*}%
Thus, the change in welfare is:%
\begin{equation*}
\hat{W}=\frac{\left( \frac{\pi _{c}\left( 1+t_{c}^{\prime }\right)
^{-(\sigma -1)}\left( \hat{p}_{e}\right) ^{-(\sigma -1)}+1-\pi _{c}}{\pi
_{c}\left( 1+t_{c}^{\prime }\right) ^{-\sigma }\left( \hat{p}_{e}\right)
^{-(\sigma -1)}+1-\pi _{c}}\right) \left( \pi _{L}+\left( 1+t_{e}^{\prime
}-\beta \right) \pi _{e}\left( 1+t_{e}^{\prime }\right) ^{-1/(1-\beta )}\hat{%
p}_{e}^{1/(1-\beta )}\right) }{\left( \pi _{c}\left( \left( 1+t_{c}^{\prime
}\right) \hat{p}_{e}\right) ^{-(\sigma -1)}+(1-\pi _{c})\right) ^{-1/(\sigma
-1)}},
\end{equation*}%
or:%
\begin{eqnarray}
\hat{W}(t_{c}^{\prime },t_{e}^{\prime },\hat{p}_{e}) &=&\frac{\left( \pi
_{c}\left( 1+t_{c}^{\prime }\right) ^{-(\sigma -1)}\hat{p}_{e}^{-(\sigma
-1)}+(1-\pi _{c})\right) ^{\sigma /(\sigma -1)}}{\pi _{c}\left(
1+t_{c}^{\prime }\right) ^{-\sigma }\hat{p}_{e}^{-(\sigma -1)}+1-\pi _{c}}%
\times  \notag \\
&&\left( \pi _{L}+\left( 1+t_{e}^{\prime }-\beta \right) \pi _{e}\left(
1+t_{e}^{\prime }\right) ^{-1/(1-\beta )}\hat{p}_{e}^{1/(1-\beta )}\right) .
\label{What simple}
\end{eqnarray}%
Given $\hat{p}_{e}$, it is easy to show that $\hat{W}$ is decreasing in both 
$t_{c}^{\prime }$ and $t_{e}^{\prime }$.

\subsubsection{Optimization}

Since the goal $G$ for reduction in global emissions is fixed, we can
substitute $\hat{p}_{e}=$ $\hat{p}_{e}^{S}(t_{e}^{\prime })$ from (\ref%
{energy price change}) into (\ref{What simple}) to obtain an expression for
the change in wefare that depends only on tax rates:%
\begin{equation*}
\hat{W}\left( t_{c}^{\prime },t_{e}^{\prime },\hat{p}_{e}^{S}(t_{e}^{\prime
})\right) =f(t_{c}^{\prime },t_{e}^{\prime }).
\end{equation*}%
We want to choose tax rates that maximize this expression, subject to the
constraint that world demand for energy also falls by the factor $G$. We can
write this constraint as:%
\begin{equation*}
G=\hat{C}_{e}^{W}(t_{c}^{\prime },t_{e}^{\prime },\hat{p}_{e}^{S}(t_{e}^{%
\prime }))=g(t_{c}^{\prime },t_{e}^{\prime }),
\end{equation*}%
where $\hat{C}_{e}^{W}(t_{c}^{\prime },t_{e}^{\prime },\hat{p}_{e})$ is
given in (\ref{demand side}). The optimization problem reduces to maximizing
the lagrangian:%
\begin{equation*}
\mathcal{L=}f(t_{c}^{\prime },t_{e}^{\prime })-\lambda \left[
G-g(t_{c}^{\prime },t_{e}^{\prime })\right] ,
\end{equation*}%
where $\lambda $ is the lagrangian multiplier.

\section{Model with a Manufacturing Sector}

Up to this point we have modeled the extraction and consumption of
carbon-based energy, but not its use in industry and its embodiment in
tradable manufactured goods. With our Simple Model we could discuss
extraction and consumption taxes, but not production taxes. To address this
shortcoming, we now add a manufacturing sector to Simple Model, with two-way
trade in differentiated manufactures.

To keep the analysis as compact as possible, we assume the two countries $%
\mathcal{H}$ and $\mathcal{F}$ differ only in their endowment of labor $L$
vs. $L^{\ast }$, fossil fuel deposits $E$ vs. $E^{\ast }$, preference
parameters $\alpha $ vs. $\alpha ^{\ast }$, and labor share in energy
extraction $\beta $ vs. $\beta ^{\ast }$. The reason for maintaining
heterogeneity in $\beta $ is to capture differences between a country like
Saudi Arabia with energy that is cheap to extract (small $\beta $) and a
country like Canada with energy that is costly to extract (large $\beta $).

\subsection{Basic Setup}

Manufactures come in a continuum of varieties, $j\in \lbrack 0,1]$. These
varieties enter preferences through a symmetric aggregator, with constant
elasticity of substitution $\rho $, to form the manufactured good, the $m$%
-good. The $m$-good, in turn, enters preferences in Cobb-Douglas combination
with the $e$-good (to form the composit $c$-good), with share $\eta $ on the 
$m$-good. (Simple Model reemerges when $\eta =0$.) Preferences can thus be
represented by the following utility funtction:%
\begin{equation}
U(C_{e},C_{l},C_{m})=\left( \alpha ^{1/\sigma }C_{c}^{(\sigma -1)/\sigma
}+\left( 1-\alpha \right) ^{1/\sigma }C_{l}^{(\sigma -1)/\sigma }\right)
^{\sigma /(\sigma -1)},  \label{Utility}
\end{equation}%
where consumption of the $c$-good is:%
\begin{equation*}
C_{c}=\left( \frac{C_{m}}{\eta }\right) ^{\eta }\left( \frac{C_{e}}{1-\eta }%
\right) ^{1-\eta }
\end{equation*}%
and consumption of the $m$-good is:%
\begin{equation*}
C_{m}=\left( \int_{0}^{1}C_{m}(j)^{\left( \rho -1\right) /\rho }dj\right)
^{\rho /\left( \rho -1\right) }.
\end{equation*}%
Other aspects of the model, such as the extraction of energy and the
production of the $l$-good, are identical to Simple Model. For example, the
aggregate price index can be taken from (\ref{aggregate price index}) with $%
p_{c}$ in place of $p_{e}$:%
\begin{equation}
p=\left( \alpha p_{c}^{-(\sigma -1)}+(1-\alpha )p_{l}^{-(\sigma -1)}\right)
^{-1/(\sigma -1)}.  \label{aggregate price index II}
\end{equation}%
The new features are in the modeling of production and trade in manufactures.

\subsubsection{Trade in Manufactures}

Our formulation of production and trade in manufactures is a version of the
Ricardian model introduced by Dornbusch, Fisher, and Samuelson (1977).
Individual varieties of the $m$-good are costlessly traded. The production
function for manufacturing variety $j$ in $\mathcal{H}$ is:%
\begin{equation*}
Q_{m}(j)=A(j)\left( \frac{L_{m}(j)}{\gamma }\right) ^{\gamma }\left( \frac{%
M_{e}(j)}{1-\gamma }\right) ^{1-\gamma },
\end{equation*}%
where $A(j)$ is $\mathcal{H}$'s productivity in variety $j$, $M_{e}$ is
energy input, and $\gamma $ is labor's share. We imagine that many
price-taking producers have access to the technology to produce each variety 
$j$. In $\mathcal{F}$ the production function has the same form, with
productivity $A^{\ast }(j)$.

Labor is perfectly mobile, within each country, across all sectors and
varieties. The labor constraint in $\mathcal{H}$ is:%
\begin{equation*}
L_{e}+L_{l}+\int_{0}^{1}L_{m}(j)dj=L.
\end{equation*}%
A similar condition holds in $\mathcal{F}$.

We parameterize productivity for each variety as:%
\begin{equation*}
A(j)=\frac{A}{j^{1/\theta }}
\end{equation*}%
in $\mathcal{H}$ and:%
\begin{equation*}
A^{\ast }(j)=\frac{A^{\ast }}{\left( 1-j\right) ^{1/\theta }}
\end{equation*}%
in $\mathcal{F}$. Here, the parameters $A$ and $A^{\ast }$ capture absolute
advantage in $\mathcal{H}$ and $\mathcal{F}$. Looking at the relative
productivity of the two countries in producing variety $j$:%
\begin{equation}
R(j)=\frac{A(j)}{A^{\ast }(j)}=\frac{A}{A^{\ast }}\left( \frac{j}{1-j}%
\right) ^{-1/\theta }.  \label{relative productivity}
\end{equation}%
For $j<j^{\prime }$, $\mathcal{H}$ has a comparative advantage in variety $j$
and $\mathcal{F}$ in $j^{\prime }$. The parameter $\theta $ captures
(inversely) the strength of comparative advantage. As $\theta \rightarrow
\infty $ relative productivity does not vary across varieties. As will
become apparent below, we need to restrict $\theta >\rho -1$.

\paragraph{Specialization in Manufactures}

Since there are no trade costs, specialization in production of manufactures
is detached from country-level demand. We can therefore determine which
country produces which variety without solving the entire model.

Due to trade in energy, prices $p_{e}$ are the same in each country. A
bundle of inputs costs $w^{\gamma }p_{e}^{1-\gamma }$ in $\mathcal{H}$ and $%
w^{\ast \gamma }p_{e}^{1-\gamma }$ in $\mathcal{F}$. Hence $\mathcal{H}$
will produce varieties $j$ for which $R(j)\geq \left( w/w^{\ast }\right)
^{\gamma }$ and $\mathcal{F}$ will produce the rest. In other words, $%
\mathcal{H}$ produces all varieties in the interval $[0,\bar{j}]$ and $%
\mathcal{F}$ in the interval $(\bar{j},1]$ where:%
\begin{equation}
\bar{j}=\frac{A^{\theta }w^{-\gamma \theta }}{A^{\theta }w^{-\gamma \theta
}+A^{\ast \theta }w^{\ast -\gamma \theta }}.  \label{jbar}
\end{equation}%
(Whether $\mathcal{H}$ or $\mathcal{F}$ produces variety $\bar{j}$ doesn't
matter since it is only one point on a continuum.)

Given specialization, we can solve for the competitive price of each
manufactured variety. For $j\leq \bar{j}$, the variety is produced in $%
\mathcal{H}$, so that:%
\begin{equation}
p_{m}(j)=\frac{w^{\gamma }p_{e}^{1-\gamma }}{A(j)},  \label{pm(j)}
\end{equation}%
while for $j>\bar{j}$, the variety is produced in $\mathcal{F}$, with:%
\begin{equation*}
p_{m}(j)=\frac{w^{\ast \gamma }p_{e}^{1-\gamma }}{A^{\ast }(j)}.
\end{equation*}%
With no trade cost we have the law of one price; these variety-level prices
apply for consumers in either country.

\paragraph{Price Index for Manufactues}

The price index for the $m$-good is:%
\begin{eqnarray}
p_{m} &=&\left( \int_{0}^{1}p_{m}(j)^{-\left( \rho -1\right) }dj\right)
^{-1/\left( \rho -1\right) }  \notag \\
&=&\left( \int_{0}^{\bar{j}}\left( w^{\gamma }p_{e}^{1-\gamma }/A(j)\right)
^{-\left( \rho -1\right) }dj+\int_{\bar{j}}^{1}\left( w^{\ast \gamma
}p_{e}^{1-\gamma }/A^{\ast }(j)\right) ^{-\left( \rho -1\right) }dj\right)
^{-1/\left( \rho -1\right) }.  \label{manufacturing price index definition}
\end{eqnarray}%
To solve this integral and to understand the properties of this price index,
consider varieties produced in $\mathcal{H}$. The lowest price of any
variety produced in $\mathcal{H}$ approaches $0$ and the highest price is
given by (\ref{pm(j)}) evaluated at $j=\bar{j}$:%
\begin{equation}
\bar{p}=p_{m}(\bar{j})=\frac{\bar{j}^{1/\theta }w^{\gamma }p_{e}^{1-\gamma }%
}{A},  \label{pbar}
\end{equation}%
which can be inverted to get:%
\begin{equation*}
\bar{j}=\left( \frac{\bar{p}A}{w^{\gamma }p_{e}^{1-\gamma }}\right) ^{\theta
}.
\end{equation*}%
If we change the variable of integration from $j$ to $p_{m}(j)$ in the
expression for the average of $p_{m}(j)^{-(\rho -1)}$ across varieties
produced in $\mathcal{H}$, we get a simple result:%
\begin{eqnarray*}
\frac{1}{\bar{j}}\int_{0}^{\bar{j}}p_{m}(j)^{-\left( \rho -1\right) }dj
&=&\left( \frac{\bar{p}A}{w^{\gamma }p_{e}^{1-\gamma }}\right) ^{-\theta
}\int_{0}^{\bar{p}}p^{-(\rho -1)}\theta \left( \frac{pA}{w^{\gamma
}p_{e}^{1-\gamma }}\right) ^{\theta -1}\frac{A}{w^{\gamma }p_{e}^{1-\gamma }}%
dp \\
&=&\bar{p}^{-\theta }\int_{0}^{\bar{p}}\theta p^{\theta -\rho }dp=\frac{%
\theta }{\theta -\left( \rho -1\right) }\bar{p}^{-\left( \rho -1\right) }.
\end{eqnarray*}%
Now consider varieties produced by $\mathcal{F}$. The prices of these goods
also vary from $0$ to $\bar{p}$, where for $\mathcal{F}$ we can write:%
\begin{equation*}
1-\bar{j}=\left( \frac{\bar{p}A^{\ast }}{w^{\ast \gamma }p_{e}^{1-\gamma }}%
\right) ^{\theta }.
\end{equation*}%
Thus, the average of $p_{m}(j)^{-(\rho -1)}$ across varieties produced in $%
\mathcal{F}$ is:%
\begin{eqnarray*}
\frac{1}{1-\bar{j}}\int_{\bar{j}}^{1}p_{m}(j)^{-\left( \rho -1\right) }dj
&=&\left( \frac{\bar{p}A^{\ast }}{w^{\ast \gamma }p_{e}^{1-\gamma }}\right)
^{-\theta }\int_{0}^{\bar{p}}p^{-(\rho -1)}\theta \left( \frac{pA^{\ast }}{%
w^{\ast \gamma }p_{e}^{1-\gamma }}\right) ^{\theta -1}\frac{A^{\ast }}{%
w^{\ast \gamma }p_{e}^{1-\gamma }}dp \\
&=&\frac{\theta }{\theta -\left( \rho -1\right) }\bar{p}^{-\left( \rho
-1\right) },
\end{eqnarray*}%
exactly the same as for $\mathcal{H}$. Since these averages are the same,
independent of where the varieties are produced, the overall price index (%
\ref{manufacturing price index definition}) can be expressed as:%
\begin{equation*}
p_{m}=\phi \bar{p},
\end{equation*}%
where 
\begin{equation*}
\phi =\left( \frac{\theta }{\theta -\left( \rho -1\right) }\right)
^{-1/(\rho -1)},
\end{equation*}%
is finite if $\theta >\rho -1$ as we have assumed.\footnote{%
If preferences over varieties are Cobb Douglas ($\rho =1$) then $\phi
=e^{-1/\theta }$. Our result, about a particular average of prices being
independent of which country produces the good in equilibrium, is a special
case of a more general result. It is easy to show that, whether produced in $%
H$ or $F$, the fraction of varieties sold at a price $p_{m}(j)\leq p$, which
we can treat as the price distribution, is $\left( p/\bar{p}\right) ^{\theta
}$.} Substituting in for $\bar{p}$ from (\ref{pbar}) and applying (\ref{jbar}%
), we get: 
\begin{eqnarray*}
p_{m} &=&\phi w^{\gamma }p_{e}^{1-\gamma }\frac{1}{A}\left( \frac{A^{\theta
}w^{-\gamma \theta }}{A^{\theta }w^{-\gamma \theta }+A^{\ast \theta }w^{\ast
-\gamma \theta }}\right) ^{1/\theta } \\
&=&\phi p_{e}^{1-\gamma }\left( A^{\theta }w^{-\gamma \theta }+A^{\ast
\theta }w^{\ast -\gamma \theta }\right) ^{-1/\theta }.
\end{eqnarray*}

Because the relevant average price is independent of where a variety is
produced, $\bar{j}$ given in (\ref{jbar}) is also the share of world
spending on the $m$-good devoted to producers in $\mathcal{H}$ and $1-\bar{j}
$ the share devoted to producers in $\mathcal{F}$. Note that this share does
not depend on on $\rho $.\footnote{%
Other than determining the value of $\phi $, the value of the parameter $%
\rho $ is irrelevant to all that follows.} The so-called trade elasticity,
giving the response of trade shares to factor costs, is $\theta $.

\subsubsection{Equilibrium}

As in Simple Model, we take $\mathcal{F}$'s wage $w^{\ast }=1$ as the
numeraire and consider only equilibria in which each country produces
positive quantities of the $l$-good. In such equilibria we have $%
p_{l}=w=w^{\ast }=1$. (Even though they equal $1$, we sometimes leave wages
in the formulas that follow for clarity.) Returning to earlier formulas, we
now have:%
\begin{equation}
\bar{j}=\frac{A^{\theta }}{A^{\theta }+A^{\ast \theta }}  \label{jbar no tax}
\end{equation}%
and%
\begin{equation}
p_{m}=\phi p_{e}^{1-\gamma }\left( A^{\theta }+A^{\ast \theta }\right)
^{-1/\theta }.  \label{pm no tax}
\end{equation}%
Thus, solving for the competitive equilibrium in this model reduces to
finding the price of energy $p_{e}$ together with the allocation of labor
across the three sectors in each country. We assume trade balance.

\paragraph{Income}

Turning to the determination of income, we have labor income $wL$ and income
from rents on energy resources:%
\begin{equation*}
rE=\left( 1-\beta \right) p_{e}Q_{e}=\left( 1-\beta \right)
p_{e}^{1/(1-\beta )}E.
\end{equation*}%
Thus, aggregate income in $\mathcal{H}$ is:%
\begin{equation}
Y=wL+\left( 1-\beta \right) p_{e}^{1/(1-\beta )}E,  \label{Y no tax}
\end{equation}%
exactly as in Simple Model. The analog holds for $\mathcal{F}$:%
\begin{equation}
Y^{\ast }=wL^{\ast }+\left( 1-\beta ^{\ast }\right) p_{e}^{1/(1-\beta ^{\ast
})}E^{\ast }.  \label{Y star no tax}
\end{equation}%
World income is the total payment to all primary factors $L$, $L^{\ast }$, $%
E $, and $E^{\ast }$:%
\begin{equation}
Y^{W}=L^{W}+\left( 1-\beta \right) p_{e}^{1/(1-\beta )}E+\left( 1-\beta
^{\ast }\right) p_{e}^{1/(1-\beta ^{\ast })}E^{\ast }.  \label{Y^W no tax}
\end{equation}

\paragraph{Demand}

With trade balance each country's income is also its total spending
(measured in terms of the numeraire, the $l$-good, whose price is $p_{l}=1$%
). The $e$-good and $m$-good combine with a Cobb-Douglas aggregator into
what we have called the composite or $c$-good. The price of the $c$-good is:%
\begin{equation}
p_{c}=p_{c}^{\ast }=p_{m}^{\eta }p_{e}^{1-\eta }=\phi ^{\eta }p_{e}^{1-\eta
\gamma }\left( A^{\theta }+A^{\ast \theta }\right) ^{-\eta /\theta }.
\label{p_c no prod or cons taxes}
\end{equation}%
We are back in the demand system of Simple Model but with $p_{c}$ replacing $%
p_{e}$. In particular, defining:%
\begin{equation}
D(p)=\frac{\alpha p^{-\left( \sigma -1\right) }}{\alpha p^{-(\sigma
-1)}+(1-\alpha )},  \label{c spending share}
\end{equation}%
spending by $\mathcal{H}$ on the $c$-good is:%
\begin{equation*}
p_{c}C_{c}=D(p_{c})Y.
\end{equation*}%
We thus know how consumers allocate spending across the $c$-good and the $l$%
-good given prices. We also know that a fraction $\eta $ of spending on the $%
c$-good is allocated to the $m$-good, with the rest spent on the $e$-good.
In $\mathcal{F}$ we have:%
\begin{equation*}
D^{\ast }(p)=\frac{\alpha ^{\ast }p^{-\left( \sigma -1\right) }}{\alpha
^{\ast }p^{-(\sigma -1)}+(1-\alpha ^{\ast })}
\end{equation*}%
and:%
\begin{equation*}
p_{c}C_{c}^{\ast }=D^{\ast }(p_{c})Y^{\ast }.
\end{equation*}

\paragraph{Equilibrium Energy Price}

We now have all the ingredients to find the price of energy that equates
world supply and demand. Rather than equating quantities of energy, we will
equate the value of those quantities. World spending on energy consists of
world spending on consumption of the $e$-good plus world spending on energy
intermediates, which represent a fraction $1-\gamma $ of world spending on
the $m$-good. We can therefore express total spending on energy in terms of
spending on the composite good:%
\begin{eqnarray*}
p_{e}\left( C_{e}+C_{e}^{\ast }\right) +p_{e}\left( M_{e}+M_{e}^{\ast
}\right) &=&\left( 1-\eta \gamma \right) p_{c}C_{c}+\left( 1-\eta \gamma
\right) p_{c}C_{c}^{\ast } \\
&=&\left( 1-\eta \gamma \right) \left( D(p_{c})Y+D^{\ast }(p_{c})Y^{\ast
}\right) .
\end{eqnarray*}%
Equating the value of world demand to the value of world energy supply
(which is unchanged from Simple Model), the equilibrium price must satisfy:%
\begin{equation*}
p_{e}^{1/(1-\beta )}E+p_{e}^{1/(1-\beta ^{\ast })}E^{\ast }=\left( 1-\eta
\gamma \right) \left( D(p_{c})Y+D^{\ast }(p_{c})Y^{\ast }\right) ,
\end{equation*}%
with $p_{c}$ given by (\ref{p_c no prod or cons taxes}), $Y$ given by (\ref%
{Y no tax}), and $Y^{\ast }$ by (\ref{Y star no tax}).

\paragraph{Employment}

What about the allocation of labor? We'll start with employment in
extraction. Having determined the equilibrium energy price, we get: 
\begin{equation*}
L_{e}=\beta \left( \frac{p_{e}}{w}\right) ^{1/(1-\beta )}E=\beta
p_{e}^{1/(1-\beta )}E,
\end{equation*}%
and likewise for $L_{e}^{\ast }$.

In solving for the equilibrium energy price, we've also solved for world
income $Y^{W}$ and the price of the composite good, $p_{c}$. Thus, from the
demand system (with wages equal to $1$) we know that world demand for labor
in the $m$-sector is:%
\begin{equation*}
L_{m}^{W}=\gamma \eta \left( D(p_{c})Y+D^{\ast }(p_{c})Y^{\ast }\right) ,
\end{equation*}%
with a fraction $\bar{j}$ employed in $\mathcal{H}$, so that:%
\begin{equation*}
L_{m}=\ \frac{A^{\theta }}{A^{\theta }+A^{\ast \theta }}L_{m}^{W}.
\end{equation*}%
To support this equilibrium, we need positive employment in the $l$-sector
in both $\mathcal{H}$ and $\mathcal{F}$. Thus we require:%
\begin{equation*}
L>L_{e}+L_{m}
\end{equation*}%
and%
\begin{equation*}
L^{\ast }>L_{e}^{\ast }+L_{m}^{\ast }.
\end{equation*}%
As a check, let's calculate world employment in the $l$-sector:%
\begin{eqnarray*}
L_{l}^{W} &=&L^{W}-L_{e}^{W}-L_{m}^{W} \\
&=&L^{W}-\beta p_{e}^{1/(1-\beta )}E-\beta ^{\ast }p_{e}^{1/(1-\beta ^{\ast
})}E^{\ast }-\gamma \eta \left( D(p_{c})Y+D^{\ast }(p_{c})Y^{\ast }\right) \\
&=&Y^{W}-p_{e}^{1/(1-\beta )}E-p_{e}^{1/(1-\beta ^{\ast })}E^{\ast }-\gamma
\eta \left( D(p_{c})Y+D^{\ast }(p_{c})Y^{\ast }\right) \\
&=&Y^{W}-p_{e}M_{e}^{W}-p_{e}C_{e}^{W}-\gamma \eta \left( D(p_{c})Y+D^{\ast
}(p_{c})Y^{\ast }\right) \\
&=&Y^{W}-\left[ \left( 1-\gamma \right) \eta +\left( 1-\eta \right) +\gamma
\eta \right] \left( D(p_{c})Y+D^{\ast }(p_{c})Y^{\ast }\right) \\
&=&\left( 1-D(p_{c})\right) Y+\left( 1-D^{\ast }(p_{c})\right) Y^{\ast }.
\end{eqnarray*}%
Note that the last term is simply world spending on the $l$-good, all of
which is paid to labor $L_{l}^{W}$, thus confirming the logic of the model.

\subsection{Carbon Taxes}

We now have three levels at which to tax carbon: extraction of energy,
production (of manufactures), and consumption of energy (directly or as
embodied in manufactures). As a notational convention, we will treate $p_{e}$
and $p_{m}$ as world prices, while we treat the price of the $c$-good as
being potentially country specific, due to taxes.

The ad valorem extraction tax, $t_{e}$, is unchanged from our formulation in
the simple model. It has the effect of reducing the extraction of energy
where it is applied, reducing world energy supply, and increasing the energy
price.

The ad valorem production tax $t_{p}$ raises the cost of the energy
intermediate for manufacturers. Since direct consumption of the $e$-good can
be thought of as household production (driving a car to produce
transportation services or running a furnace to produce heating services),
we want the production tax to apply there as well. The production tax is
straightforward since it simply raises the price of energy (for all users of
energy) from $p_{e}$ to $(1+t_{p})p_{e}$.

In Simple Model we had an ad valorem tax $t_{c}$ on direct consumption of
energy, raising the price faced by consumers from $p_{e}$ to $(1+t_{c})p_{e}$%
. We now need to expand the tax base to also cover the energy embodied in
manufactures. The tax on consumption of the $m$-good is more complicated
since we want it to mimic the incentive effects, on both producers and
consumers, of the production tax. We discuss this issue below. The key
distinction between a production and a consumption tax arises when these
taxes are applied in only one country, say $\mathcal{H}$. In this case, the
consumption tax, unlike the production tax, applies to $\mathcal{H}$'s
imports of manufactures and does not apply to its manufacturing exports.

\subsubsection{Implementing the Consumption Tax}

Consider a firm producing variety $j$ of the $m$-good in $\mathcal{H}$. It
takes as given the competitive prices for this good, $p(j)$ in $\mathcal{H}$
and $p^{\ast }(j)$ in $\mathcal{F}$, inclusive of carbon taxes on
consumption. The consumption tax is imposed at an ad valorem rate of $t_{c}$
and $t_{c}^{\ast }$ on the value of energy embodied in the goods sold in $%
\mathcal{H}$ and $\mathcal{F}$, respectively. Consumers must purchase the
good at a discount, so that after paying the consumption tax the total cost
to a consumer in $\mathcal{H}$ is $p_{m}(j)$ and to a consumer in $\mathcal{F%
}$ is $p_{m}^{\ast }(j)$. (It simplifies notation to define the price of
manufactured goods inclusive of taxes.)

First consider the firm's sales to consumers in $\mathcal{H}$. The firm's
problem is to choose inputs of labor $l$ and energy $e$ so as to maximize
revenue per unit, taking into account that the consumption tax is not
revenue:%
\begin{equation*}
\max_{l,e}\left\{ \left( p_{m}(j)-t_{c}p_{e}e\right) -wl-p_{e}e\right\}
\end{equation*}%
subject to:%
\begin{equation*}
A(j)\left( \frac{l}{\gamma }\right) ^{\gamma }\left( \frac{e}{1-\gamma }%
\right) ^{1-\gamma }=1.
\end{equation*}%
Since the firm takes $p_{m}(j)$ as given, we can reformulate this problem as
cost minimization:%
\begin{equation*}
\min_{l,e}\left\{ wl+\left( 1+t_{c}\right) p_{e}e\right\}
\end{equation*}%
subject to%
\begin{equation*}
A(j)\left( \frac{l}{\gamma }\right) ^{\gamma }\left( \frac{e}{1-\gamma }%
\right) ^{1-\gamma }=1.
\end{equation*}%
The solution to this problem is a unit cost function:%
\begin{equation*}
c(j)=\frac{w^{\gamma }\left( \left( 1+t_{c}\right) p_{e}\right) ^{1-\gamma }%
}{A(j)}
\end{equation*}%
and associated input shares:%
\begin{equation*}
wl(j)=\gamma c(j)
\end{equation*}%
\begin{equation*}
\left( 1+t_{c}\right) p_{e}e(j)=\left( 1-\gamma \right) c(j).
\end{equation*}%
The competitive zero-profit condition is simply:%
\begin{equation*}
p_{m}(j)=c(j)=\left( 1+t_{c}\right) ^{1-\gamma }\frac{w^{\gamma
}p_{e}^{1-\gamma }}{A(j)}.
\end{equation*}

For the firm's sales to consumers in $\mathcal{F}$ the same derivation
applies, so that:%
\begin{equation*}
p_{m}^{\ast }(j)=\left( 1+t_{c}^{\ast }\right) ^{1-\gamma }\frac{w^{\gamma
}p_{e}^{1-\gamma }}{A(j)}.
\end{equation*}%
For a firm in $\mathcal{F}$ producing good $j^{\prime }$, the price of its
good (again, inclusive of the consumption tax) to consumers in $\mathcal{F}$
is:%
\begin{equation*}
p_{m}^{\ast }(j^{\prime })=\left( 1+t_{c}^{\ast }\right) ^{1-\gamma }\frac{%
w^{\ast \gamma }p_{e}^{1-\gamma }}{A^{\ast }(j)},
\end{equation*}%
and to consumers in $\mathcal{H}$ is:%
\begin{equation*}
p_{m}(j^{\prime })=\left( 1+t_{c}\right) ^{1-\gamma }\frac{w^{\ast \gamma
}p_{e}^{1-\gamma }}{A^{\ast }(j)}.
\end{equation*}

Note that the consumption tax will give producers the same incentives to
reduce energy inputs as would a production tax. Thus, it is not a simple ad
valorem tax on consumption. Rather, it requires information on the actual
energy used in production. Nonetheless, due to the Cobb-Douglas technology,
in equilibrium it will appear as an ad valorem tax. For sales in $\mathcal{H}
$ the ad valorem rate is $t_{m}$ on consumption of the $m$-good, with $%
1+t_{m}=$ $\left( 1+t_{c}\right) ^{1-\gamma }$. The calculation of tax
revenue differs from what would be obtained from an ad valorem tax at rate $%
t_{m}$, however. For example, the tax revenue generated in $\mathcal{H}$ on
consumption of variety $j$ produced in $\mathcal{H}$ is:%
\begin{equation*}
T_{c}(j)=t_{c}p_{e}M_{e}^{HH}(j)=\frac{t_{c}}{1+t_{c}}\left( 1-\gamma
\right) p_{m}(j)C_{m}(j),
\end{equation*}%
where $M_{e}^{HH}(j)$ represents the quantity of the energy intermediate
used by producers of $j$ in $\mathcal{H}$ to supply consumers in $\mathcal{H}
$.

A firm in $\mathcal{F}$ producing some variety $j^{\prime }$ to sell in $%
\mathcal{H}$ will face similar considerations due to the consumption tax.
The tax revenue generated in $\mathcal{H}$ on consumption of variety $%
j^{\prime }$ produced in $\mathcal{F}$ is%
\begin{equation*}
T_{c}(j^{\prime })=t_{c}p_{e}M_{e}^{HF}(j^{\prime })=\frac{t_{c}}{1+t_{c}}%
\left( 1-\gamma \right) p_{m}(j^{\prime })C_{m}(j^{\prime }),
\end{equation*}%
where $M_{e}^{HF}(j)$ represents the quantity of the energy intermediate
used by producers of $j^{\prime }$ in $\mathcal{F}$ to supply consumers in $%
\mathcal{H}$.

\subsubsection{Uniform Taxes}

Suppose both countries impose carbon taxes at the same rate. We will
continue to assume that the world equilibrium remains one in which both
countries continue to produce positive quantities of the $l$-good. We will
consider taxes one by one, but accomodate the possibility that an extraction
tax is imposed together with either a production or consumption tax.

\paragraph{Extraction Tax}

With an extraction tax, the extraction sector behaves as in Simple Model.
World extraction is:%
\begin{equation*}
Q_{e}^{W}=\left( \frac{p_{e}}{1+t_{e}}\right) ^{\beta /(1-\beta )}E+\left( 
\frac{p_{e}}{1+t_{e}}\right) ^{\beta ^{\ast }/(1-\beta ^{\ast })}E^{\ast }.
\end{equation*}%
After-tax revenue of $\mathcal{H}$ in the $e$-sector is:%
\begin{equation*}
R_{e}=\frac{p_{e}}{1+t_{e}}Q_{e}=\left( \frac{p_{e}}{1+t_{e}}\right)
^{1/(1-\beta )}E,
\end{equation*}%
generating tax revenue:%
\begin{equation*}
T_{e}=t_{e}R_{e}=t_{e}\left( \frac{p_{e}}{1+t_{e}}\right) ^{1/(1-\beta )}E.
\end{equation*}%
Parallel equations hold for $\mathcal{F}$. Given its effect on the
equilibrium price of energy, the extraction tax has no other effects that
interact with the production or consumption tax.

\paragraph{Production Tax}

As explained above, a production tax is just a tax on using energy for
either production or consumption. Energy goods now cost $p_{e}(1+t_{p})$ in
either country, whether used as intermediates by manufacturers or as
consumption goods by households.

With a tax on energy inputs, producers substitute away from them toward the
labor input. The cost of producing manufactures rises by the factor $%
(1+t_{p})^{1-\gamma }$ in both countries. Specialization in the production
of manufactures remains unchanged, with $\bar{j}$ still given by (\ref{jbar
no tax}). The price of manufactures rises to: 
\begin{equation*}
p_{m}=\phi (p_{e}\left( 1+t_{p}\right) )^{1-\gamma }\left( A^{\theta
}+A^{\ast \theta }\right) ^{-1/\theta }.
\end{equation*}%
The statements above treat $p_{e}$ as fixed. Of course the reduced demand
leads to a fall in the price of energy, dampening the effects.

The production tax also raises the price of energy consumed directly by
households. The price of the composite good becomes:%
\begin{equation}
p_{c}=p_{m}^{\eta }(\left( 1+t_{p}\right) p_{e})^{1-\eta }=\phi ^{\eta
}(\left( 1+t_{p}\right) p_{e})^{1-\eta \gamma }\left( A^{\theta }+A^{\ast
\theta }\right) ^{-\eta /\theta },  \label{pc uniform production tax}
\end{equation}%
Given $p_{c}$, spending on the $c$-good is as above. Hence consumption
spending on the $e$-good in $\mathcal{H}$ is:%
\begin{equation*}
(1+t_{p})p_{e}C_{e}=\left( 1-\eta \right) D(p_{c})Y,
\end{equation*}%
with the spending share $D(p_{c})$ given by (\ref{c spending share}). The
analog holds in $\mathcal{F}$.

Production tax revenue consists of the taxes collected from firms and those
collected from households. The total in $\mathcal{H}$ is:%
\begin{eqnarray*}
T_{p} &=&t_{p}p_{e}M_{e}^{WH}+t_{p}p_{e}C_{e} \\
&=&\frac{t_{p}}{1+t_{p}}\left[ \left( 1-\gamma \right) \frac{A^{\theta }}{%
A^{\theta }+A^{\ast \theta }}\eta \left( D(p_{c})Y+D^{\ast }(p_{c})Y^{\ast
}\right) +\left( 1-\eta \right) D(p_{c})Y\right] ,
\end{eqnarray*}%
Where $M_{e}^{WH}$ is the quantity of the energy input used by producers in $%
\mathcal{H}$ to supply the world. In $\mathcal{F}$:%
\begin{equation*}
T_{p}^{\ast }=\frac{t_{p}}{1+t_{p}}\left[ \left( 1-\gamma \right) \frac{%
A^{\ast \theta }}{A^{\theta }+A^{\ast \theta }}\eta \left( D(p_{c})Y+D^{\ast
}(p_{c})Y^{\ast }\right) +\left( 1-\eta \right) D(p_{c})Y^{\ast }\right] .
\end{equation*}%
Hence, tax revenue for the world is:%
\begin{equation*}
T_{p}^{W}=\frac{t_{p}}{1+t_{p}}\left( 1-\gamma \eta \right) \left(
D(p_{c})Y+D^{\ast }(p_{c})Y^{\ast }\right) .
\end{equation*}%
Income in $\mathcal{H}$ is given by:%
\begin{equation*}
Y=wL+rE+T_{e}+T_{p}=L+\left( 1-\beta +t_{e}\right) R_{e}+T_{p},
\end{equation*}%
while in $\mathcal{F}$:%
\begin{equation*}
Y^{\ast }=L^{\ast }+\left( 1-\beta ^{\ast }+t_{e}\right) R_{e}^{\ast
}+T_{p}^{\ast }.
\end{equation*}

\paragraph{Consumption Tax}

As explained above, a consumption tax raises the price faced by households
for both the $m$-good and the $e$-good. In considering a consumption tax at
rate $t_{c}$ we will set $t_{p}=0$ (below we consider a combination of the
two taxes). We put no restriction on $t_{e}$. Since $t_{p}=0$, we are back
to:%
\begin{equation*}
p_{m}=\phi (1+t_{c})^{1-\gamma }p_{e}^{1-\gamma }\left( A^{\theta }+A^{\ast
\theta }\right) ^{-1/\theta }.
\end{equation*}

A household faces prices $(1+t_{c})p_{e}$ and $p_{m}$ (recall that the price
for manufactures is inclusive of the consumption tax). The after-tax price
of the $c$-good is thus:%
\begin{equation*}
p_{c}=(1+t_{c})^{1-\eta }p_{m}^{\eta }p_{e}^{1-\eta }=\phi ^{\eta }\left(
(1+t_{c})p_{e}\right) ^{1-\eta \gamma }\left( A^{\theta }+A^{\ast \theta
}\right) ^{-\eta /\theta },
\end{equation*}%
which is the same as (\ref{pc uniform production tax}) if $t_{c}=t_{p}$.
Given $p_{c}$, spending by households on the $e$-good in $\mathcal{H}$ is:%
\begin{equation*}
(1+t_{c})p_{e}C_{e}=\left( 1-\eta \right) D(p_{c})Y.
\end{equation*}%
On the $m$-good households spend:%
\begin{equation*}
p_{m}C_{m}=\eta D(p_{c})Y,
\end{equation*}%
so that:%
\begin{equation*}
(1+t_{c})p_{e}M_{e}^{HH}=\left( 1-\gamma \right) p_{m}C_{m}=\left( 1-\gamma
\right) \eta D(p_{c})Y.
\end{equation*}%
Tax revenue in $\mathcal{H}$ is:%
\begin{eqnarray*}
T_{c} &=&\frac{t_{c}}{1+t_{c}}\left( 1-\gamma \right) \eta D(p_{c})Y+\frac{%
t_{c}}{1+t_{c}}\left( 1-\eta \right) D(p_{c})Y \\
&=&\frac{t_{c}}{1+t_{c}}\left( 1-\gamma \eta \right) D(p_{c})Y,
\end{eqnarray*}%
while for the world:%
\begin{equation*}
T_{c}^{W}=\frac{t_{c}}{1+t_{c}}\left( 1-\gamma \eta \right) \left(
D(p_{c})Y+D^{\ast }(p_{c})Y^{\ast }\right) .
\end{equation*}%
Income in $\mathcal{H}$ is given by:%
\begin{equation*}
Y=L+\left( 1-\beta +t_{e}\right) R_{e}+T_{c}.
\end{equation*}

\paragraph{Tax Equivalence}

Equilibrium amounts to a price $p_{e}$ that equates world supply and demand
for energy. The value of world supply, given a uniform extraction tax, is:%
\begin{equation*}
p_{e}Q_{e}^{W}=\left( 1+t_{e}\right) ^{-\beta /(1-\beta )}p_{e}^{1/(1-\beta
)}E+\left( 1+t_{e}\right) ^{-\beta ^{\ast }/(1-\beta ^{\ast
})}p_{e}^{1/(1-\beta ^{\ast })}E^{\ast },
\end{equation*}%
so at the equilbrium price:%
\begin{equation*}
p_{e}Q_{e}^{W}=p_{e}C_{e}^{W}+p_{e}M_{e}^{W},
\end{equation*}%
where:%
\begin{equation*}
M_{e}^{W}=M_{e}^{WH}+M_{e}^{WF},
\end{equation*}%
is total energy intermediates used in the world. For the demand side of this
equation, we want to show that a uniform production tax is equivalent to a
uniform consumption tax at rate $t_{c}=t_{p}=t$.

Under either a production tax or a consumption tax, the value of world
demand for energy by households is:%
\begin{equation*}
p_{e}C_{e}^{W}=\frac{1}{1+t}\left( 1-\eta \right) \left( D(p_{c})Y+D^{\ast
}(p_{c})Y^{\ast }\right) ,
\end{equation*}%
and the value of world demand for energy by firms is:%
\begin{equation*}
p_{e}M_{e}^{W}=\frac{1}{1+t}\left( 1-\gamma \right) \eta \left(
D(p_{c})Y+D^{\ast }(p_{c})Y^{\ast }\right) ,
\end{equation*}%
so that total demand is:%
\begin{equation*}
p_{e}Q_{e}^{W}=\frac{1}{1+t}\left( 1-\eta \gamma \right) \left(
D(p_{c})Y+D^{\ast }(p_{c})Y^{\ast }\right) ,
\end{equation*}%
with $p_{c}$, in either case, given by (\ref{pc uniform production tax}).
World income $Y^{W}$ under either tax is:%
\begin{equation*}
Y^{W}=L^{W}+\left( 1-\beta +t_{e}\right) R_{e}+\left( 1-\beta ^{\ast
}+t_{e}\right) R_{e}^{\ast }+T^{W},
\end{equation*}%
where total tax revenue from the production or consumption tax is:%
\begin{equation*}
T^{W}=\frac{t}{1+t}\left( 1-\gamma \eta \right) \left( D(p_{c})Y+D^{\ast
}(p_{c})Y^{\ast }\right)
\end{equation*}

Thus, the equilibrium price of energy is the same under a production tax at
rate $t_{p}=t$ or a consumption tax at rate $t_{c}=t$. There is a difference
in the distribution of tax revenue, however, since%
\begin{equation*}
T_{p}=\frac{t}{1+t}\left[ \left( 1-\gamma \right) \frac{A^{\theta }}{%
A^{\theta }+A^{\ast \theta }}\eta \left( D(p_{c})Y+D^{\ast }(p_{c})Y^{\ast
}\right) +\left( 1-\eta \right) D(p_{c})Y\right]
\end{equation*}%
need not equal%
\begin{equation*}
T_{c}=\frac{t}{1+t}\left( 1-\gamma \eta \right) D(p_{c})Y.
\end{equation*}%
Home will do better under a uniform production tax iff:%
\begin{equation*}
\frac{A^{\theta }}{A^{\theta }+A^{\ast \theta }}Y^{W}>Y,
\end{equation*}%
i.e. if it has a comparative advantage in manufactures.

\subsubsection{Taxes in Home}

Suppose only $\mathcal{H}$ taxes carbon. We will continue to assume that the
world equilibrium remains one in which both countries continue to produce
positive quantities of the $l$-good. We will consider taxes one by one, but
accomodate the possibility that an extraction tax is imposed together with
either a production or consumption tax. As above, we do not consider the
simultaneous imposition of both a production and consumption tax.

\paragraph{Extraction Tax}

With an extraction tax, the extraction sector behaves as in Simple Model.
Extraction in $\mathcal{H}$ is:%
\begin{equation*}
Q_{e}=\left( \frac{p_{e}}{1+t_{e}}\right) ^{\beta /(1-\beta )}E,
\end{equation*}%
generating after-tax revenue in the $e$-sector of:%
\begin{equation*}
R_{e}=\frac{p_{e}}{1+t_{e}}Q_{e}=\left( \frac{p_{e}}{1+t_{e}}\right)
^{1/(1-\beta )}E.
\end{equation*}%
Tax revenue is:%
\begin{equation*}
T_{e}=t_{e}R_{e}=t_{e}\left( \frac{p_{e}}{1+t_{e}}\right) ^{1/(1-\beta )}E.
\end{equation*}%
In $\mathcal{F}$ we have:%
\begin{equation*}
Q_{e}^{\ast }=p_{e}^{\beta ^{\ast }/(1-\beta ^{\ast })}E^{\ast }.
\end{equation*}%
Since there are no taxes in $\mathcal{F}$, we have:%
\begin{equation}
Y^{\ast }=L^{\ast }+\left( 1-\beta ^{\ast }\right) p_{e}^{1/(1-\beta ^{\ast
})}E^{\ast }.  \label{Y star}
\end{equation}%
Given its effect on the equilibrium price of energy, the extraction tax has
no other effects that interact with the production or consumption tax.

\paragraph{Production Tax}

As explained above, a production tax is just a tax on using energy for
either production or consumption. Energy goods now cost $p_{e}(1+t_{p})$
whether used as intermediates by manufacturers in $\mathcal{H}$ or as
consumption goods by households in $\mathcal{H}$.

The cost of producing manufactures in $\mathcal{H}$ rises by the factor $%
(1+t_{p})^{1-\gamma }$. Now $\mathcal{H}$ produces all varieties $j$ for
which:%
\begin{equation*}
R(j)\geq \left( 1+t_{p}\right) ^{1-\gamma },
\end{equation*}%
where relative productivity $R(j)$ is given by (\ref{relative productivity}%
). Thus $\mathcal{H}$ produces all varieties in the interval $[0,\bar{j}]$,
where now:%
\begin{equation}
\bar{j}=\frac{A^{\theta }\left( 1+t_{p}\right) ^{-\theta \left( 1-\gamma
\right) }}{A^{\theta }\left( 1+t_{p}\right) ^{-\theta \left( 1-\gamma
\right) }+A^{\ast \theta }}.  \label{jbar home production tax}
\end{equation}%
The production tax alters how countries specialize across varieties of
manufactures.

With a production tax, the maximum price of a manufacturing variety becomes:%
\begin{eqnarray*}
\bar{p} &=&\frac{\bar{j}^{1/\theta }p_{e}^{1-\gamma }\left( 1+t_{p}\right)
^{1-\gamma }}{A} \\
&=&p_{e}^{1-\gamma }\left( A^{\theta }\left( 1+t_{p}\right) ^{-\theta \left(
1-\gamma \right) }+A^{\ast \theta }\right) ^{-1/\theta }.
\end{eqnarray*}%
We still have that the manufacturing price index is $p_{m}=\phi \bar{p}$, so
that when we substitute out $\bar{p}$: 
\begin{equation*}
p_{m}=\phi p_{e}^{1-\gamma }\left( A^{\theta }\left( 1+t_{p}\right)
^{-\theta \left( 1-\gamma \right) }+A^{\ast \theta }\right) ^{-1/\theta }.
\end{equation*}%
Holding $p_{e}$ fixed, the production tax in $\mathcal{H}$ makes
manufactures more expensive to consumers in both countries.

The production tax also raises the price of energy consumed directly by
households. The after-tax price of the composite good in $\mathcal{H}$
becomes:%
\begin{equation}
p_{c}=p_{m}^{\eta }p_{e}^{1-\eta }(1+t_{p})^{1-\eta }=\phi ^{\eta
}p_{e}^{1-\eta \gamma }(1+t_{p})^{1-\eta }\left( A^{\theta }\left(
1+t_{p}\right) ^{-\theta \left( 1-\gamma \right) }+A^{\ast \theta }\right)
^{-\eta /\theta },  \label{p_c home production tax}
\end{equation}%
while in $\mathcal{F}$ it is:%
\begin{equation}
p_{c}^{\ast }=p_{m}^{\eta }p_{e}^{1-\eta }=\phi ^{\eta }p_{e}^{1-\eta \gamma
}\left( A^{\theta }\left( 1+t_{p}\right) ^{-\theta \left( 1-\gamma \right)
}+A^{\ast \theta }\right) ^{-\eta /\theta }.
\label{p_c star home production tax}
\end{equation}%
Given $p_{c}$, spending on the $c$-good is as above. Hence consumption
spending on the $e$-good is:%
\begin{equation*}
(1+t_{p})p_{e}C_{e}=\left( 1-\eta \right) D(p_{c})Y,
\end{equation*}%
with the spending share $D(p_{c})$ given by (\ref{c spending share}).
Production tax revenue consists of the taxes collected from firms and those
collected from households. The total is:%
\begin{eqnarray*}
T_{p} &=&t_{p}p_{e}M_{e}^{WH}+t_{p}p_{e}C_{e} \\
&=&\frac{t_{p}}{1+t_{p}}\left[ \left( 1-\gamma \right) \bar{j}\eta \left[
D(p_{c})Y+D^{\ast }(p_{c}^{\ast })Y^{\ast }\right] +\left( 1-\eta \right)
D(p_{c})Y\right] ,
\end{eqnarray*}%
with $\bar{j}$ given by (\ref{jbar home production tax}), and income in $%
\mathcal{H}$ given by:%
\begin{equation}
Y=wL+rE+T_{e}+T_{p}=L+\left( 1-\beta +t_{e}\right) R_{e}+T_{p}.
\label{Y home production tax}
\end{equation}

\paragraph{Consumption Tax}

As explained above, a consumption tax raises the price of both the $m$-good
and the $e$-good to households in $\mathcal{H}$. In considering a
consumption tax at rate $t_{c}$ we continue to assume $t_{p}=0$, turning
only later to the case of both taxes. We put no restriction on $t_{e}$.
Since $t_{p}=0$, we are back to:%
\begin{equation*}
p_{m}=\phi (1+t_{c})^{1-\gamma }p_{e}^{1-\gamma }\left( A^{\theta }+A^{\ast
\theta }\right) ^{-1/\theta }.
\end{equation*}

Since the household in $\mathcal{H}$ faces prices $(1+t_{c})p_{e}$ and $p_{m}
$, the after-tax price of the $c$-good is:%
\begin{eqnarray}
p_{c} &=&p_{m}^{\eta }\left( p_{e}(1+t_{c})\right) ^{1-\eta }  \notag \\
&=&\phi ^{\eta }p_{e}^{1-\eta \gamma }(1+t_{c})^{1-\eta \gamma }\left(
A^{\theta }+A^{\ast \theta }\right) ^{-\eta /\theta },
\label{p_c home consumption tax1}
\end{eqnarray}%
while the price in $\mathcal{F}$ is simply:%
\begin{equation}
p_{c}^{\ast }=\phi ^{\eta }p_{e}^{1-\eta \gamma }\left( A^{\theta }+A^{\ast
\theta }\right) ^{-\eta /\theta }.  \label{p_c star home consumption tax}
\end{equation}%
Given $p_{c}$, spending by households on the $e$-good is:%
\begin{equation*}
(1+t_{c})p_{e}C_{e}=\left( 1-\eta \right) D(p_{c})Y,
\end{equation*}%
while on the $m$-good:%
\begin{equation*}
p_{m}C_{m}=\eta D(p_{c})Y.
\end{equation*}%
Spending on the energy input in $\mathcal{H}$ for sales of manufactures in $%
\mathcal{H}$ is:%
\begin{equation*}
(1+t_{c})p_{e}M_{e}^{HH}=\left( 1-\gamma \right) \bar{j}\eta D(p_{c})Y.
\end{equation*}%
Since there is no consumption tax in $\mathcal{F}$, total spending on energy
inputs in $\mathcal{H}$ is:%
\begin{equation*}
p_{e}M_{e}^{WH}=p_{e}M_{e}^{HH}+p_{e}M_{e}^{FH}=\left( 1-\gamma \right) \bar{%
j}\eta \left[ \frac{1}{1+t_{c}}D(p_{c})Y+D^{\ast }(p_{c}^{\ast })Y^{\ast }%
\right] .
\end{equation*}%
Total spending on energy inputs in $\mathcal{F}$ is:%
\begin{equation*}
p_{e}M_{e}^{WF}=p_{e}M_{e}^{HF}+p_{e}M_{e}^{FF}=\left( 1-\gamma \right)
\left( 1-\bar{j}\right) \eta \left[ \frac{1}{1+t_{c}}D(p_{c})Y+D^{\ast
}(p_{c}^{\ast })Y^{\ast }\right] ,
\end{equation*}%
so that world spending on energy inputs is:%
\begin{equation*}
p_{e}M_{e}^{W}=\left( 1-\gamma \right) \eta \left[ \frac{1}{1+t_{c}}%
D(p_{c})Y+D^{\ast }(p_{c}^{\ast })Y^{\ast }\right] .
\end{equation*}

Tax revenue in $\mathcal{H}$ is:%
\begin{eqnarray*}
T_{c} &=&t_{c}p_{e}M_{e}^{HH}+t_{c}p_{e}M_{e}^{HF}+t_{c}p_{e}C_{e} \\
&=&\frac{t_{c}}{1+t_{c}}\left( 1-\gamma \right) \eta D(p_{c})Y+\frac{t_{c}}{%
1+t_{c}}\left( 1-\eta \right) D(p_{c})Y \\
&=&\frac{t_{c}}{1+t_{c}}\left( 1-\eta \gamma \right) D(p_{c})Y.
\end{eqnarray*}%
Income in $\mathcal{H}$ is given by:%
\begin{equation}
Y=L+\left( 1-\beta +t_{e}\right) R_{e}+T_{c}.  \label{Y home consumption tax}
\end{equation}

\paragraph{Equilibrium with Taxes}

Equilibrium amounts to a price $p_{e}$ that equates world supply and demand
for energy, both valued at world prices. The value of world supply is:%
\begin{equation*}
p_{e}Q_{e}^{W}=\left( 1+t_{e}\right) ^{-\beta /(1-\beta )}p_{e}^{1/(1-\beta
)}E+p_{e}^{1/(1-\beta ^{\ast })}E^{\ast }.
\end{equation*}%
At the equilbrium price we have:%
\begin{equation*}
p_{e}Q_{e}^{W}=p_{e}C_{e}^{W}+p_{e}M_{e}^{W}.
\end{equation*}%
For the demand side of this equation, we separately consider the case of the
production tax and the consumption tax.

Under a production tax in home, the value of world demand for energy by
households is:%
\begin{equation*}
p_{e}C_{e}^{W}=\left( 1-\eta \right) \left[ \frac{1}{1+t_{p}}%
D(p_{c})Y+D^{\ast }(p_{c}^{\ast })Y^{\ast }\right] ,
\end{equation*}%
and the value of world demand for energy by firms is:%
\begin{equation*}
p_{e}M_{e}^{W}=\left( 1-\gamma \right) \eta \left[ D(p_{c})Y+D^{\ast
}(p_{c}^{\ast })Y^{\ast }\right] \left( \frac{1}{1+t_{p}}\bar{j}+\left( 1-%
\bar{j}\right) \right) ,
\end{equation*}%
with $p_{c}$ given by (\ref{p_c home production tax}), $p_{c}^{\ast }$ given
by (\ref{p_c star home production tax}), $\bar{j}$ given by (\ref{jbar home
production tax}), $Y$ given by (\ref{Y home production tax}), and $Y^{\ast }$
given by (\ref{Y star}).

Under a consumption tax in home:%
\begin{equation*}
p_{e}C_{e}^{W}=\left( 1-\eta \right) \left[ \frac{1}{1+t_{c}}%
D(p_{c})Y+D^{\ast }(p_{c}^{\ast })Y^{\ast }\right]
\end{equation*}%
and%
\begin{equation*}
p_{e}M_{e}^{W}=\left( 1-\gamma \right) \eta \left[ \frac{1}{1+t_{c}}%
D(p_{c})Y+D^{\ast }(p_{c}^{\ast })Y^{\ast }\right] .
\end{equation*}%
Thus:%
\begin{equation*}
p_{e}C_{e}^{W}+p_{e}M_{e}^{W}=\left( 1-\eta \gamma \right) \left[ \frac{1}{%
1+t_{c}}D(p_{c})Y+D^{\ast }(p_{c}^{\ast })Y^{\ast }\right] ,
\end{equation*}%
with $p_{c}$ given by (\ref{p_c home consumption tax1}), $p_{c}^{\ast }$
given by (\ref{p_c star home consumption tax}), $Y$ given by (\ref{Y home
consumption tax}), and $Y^{\ast }$ given by (\ref{Y star}).

\paragraph{Border Adjustments}

We now show that a production tax at rate $t_{p}$, with border tax
adjustments (BTA's), is equivalent to a consumption tax at rate $t_{c}=t_{p}$%
. Starting with the production tax, let's now add a tax on the value of
energy embodied in $\mathcal{H}$'s imports of the $m$-good, at rate $t_{p}$,
and a tax rebate on the value of energy embodied in $\mathcal{H}$'s exports
of the $m$-good. Taking into account the tax on imports, to supply variety $%
j $ to consumers in $\mathcal{H}$, a firm in $\mathcal{F}$ must charge%
\begin{equation*}
\frac{w^{\ast \gamma }p_{e}^{1-\gamma }(1+t_{p})^{1-\gamma }}{A^{\ast }(j)}=%
\frac{(p_{e}\left( 1+t_{p}\right) )^{1-\gamma }}{A^{\ast }(j)},
\end{equation*}%
while, taking into account the production tax, a firm in $\mathcal{H}$
charges 
\begin{equation*}
\frac{w^{\gamma }(p_{e}\left( 1+t_{p}\right) )^{1-\gamma }}{A(j)}=\frac{%
(p_{e}\left( 1+t_{p}\right) )^{1-\gamma }}{A(j)}.
\end{equation*}%
Thus, the firm in $\mathcal{H}$ supplies variety $j$ if and only if it can
produce that variety more efficiently: 
\begin{equation*}
R(j)=\frac{A(j)}{A^{\ast }(j)}\geq 1,
\end{equation*}%
i.e. if and only if:%
\begin{equation*}
j\leq \bar{j}=\frac{A^{\theta }}{A^{\theta }+A^{\ast \theta }}.
\end{equation*}%
Similarly, taking into account the tax rebate on exports, firms from $%
\mathcal{H}$ and $\mathcal{F}$ both act as if they are untaxed when
supplying consumers in $\mathcal{F}$. The same $\bar{j}$ applies, which is
also the $\bar{j}$ for the case of a consumption tax.

Now, consider the price of the $m$-good (recall that it includes taxes)
faced by consumers in each country. Taking into account how $\bar{j}$
responds to the BTA's, the price of the $m$-good in $\mathcal{F}$ is simply:%
\begin{equation*}
p_{m}=\phi p_{e}^{1-\gamma }\left( A^{\theta }+A^{\ast \theta }\right)
^{-1/\theta }
\end{equation*}%
while in $\mathcal{H}$, since every supplier must cover taxes on the energy
input, the price is:%
\begin{equation*}
p_{m}=\phi (p_{e}\left( 1+t_{p}\right) )^{1-\gamma }\left( A^{\theta
}+A^{\ast \theta }\right) ^{-1/\theta }.
\end{equation*}%
The price of the composite good in $\mathcal{H}$ is:%
\begin{equation*}
p_{c}=p_{m}^{\eta }\left( \left( 1+t_{c}\right) p_{e}\right) ^{1-\eta }=\phi
^{\eta }(p_{e}\left( 1+t_{c}\right) )^{1-\eta \gamma }\left( A^{\theta
}+A^{\ast \theta }\right) ^{-\eta /\theta },
\end{equation*}%
while in $\mathcal{F}$ it is:%
\begin{equation*}
p_{c}^{\ast }=p_{m}^{\eta }p_{e}^{1-\eta }.
\end{equation*}%
Again, the production tax with BTA's gives the same outcomes as the
consumption tax.

Let's consider the determination of the equilibrium energy price under a
production tax with BTA's. With BTA's, the global value of demand for energy
by households is:%
\begin{equation*}
p_{e}C_{e}^{W}=\left( 1-\eta \right) \left[ \frac{1}{1+t_{p}}%
D(p_{c})Y+D^{\ast }(p_{c}^{\ast })Y^{\ast }\right] ,
\end{equation*}%
while the global value of demand for energy inputs is%
\begin{equation*}
p_{e}M_{e}^{W}=\left( 1-\gamma \right) \eta \left[ \frac{1}{1+t_{p}}%
D(p_{c})Y+D^{\ast }(p_{c}^{\ast })Y^{\ast }\right] .
\end{equation*}%
Thus the total value of global demand for energy is:%
\begin{equation*}
p_{e}C_{e}^{W}+p_{e}M_{e}^{W}=\left( 1-\eta \gamma \right) \left[ \frac{1}{%
1+t_{p}}D(p_{c})Y+D^{\ast }(p_{c}^{\ast })Y^{\ast }\right] ,
\end{equation*}%
exactly the same as with a consumption tax at rate $t_{c}=t_{p}$. It follows
that the equilibrium energy price is the same.

Note that $\mathcal{H}$ gets tax revenue only on energy consumed directly by
households in $\mathcal{H}$ and energy embodied in the $m$-good consumed in $%
\mathcal{H}$, just as with a consumption tax. Thus tax revenue and hence
income will be the same under a production tax with BTA's as under a
consumption tax.

\paragraph{Welfare Results}

We now ask whether a consumption tax (which we have shown to be a production
tax with BTA's) has any welfare advantages for home over a production tax
without BTA's. Note that in making this comparison, we should not set $%
t_{c}=t_{p}$. Rather, we should hold $p_{e}$ fixed in this comparison, since
by doing so we fix global emissions of carbon. Thus, we want to ask, fixing
global emissions, is it advantageous for $\mathcal{H}$ to impose a
consumption tax at rate $t_{c}$ rather than a production tax at rate $t_{p}$%
. Equivalently, is it advantageous for $\mathcal{H}$ to add BTA's to a
production tax, changing the tax rate from $t_{p}$ to $t_{c}$ so as to
acheive the same level of global emissions?

By fixing $p_{e}$ we obtain some nice simplifications. First, income in $%
\mathcal{F}$ is fixed while in $\mathcal{H}$ it will change only to the
extent of a change in tax revenue. Second, of the three goods prices, only
the price of the $m$-good will differ across these two tax regimes, since
the price of the $e$-good and the $l$-good are fixed.

\subparagraph{Home Dominates Manufacturing}

To get sharp results, let's consider an extreme case. Suppose $A^{\ast }=0$,
so that $\mathcal{H}$ produces all of the $m$-good for the world. We want to
compare the equilibrium under a production tax with the equilibrium under a
consumption tax.

Under a production tax:%
\begin{equation*}
p_{m}=\frac{\phi }{A}\left( \left( 1+t_{p}\right) p_{e}\right) ^{1-\gamma }.
\end{equation*}%
Hence the price of the $c$-good in $\mathcal{H}$ is:%
\begin{equation*}
p_{c}=\left( \frac{\phi }{A}\right) ^{\eta }\left( \left( 1+t_{p}\right)
p_{e}\right) ^{1-\eta \gamma }.
\end{equation*}%
Since the price of the $m$-good is the same everywhere, while the $e$-good
is untaxed in $F$, consumers there pay:%
\begin{equation*}
p_{c}^{\ast }=\left( \frac{\phi }{A}\right) ^{\eta }\left( 1+t_{p}\right)
^{\eta }p_{e}^{1-\eta \gamma }.
\end{equation*}%
The total value of demand for energy intermediates is:%
\begin{equation*}
p_{e}M_{e}^{W}=\left( 1-\gamma \right) \eta \frac{1}{1+t_{p}}\left[
D(p_{c})Y+D^{\ast }(p_{c}^{\ast })Y^{\ast }\right] ,
\end{equation*}%
while the total value of demand for energy by households is:%
\begin{equation*}
p_{e}C_{e}^{W}=\left( 1-\eta \right) \left[ \frac{1}{1+t_{p}}%
D(p_{c})Y+D^{\ast }(p_{c}^{\ast })Y^{\ast }\right] .
\end{equation*}%
Thus, the total value of demand for energy is:%
\begin{equation*}
p_{e}M_{e}^{W}+p_{e}C_{e}^{W}=\left( 1-\eta \gamma \right) \left[ \frac{1}{%
1+t_{p}}D(p_{c})Y+D^{\ast }(p_{c}^{\ast })Y^{\ast }\right] -\frac{t_{p}}{%
1+t_{p}}\left( 1-\gamma \right) \eta D^{\ast }(p_{c}^{\ast })Y^{\ast }
\end{equation*}%
Finally, tax revenue in $\mathcal{H}$ is:%
\begin{equation*}
T_{p}=\frac{t_{p}}{1+t_{p}}\left( 1-\eta \gamma \right) D(p_{c})Y+\frac{t_{p}%
}{1+t_{p}}\left( 1-\gamma \right) \eta D^{\ast }(p_{c}^{\ast })Y^{\ast }
\end{equation*}

Under a consumption tax, consumers in $\mathcal{H}$ pay:%
\begin{equation*}
p_{m}=\frac{\phi }{A}\left( \left( 1+t_{c}\right) p_{e}\right) ^{1-\gamma },
\end{equation*}%
so that the price of the $c$-good in $\mathcal{H}$ is:%
\begin{equation*}
p_{c}=\left( \frac{\phi }{A}\right) ^{\eta }\left( \left( 1+t_{c}\right)
p_{e}\right) ^{1-\eta \gamma }.
\end{equation*}%
Consumers in $\mathcal{F}$ pay only:%
\begin{equation*}
p_{c}^{\ast }=\left( \frac{\phi }{A}\right) ^{\eta }p_{e}^{1-\eta \gamma }.
\end{equation*}%
The total value of demand for energy is:%
\begin{equation*}
p_{e}M_{e}^{W}+p_{e}C_{e}^{W}=\left( 1-\eta \gamma \right) \left[ \frac{1}{%
1+t_{c}}D(p_{c})Y+D^{\ast }(p_{c}^{\ast })Y^{\ast }\right] .
\end{equation*}%
Tax revenue is:%
\begin{equation*}
T_{c}=\frac{t_{c}}{1+t_{c}}\left( 1-\eta \gamma \right) D(p_{c})Y.
\end{equation*}

Let's compare the two tax regimes. If $t_{p}=t_{c}$, which tax regime would
lead to more spending on energy? If the value of spending on energy would be
higher under the consumption tax, then we can conclude that $t_{p}<t_{c}$ in
order for $p_{e}$ to be the same in both regimes. Under the hypothetical in
which $t_{p}=t_{c}$, there are three differences in the value of world
energy demand. First, unlike the consumption tax, the production tax raises
the price of manufactures everywhere. Hence spending under the production
tax is lower by the amount $\frac{t_{p}}{1+t_{p}}\left( 1-\gamma \right)
\eta D^{\ast }(p_{c}^{\ast })Y^{\ast }$. Second, for the same reason, the
price of the $c$-good is lower in $\mathcal{F}$ under the consumption tax,
which leads to more spending on energy. Third, tax revenue is higher under
the production tax by the amount $\frac{t_{p}}{1+t_{p}}\left( 1-\gamma
\right) \eta D^{\ast }(p_{c}^{\ast })Y^{\ast }$, which raises income by that
amount in $\mathcal{H}$. Note that this third effect, which would lead to
greater spending on energy under the production tax, is more than offset by
the first effect. Thus, overall, spending on energy is lower under the
production tax regime if $t_{p}=t_{c}$. To equate supply and demand in both
regimes at the same energy price $p_{e}$, it must be that $t_{c}>t_{p}$.
But, from the price equations, we see that if $t_{c}>t_{p}$ then the price
of the $c$-good is higher under the consumption tax. Since income is higher
under the production tax, and prices are equal or lower under the production
tax, welfare must be higher under the prodution tax.

What we've shown is that in an extreme case in which $\mathcal{H}$ is the
dominant manufacturing country, the home country prefers a production tax.
It would strictly prefer not to impose BTA's. The weakness of this extreme
case is that, by setting $A^{\ast }=1$, we've ruled out an argument for
BTA's, which is that they prevent a loss in $\mathcal{H}$'s market share in
manufacturing (in this extreme case $\mathcal{H}$'s market share is fixed at 
$1$). But, since in this extreme case $\mathcal{H}$ stictly prefers a
production tax, by continuity it will also likely prefer a production tax if 
$\mathcal{F}$'s market share in manufacturing is not too large.

Finally, note that $\mathcal{F}$ will prefer the consumption tax since its
income is invariant to taxes and the price of the $c$-good is lower with a
consumption tax. The consumption tax removes any tax burden to $\mathcal{F}$.

\subsection{Partial Border Tax Adjustments}

We now consider a scenario in which $\mathcal{H}$ imposes a production tax
together with some level of border tax adjustments (a tax on imports of the $%
m$-good and a rebate on exports of the $m$-good). The two policy levers are
thus the production tax rate $t_{p}\geq 0$ and the border tax adjustment
rate $t_{b}\in \left[ 0,t_{p}\right] $. The production tax raises the
effective cost of energy to consumers in $\mathcal{H}$ and of energy
intermediates to producers in $\mathcal{H}$ by the factor $1+t_{p}$. The
border tax adjustment has two parts: (i) a tariff of $t_{b}$ on the carbon
content of $\mathcal{H}$'s imports of the $m$-good, effectively raising the
cost of energy intermediates to producers in $\mathcal{F}$ selling in $%
\mathcal{H}$ by the factor $1+t_{b}$ and (ii) a rebate to producers in $%
\mathcal{H}$ on a portion of the production tax paid on the carbon content
of their exports of the $m$-good, effectively lowering their costs of energy
intermediates by a factor $1/(1+t_{b})$ from what that cost would be with
the production tax on its own.

If the border tax adjustment is set to $t_{b}=0$ then the policy is one of a
pure production tax at rate $t_{p}$. If the border tax adjustment is set to $%
t_{b}=t_{p}$ then the policy is one of a pure consumption tax at rate $t_{b}$%
. We can also consider cases between these two extremes. Setting $t_{p}>0$
in combination with $t_{b}\in (0,t_{p})$ is equivalent to a consumption tax
at rate:%
\begin{equation*}
\tilde{t}_{c}=t_{b}
\end{equation*}%
combined with a production tax at rate\footnote{%
We thank David Weisbach's colleague ??? for this interpretation of partial
BTA's.}:%
\begin{equation*}
\tilde{t}_{p}=\frac{1+t_{p}}{1+t_{b}}-1=\frac{t_{p}-t_{b}}{1+t_{b}}.
\end{equation*}%
This production tax, which is applied on top of the consumption tax, yields
an effective tax on production in $\mathcal{H}$ for domestic consumption of:%
\begin{equation*}
\left( 1+\tilde{t}_{p}\right) \left( 1+\tilde{t}_{c}\right) -1=\frac{1+t_{p}%
}{1+t_{b}}\left( 1+t_{b}\right) -1=t_{p}.
\end{equation*}

\subsubsection{Specialization and Prices}

As explained above, a production tax is a tax on using energy for either
production or consumption. Energy goods now cost $p_{e}(1+t_{p})$ in $%
\mathcal{H}$ whether used there as an intermediate by manufacturers (to
supply consumers in $\mathcal{H}$) or directly as a consumption good by
households in $\mathcal{H}$.

Consider producers in either country supplying manufactures to consumers in $%
\mathcal{H}$. For producers in $\mathcal{H}$ the production tax raises costs
by the factor $(1+t_{p})^{1-\gamma }$. For producers in $\mathcal{F}$ the
border tax raises costs by the factor $(1+t_{b})^{1-\gamma }$. Thus
producers in $\mathcal{H}$ supply all varieties $j$ for which:%
\begin{equation*}
R(j)\geq \left( \frac{1+t_{p}}{1+t_{b}}\right) ^{1-\gamma },
\end{equation*}%
i.e. all varieties in the interval $[0,\bar{j}]$, where, from (\ref{relative
productivity}):%
\begin{equation}
\bar{j}=\frac{A^{\theta }\left( \frac{1+t_{p}}{1+t_{b}}\right) ^{-\theta
\left( 1-\gamma \right) }}{A^{\theta }\left( \frac{1+t_{p}}{1+t_{b}}\right)
^{-\theta \left( 1-\gamma \right) }+A^{\ast \theta }}=\frac{A^{\theta
}\left( 1+\tilde{t}_{p}\right) ^{-\theta \left( 1-\gamma \right) }}{%
A^{\theta }\left( 1+\tilde{t}_{p}\right) ^{-\theta \left( 1-\gamma \right)
}+A^{\ast \theta }}.  \label{jbar partial bta}
\end{equation}%
These taxes alter how countries specialize across varieties of manufactures,
reducing $\mathcal{H}$'s share if $t_{p}>t_{b}$ (i.e. if $\tilde{t}_{p}>0$).
To illustrate the symmetry, we can also write:%
\begin{equation*}
\bar{j}=\frac{A^{\theta }\left( 1+t_{p}\right) ^{-\theta \left( 1-\gamma
\right) }}{A^{\theta }\left( 1+t_{p}\right) ^{-\theta \left( 1-\gamma
\right) }+A^{\ast \theta }\left( 1+t_{b}\right) ^{-\theta \left( 1-\gamma
\right) }}.
\end{equation*}%
Just as the production tax hits $\mathcal{H}$, so the border tax adjustment
hits $\mathcal{F}$.

Now consider producers in either country supplying manufactures to consumers
in $\mathcal{F}$. For producers in $\mathcal{H}$ the production tax,
together with the border rebate of taxes on exports, raises costs by the
factor:%
\begin{equation*}
\left( (1+t_{p})/(1+t_{b})\right) ^{1-\gamma }=\left( 1+\tilde{t}_{p}\right)
^{1-\gamma }.
\end{equation*}%
Producers in $\mathcal{F}$ face no taxes. Thus (\ref{jbar partial bta}) is
the cutoff determining which country supplies which varieties to consumers
anywhere.

The maximum price of a manufacturing variety supplied to consumers in $%
\mathcal{H}$ is:%
\begin{eqnarray*}
\bar{p} &=&\frac{\bar{j}^{1/\theta }p_{e}^{1-\gamma }\left( 1+t_{p}\right)
^{1-\gamma }}{A} \\
&=&p_{e}^{1-\gamma }\left( A^{\theta }\left( 1+t_{p}\right) ^{-\theta \left(
1-\gamma \right) }+A^{\ast \theta }\left( 1+t_{b}\right) ^{-\theta \left(
1-\gamma \right) }\right) ^{-1/\theta }.
\end{eqnarray*}%
We still have that the manufacturing price index is $p_{m}=\phi \bar{p}$, so
that when we substitute out $\bar{p}$: 
\begin{equation*}
p_{m}=\phi p_{e}^{1-\gamma }\left( A^{\theta }\left( 1+t_{p}\right)
^{-\theta \left( 1-\gamma \right) }+A^{\ast \theta }\left( 1+t_{b}\right)
^{-\theta \left( 1-\gamma \right) }\right) ^{-1/\theta }.
\end{equation*}%
Holding $p_{e}$ fixed, the production tax and border tax adjustment in $%
\mathcal{H}$ makes manufactures more expensive to consumers there. What
about for consumers in $\mathcal{F}$? The maximum price there is:%
\begin{equation*}
\bar{p}^{\ast }=\frac{\bar{j}^{1/\theta }p_{e}^{1-\gamma }\left( \frac{%
1+t_{p}}{1+t_{b}}\right) ^{1-\gamma }}{A},
\end{equation*}%
so that%
\begin{equation*}
p_{m}^{\ast }=\phi p_{e}^{1-\gamma }\left( A^{\theta }\left( \frac{1+t_{p}}{%
1+t_{b}}\right) ^{-\theta \left( 1-\gamma \right) }+A^{\ast \theta }\right)
^{-1/\theta }=\left( 1+t_{b}\right) ^{-\left( 1-\gamma \right) }p_{m}.
\end{equation*}

The production tax also raises the price of energy consumed directly by
households in $\mathcal{H}$. The after-tax price of the composite good in $%
\mathcal{H}$ becomes:%
\begin{eqnarray}
p_{c} &=&p_{m}^{\eta }p_{e}^{1-\eta }(1+t_{p})^{1-\eta }  \notag \\
&=&\phi ^{\eta }p_{e}^{1-\eta \gamma }(1+t_{p})^{1-\eta }\left( A^{\theta
}\left( 1+t_{p}\right) ^{-\theta \left( 1-\gamma \right) }+A^{\ast \theta
}\left( 1+t_{b}\right) ^{-\theta \left( 1-\gamma \right) }\right) ^{-\eta
/\theta }.  \label{p_c partial bta}
\end{eqnarray}%
Since there is no production tax in $\mathcal{F}$, the after-tax price of
the composite good there is:%
\begin{equation}
p_{c}^{\ast }=\phi ^{\eta }p_{e}^{1-\eta \gamma }\left( A^{\theta }\left( 
\frac{1+t_{p}}{1+t_{b}}\right) ^{-\theta \left( 1-\gamma \right) }+A^{\ast
\theta }\right) ^{-\eta /\theta }.  \label{p_c star partial bta}
\end{equation}%
Given $p_{c}$, spending on the $c$-good is $D(p_{c})Y$, with spending share $%
D(p_{c})$ given by (\ref{c spending share}). Hence consumption spending on
the $e$-good is:%
\begin{equation*}
(1+t_{p})p_{e}C_{e}=\left( 1-\eta \right) D(p_{c})Y.
\end{equation*}%
In $F$ we have:%
\begin{equation*}
p_{e}C_{e}^{\ast }=\left( 1-\eta \right) D^{\ast }(p_{c}^{\ast })Y^{\ast }.
\end{equation*}

\subsubsection{Income}

If $\mathcal{H}$ also imposes an extraction tax, world energy production is:%
\begin{equation*}
Q_{e}^{W}=\left( \frac{p_{e}}{1+t_{e}}\right) ^{\beta /(1-\beta
)}E+p_{e}^{\beta ^{\ast }/(1-\beta ^{\ast })}E^{\ast }.
\end{equation*}%
Since there are no taxes in $\mathcal{F}$, total income there is:%
\begin{equation}
Y^{\ast }=L^{\ast }+\left( 1-\beta ^{\ast }\right) p_{e}^{1/(1-\beta ^{\ast
})}E^{\ast }.  \label{Y star partial bta}
\end{equation}%
In $\mathcal{H}$, income is:%
\begin{equation*}
Y=L+\left( 1-\beta +t_{e}\right) \left( \frac{p_{e}}{1+t_{e}}\right)
^{1/(1-\beta )}E+T_{p}+T_{b},
\end{equation*}%
where $T_{p}$ is revenue from the production tax and $T_{b}$ is revenue
(positive or negative) from border tax adjustments.

To calculate tax revenue, we need expressions for spending on energy
intermediates. Spending on the energy input in $\mathcal{H}$ for sales of
manufactures in $\mathcal{H}$ is:%
\begin{equation*}
(1+t_{p})p_{e}M_{e}^{HH}=\left( 1-\gamma \right) \bar{j}\eta D(p_{c})Y,
\end{equation*}%
while for $\mathcal{H}$'s sales of manufactures in $\mathcal{F}$ the border
adjustment applies:%
\begin{equation*}
\left( 1+\tilde{t}_{p}\right) p_{e}M_{e}^{FH}=\frac{1+t_{p}}{1+t_{b}}%
p_{e}M_{e}^{FH}=\left( 1-\gamma \right) \bar{j}\eta D^{\ast }(p_{c}^{\ast
})Y^{\ast }.
\end{equation*}%
Spending on the energy input in $\mathcal{F}$ for sales of manufactures in $%
\mathcal{H}$ is:%
\begin{equation*}
\left( 1+t_{b}\right) p_{e}M_{e}^{HF}=\left( 1-\gamma \right) \left( 1-\bar{j%
}\right) \eta D(p_{c})Y,
\end{equation*}%
while for $\mathcal{F}$'s sales of manufactures in $\mathcal{F}$ no tax
applies:%
\begin{equation*}
p_{e}M_{e}^{FF}=\left( 1-\gamma \right) \left( 1-\bar{j}\right) \eta D^{\ast
}(p_{c}^{\ast })Y^{\ast }.
\end{equation*}

Production tax revenue consists of the taxes collected from firms and those
collected from households. The total is:%
\begin{eqnarray*}
T_{p} &=&t_{p}p_{e}M_{e}^{HH}+t_{p}p_{e}M_{e}^{FH}+t_{p}p_{e}C_{e} \\
&=&\frac{t_{p}}{1+t_{p}}\left( 1-\gamma \right) \bar{j}\eta D(p_{c})Y+\frac{%
t_{p}}{1+\tilde{t}_{p}}\left( 1-\gamma \right) \bar{j}\eta D^{\ast
}(p_{c}^{\ast })Y^{\ast }+\frac{t_{p}}{1+t_{p}}\left( 1-\eta \right)
D(p_{c})Y \\
&=&\frac{t_{p}}{1+t_{p}}\left[ \left( 1-\gamma \right) \bar{j}\eta
D(p_{c})Y+\left( 1-\eta \right) D(p_{c})Y+\left( 1+t_{b}\right) \left(
1-\gamma \right) \bar{j}\eta D^{\ast }(p_{c}^{\ast })Y^{\ast }\right] ,
\end{eqnarray*}%
with $\bar{j}$ given by (\ref{jbar partial bta}). The revenue (positive or
negative) from border tax adjustments is:%
\begin{eqnarray*}
T_{b} &=&t_{b}p_{e}M_{e}^{HF}-\left( t_{p}-\tilde{t}_{p}\right)
p_{e}M_{e}^{FH} \\
&=&\frac{t_{b}}{1+t_{b}}\left( 1-\gamma \right) \left( 1-\bar{j}\right) \eta
D(p_{c})Y-\frac{t_{p}-\tilde{t}_{p}}{1+\tilde{t}_{p}}\left( 1-\gamma \right) 
\bar{j}\eta D^{\ast }(p_{c}^{\ast })Y^{\ast } \\
&=&\frac{t_{b}}{1+t_{b}}\left[ \left( 1-\gamma \right) \left( 1-\bar{j}%
\right) \eta D(p_{c})Y-\left( 1+t_{b}\right) \left( 1-\gamma \right) \bar{j}%
\eta D^{\ast }(p_{c}^{\ast })Y^{\ast }\right] 
\end{eqnarray*}%
Adding in tax revenue, income in $\mathcal{H}$ is given by:%
\begin{eqnarray}
Y &=&L+\left( 1-\beta +t_{e}\right) R_{e}  \notag \\
&&+\left( \frac{t_{p}}{1+t_{p}}\left[ \left( 1-\gamma \right) \bar{j}\eta
+\left( 1-\eta \right) \right] +\frac{t_{b}}{1+t_{b}}\left( 1-\gamma \right)
\left( 1-\bar{j}\right) \eta \right) D(p_{c})Y  \notag \\
&&+\left( \frac{t_{p}-t_{b}}{1+t_{p}}\right) \left( 1-\gamma \right) \bar{j}%
\eta D^{\ast }(p_{c}^{\ast })Y^{\ast }.  \label{Y home partial bta}
\end{eqnarray}

\subsubsection{Equilibrium with Partial BTA's}

Equilibrium consists of a price $p_{e}$ that equates world supply and demand
for energy. Valuing both at world prices, we have:%
\begin{equation*}
\left( 1+t_{e}\right) ^{-\beta /(1-\beta )}p_{e}^{1/(1-\beta
)}E+p_{e}^{1/(1-\beta ^{\ast })}E^{\ast }=p_{e}C_{e}^{W}+p_{e}M_{e}^{W}.
\end{equation*}%
The value of world demand for energy by households is:%
\begin{equation}
p_{e}C_{e}^{W}=\left( 1-\eta \right) \left( \frac{1}{1+t_{p}}%
D(p_{c})Y+D^{\ast }(p_{c}^{\ast })Y^{\ast }\right) ,
\label{world consumption spending}
\end{equation}%
and the value of world demand for energy by firms is:%
\begin{eqnarray}
p_{e}M_{e}^{W} &=&p_{e}M_{e}^{HW}+p_{e}M_{e}^{FW}  \notag \\
&=&\left( 1-\gamma \right) \eta \left[ \left( \frac{\bar{j}}{1+t_{p}}+\frac{%
1-\bar{j}}{1+t_{b}}\right) D(p_{c})Y+\left( \frac{\bar{j}}{1+\tilde{t}_{p}}%
+1-\bar{j}\right) D^{\ast }(p_{c}^{\ast })Y^{\ast }\right]   \notag \\
&=&\left( 1-\gamma \right) \eta \left( \frac{1+t_{b}}{1+t_{p}}\bar{j}+1-\bar{%
j}\right) \left( \frac{1}{1+t_{b}}D(p_{c})Y+D^{\ast }(p_{c}^{\ast })Y^{\ast
}\right) .  \label{world intermediate spending}
\end{eqnarray}%
with $p_{c}$ given by (\ref{p_c partial bta}), $p_{c}^{\ast }$ given by (\ref%
{p_c star partial bta}), $\bar{j}$ given by (\ref{jbar partial bta}), $Y$
given by (\ref{Y home partial bta}), and $Y^{\ast }$ given by (\ref{Y star
partial bta}).

\subsection{Optimal Partial BTA's}

We now consider what level of border tax adjustment is optimal for $\mathcal{%
H}$ given some goal $\hat{Q}_{e}^{W}=G<1$ for reduction of global emissions.
The two policy levers are the production tax $t_{p}^{\prime }\geq 0$ and the
border tax adjustment $t_{b}^{\prime }\in \left[ 0,t_{p}^{\prime }\right] $.
For now we ignore extraction taxes, setting $t_{e}^{\prime }=0$. We will
work out the answer in the form of changes from a baseline that can be
calibrated to data. For simplicity, we assume no taxes in this baseline. We
also set $\beta =\beta ^{\ast }$.

\subsubsection{Properties of the Baseline}

In the baseline, income in $\mathcal{H}$ is:%
\begin{equation*}
Y=L+rE=L+(1-\beta )p_{e}Q_{e}
\end{equation*}%
so that, dividing both sides by $Y$:%
\begin{equation*}
1-\pi _{L}=(1-\beta )\pi _{e},
\end{equation*}%
where%
\begin{equation*}
\pi _{e}=\frac{p_{e}Q_{e}}{Y}
\end{equation*}%
is the share of the energy or extraction sector in GDP. The same holds in $%
\mathcal{F}$:%
\begin{equation*}
1-\pi _{L}^{\ast }=(1-\beta )\pi _{e}^{\ast }.
\end{equation*}

The cutoff manufacturing variety in the baseline is:%
\begin{equation*}
\bar{j}=\frac{A^{\theta }}{A^{\theta }+A^{\ast \theta }}.
\end{equation*}%
In data, the analog of $\bar{j}$ is the fraction of spending on manufactures
devoted to those produced in $\mathcal{H}$. We also need an expression for
how $\bar{j}$ evolves from its baseline when we introduce carbon taxes. From
(\ref{jbar partial bta}) we get the convenient expression:%
\begin{equation}
\bar{j}^{\prime }(t_{b}^{\prime },t_{p}^{\prime })=\frac{\bar{j}\left( \frac{%
1+t_{p}^{\prime }}{1+t_{b}^{\prime }}\right) ^{-\theta \left( 1-\gamma
\right) }}{\bar{j}\left( \frac{1+t_{p}^{\prime }}{1+t_{b}^{\prime }}\right)
^{-\theta \left( 1-\gamma \right) }+1-\bar{j}}.  \label{jbar prime}
\end{equation}

Our definition of the consumption share is now:%
\begin{equation*}
\pi _{c}=\frac{p_{c}C_{c}}{Y}=D(p_{c})
\end{equation*}%
in $\mathcal{H}$ and 
\begin{equation*}
\pi _{c}^{\ast }=D^{\ast }(p_{c}^{\ast })
\end{equation*}%
in $\mathcal{F}$, i.e. $1$ minus the share of income spent on the $l$-good
(reducing to Simple Model for the case of $\eta =0$). Applying (\ref{c
spending share}) we have:%
\begin{equation*}
\pi _{c}=\frac{\alpha p_{c}^{-\left( \sigma -1\right) }}{\alpha
p_{c}^{-(\sigma -1)}+1-\alpha }.
\end{equation*}%
When we introduced carbon taxes, this share in $H$ evolves to:%
\begin{equation}
\pi _{c}^{\prime }=\frac{\pi _{c}\hat{p}_{c}^{-\left( \sigma -1\right) }}{%
\pi _{c}\hat{p}_{c}^{-\left( \sigma -1\right) }+1-\pi _{c}}.
\label{c share prime}
\end{equation}%
In $F$ it evolves to:%
\begin{equation}
\pi _{c}^{\ast \prime }=\frac{\pi _{c}^{\ast }\left( \hat{p}_{c}^{\ast
}\right) ^{-\left( \sigma -1\right) }}{\pi _{c}^{\ast }\left( \hat{p}%
_{c}^{\ast }\right) ^{-\left( \sigma -1\right) }+1-\pi _{c}^{\ast }}.
\label{c share star prime}
\end{equation}%
Carbon taxes do not show up directly in these expressions because $\hat{p}%
_{c}$ and $\hat{p}_{c}^{\ast }$ already embody the effect of the tax on the
spending share.

\subsubsection{Energy Supply}

Recall that $G$ represents the goal for a reduction in global emissions.
Because there is no extraction tax, the supply side is simply:%
\begin{equation}
G=\omega _{e}\hat{p}_{e}^{\beta /(1-\beta )}+\left( 1-\omega _{e}\right) 
\hat{p}_{e}^{\beta /(1-\beta )},  \label{supply change partial btas}
\end{equation}%
where%
\begin{equation*}
\omega _{e}=\frac{\pi _{e}Y}{\pi _{e}Y+\pi _{e}^{\ast }Y^{\ast }}=\frac{%
p_{e}Q_{e}}{p_{e}Q_{e}^{W}}.
\end{equation*}%
The global emission goal $G$ nails down the change in the energy price:%
\begin{equation}
\hat{p}_{e}=G^{\left( 1-\beta \right) /\beta },  \label{phat energy of G}
\end{equation}%
which we can henceforth take as given.

\subsubsection{Prices}

With the change in the energy price nailed down, we can solve for the other
prices. From (\ref{p_c partial bta}) we have in $\mathcal{H}$:%
\begin{equation*}
\hat{p}_{c}=\hat{p}_{m}^{\eta }\hat{p}_{e}^{1-\eta }(1+t_{p}^{\prime
})^{1-\eta }
\end{equation*}%
with $\hat{p}_{m}$ given by%
\begin{equation*}
\hat{p}_{m}=\hat{p}_{e}^{1-\gamma }\left( \bar{j}\left( 1+t_{p}^{\prime
}\right) ^{-\theta \left( 1-\gamma \right) }+\left( 1-\bar{j}\right) \left(
1+t_{b}^{\prime }\right) ^{-\theta \left( 1-\gamma \right) }\right)
^{-1/\theta }.
\end{equation*}%
Combining these pieces, the change in the price of the $c$-good in $\mathcal{%
H}$ is:%
\begin{equation}
\hat{p}_{c}(t_{b}^{\prime },t_{p}^{\prime })=\hat{p}_{e}^{1-\gamma \eta
}(1+t_{p}^{\prime })^{1-\eta }\left( \bar{j}\left( 1+t_{p}^{\prime }\right)
^{-\theta \left( 1-\gamma \right) }+\left( 1-\bar{j}\right) \left(
1+t_{b}^{\prime }\right) ^{-\theta \left( 1-\gamma \right) }\right) ^{-\eta
/\theta }.  \label{phat c given G}
\end{equation}%
The change in the price of the $c$-good in $\mathcal{F}$ reflects the fact
that neither energy used in final consumption nor manufactures produced in $%
\mathcal{F}$ for domestic consumption are taxed:%
\begin{equation}
\hat{p}_{c}^{\ast }(t_{b}^{\prime },t_{p}^{\prime })=\hat{p}_{e}^{1-\gamma
\eta }\left( \bar{j}\left( \frac{1+t_{p}^{\prime }}{1+t_{b}^{\prime }}%
\right) ^{-\theta \left( 1-\gamma \right) }+\left( 1-\bar{j}\right) \right)
^{-\eta /\theta }.  \label{phat c star given G}
\end{equation}%
We can substitute (\ref{phat c given G}) into (\ref{c share prime}) to get:%
\begin{equation}
\pi _{c}^{\prime }(t_{b}^{\prime },t_{p}^{\prime })=\frac{\pi _{c}\hat{p}%
_{c}(t_{b}^{\prime },t_{p}^{\prime })^{-\left( \sigma -1\right) }}{\pi _{c}%
\hat{p}_{c}(t_{b}^{\prime },t_{p}^{\prime })^{-\left( \sigma -1\right)
}+1-\pi _{c}},  \label{c share prime function}
\end{equation}%
and (\ref{phat c star given G}) into (\ref{c share star prime}) to get: 
\begin{equation}
\pi _{c}^{\ast \prime }(t_{b}^{\prime },t_{p}^{\prime })=\frac{\pi
_{c}^{\ast }\hat{p}_{c}^{\ast }(t_{b}^{\prime },t_{p}^{\prime })^{-\left(
\sigma -1\right) }}{\pi _{c}^{\ast }\hat{p}_{c}^{\ast }(t_{b}^{\prime
},t_{p}^{\prime })^{-\left( \sigma -1\right) }+1-\pi _{c}^{\ast }}.
\label{c share star prime function}
\end{equation}

\subsubsection{Income}

For the change in $\mathcal{H}$'s income, we can rewrite (\ref{Y home
partial bta}) as:

\begin{eqnarray*}
\hat{Y} &=&\pi _{L}+\left( 1-\beta \right) \pi _{e}\hat{p}_{e}^{1/(1-\beta )}
\\
&&+\left( \frac{t_{p}^{\prime }}{1+t_{p}^{\prime }}\left( \left( 1-\gamma
\right) \bar{j}^{\prime }\eta +\left( 1-\eta \right) \right) +\frac{%
t_{b}^{\prime }}{1+t_{b}^{\prime }}\left( 1-\gamma \right) \left( 1-\bar{j}%
^{\prime }\right) \eta \right) \pi _{c}^{\prime }\hat{Y} \\
&&+\left( \frac{t_{p}^{\prime }-t_{b}^{\prime }}{1+t_{p}^{\prime }}\right)
\left( 1-\gamma \right) \bar{j}^{\prime }\eta \pi _{c}^{\ast \prime }\frac{%
Y^{\ast }}{Y}\hat{Y}^{\ast }.
\end{eqnarray*}%
Hence:%
\begin{equation}
\hat{Y}(t_{b}^{\prime },t_{p}^{\prime })=\frac{\pi _{L}+\left( 1-\beta
\right) \pi _{e}\hat{p}_{e}^{1/(1-\beta )}+\left( \frac{t_{p}^{\prime
}-t_{b}^{\prime }}{1+t_{p}^{\prime }}\right) \left( 1-\gamma \right) \bar{j}%
^{\prime }\eta \pi _{c}^{\ast \prime }\frac{Y^{\ast }}{Y}\hat{Y}^{\ast }}{%
1-\left( \frac{t_{p}^{\prime }}{1+t_{p}^{\prime }}\left( \left( 1-\gamma
\right) \bar{j}^{\prime }\eta +\left( 1-\eta \right) \right) +\frac{%
t_{b}^{\prime }}{1+t_{b}^{\prime }}\left( 1-\gamma \right) \left( 1-\bar{j}%
^{\prime }\right) \eta \right) \pi _{c}^{\prime }},
\label{Y hat partial btas}
\end{equation}%
where $\bar{j}^{\prime }=\bar{j}^{\prime }(t_{b}^{\prime },t_{p}^{\prime })$
given by (\ref{jbar prime}), $\pi _{c}^{\prime }=\pi _{c}^{\prime
}(t_{b}^{\prime },t_{p}^{\prime })$ by (\ref{c share prime function}), $\pi
_{c}^{\ast \prime }=$ $\pi _{c}^{\ast \prime }(t_{b}^{\prime },t_{p}^{\prime
})$ by (\ref{c share star prime function}), and the change in $\mathcal{F}$%
's income:%
\begin{equation}
\hat{Y}^{\ast }=\pi _{L}^{\ast }+\left( 1-\beta \right) \pi _{e}^{\ast }\hat{%
p}_{e}^{1/(1-\beta )}.  \label{Y star hat partial btas}
\end{equation}%
Everywhere we evaluate $\hat{p}_{e}$ using (\ref{phat c given G}).

\subsubsection{Demand}

Combining (\ref{world consumption spending}) and (\ref{world intermediate
spending}), with taxes world spending on energy becomes:%
\begin{eqnarray*}
X_{e}^{W^{\prime }} &=&p_{e}^{\prime }C_{e}^{W\prime }+p_{e}^{\prime
}M_{e}^{W\prime }=\left( 1-\eta \right) \left( \frac{1}{1+t_{p}^{\prime }}%
D(p_{c}^{\prime })Y^{\prime }+D^{\ast }(p_{c}^{\ast \prime })Y^{\ast \prime
}\right) \\
&&+\left( 1-\gamma \right) \eta \left( \frac{1+t_{b}^{\prime }}{%
1+t_{p}^{\prime }}\bar{j}^{\prime }+1-\bar{j}^{\prime }\right) \left( \frac{1%
}{1+t_{b}^{\prime }}D(p_{c}^{\prime })Y^{\prime }+D^{\ast }(p_{c}^{\ast
\prime })Y^{\ast \prime }\right) \\
&=&\left( \frac{1-\eta }{1+t_{p}^{\prime }}+\left( \frac{1-\gamma }{%
1+t_{b}^{\prime }}\right) \eta \left( \frac{1+t_{b}^{\prime }}{%
1+t_{p}^{\prime }}\bar{j}^{\prime }+1-\bar{j}^{\prime }\right) \right)
D(p_{c}^{\prime })Y^{\prime } \\
&&+\left( \left( 1-\eta \right) +\left( 1-\gamma \right) \eta \left( \frac{%
1+t_{b}^{\prime }}{1+t_{p}^{\prime }}\bar{j}^{\prime }+1-\bar{j}^{\prime
}\right) \right) D^{\ast }(p_{c}^{\ast \prime })Y^{\ast \prime }.
\end{eqnarray*}%
Thus, the change in world energy spending can be written as:%
\begin{eqnarray}
\hat{X}_{e}^{W}(t_{b}^{\prime },t_{p}^{\prime }) &=&\left( \frac{1-\eta }{%
1+t_{p}^{\prime }}+\left( \frac{1-\gamma }{1+t_{b}^{\prime }}\right) \eta
\left( \frac{1+t_{b}^{\prime }}{1+t_{p}^{\prime }}\bar{j}^{\prime }+1-\bar{j}%
^{\prime }\right) \right) \frac{\omega _{e}}{\pi _{e}}\pi _{c}^{\prime }\hat{%
Y}  \notag \\
&&+\left( \left( 1-\eta \right) +\left( 1-\gamma \right) \eta \left( \frac{%
1+t_{b}^{\prime }}{1+t_{p}^{\prime }}\bar{j}^{\prime }+1-\bar{j}^{\prime
}\right) \right) \frac{\omega _{e}^{\ast }}{\pi _{e}^{\ast }}\pi _{c}^{\ast
\prime }\hat{Y}^{\ast }.  \label{demand change partial btas}
\end{eqnarray}%
where $\bar{j}^{\prime }=\bar{j}^{\prime }(t_{b}^{\prime },t_{p}^{\prime })$
is given by (\ref{jbar prime}), $\pi _{c}^{\prime }=\pi _{c}^{\prime
}(t_{b}^{\prime },t_{p}^{\prime })$ by (\ref{c share prime function}), $\pi
_{c}^{\ast \prime }=$ $\pi _{c}^{\ast \prime }(t_{b}^{\prime },t_{p}^{\prime
})$ by (\ref{c share star prime function}), $\hat{Y}=\hat{Y}(t_{b}^{\prime
},t_{p}^{\prime })$ by (\ref{Y hat partial btas}), $\hat{Y}^{\ast }$ by (\ref%
{Y star hat partial btas}), and all evaluated at (\ref{phat c star given G}).

\subsubsection{Welfare}

Welfare in $\mathcal{H}$ is:%
\begin{equation*}
W=\frac{Y}{p},
\end{equation*}%
where the aggregate price index is from (\ref{aggregate price index II}). We
want to select the tax pair that maximizes:%
\begin{equation*}
\hat{W}=\frac{W^{\prime }}{W}=\frac{\hat{Y}}{\hat{p}}.
\end{equation*}%
We already have an expression (\ref{Y hat partial btas}) for $\hat{Y}$. From
(\ref{aggregate price index II}), the change in the aggregate price level
can be written as: 
\begin{equation}
\hat{p}(t_{b}^{\prime },t_{p}^{\prime })=\left( \pi _{c}\hat{p}%
_{c}(t_{b}^{\prime },t_{p}^{\prime })^{-\left( \sigma -1\right) }+(1-\pi
_{c})\right) ^{-1/(\sigma -1)},  \label{phat aggregate}
\end{equation}%
with $\hat{p}_{c}(t_{b}^{\prime },t_{p}^{\prime })$ given by (\ref{phat c
given G}).

As a result of these substitutions, we can express the change in welfare as
a function of just the two policy instruments:%
\begin{equation*}
\hat{W}(t_{b}^{\prime },t_{p}^{\prime })=\frac{\hat{Y}(t_{b}^{\prime
},t_{p}^{\prime })}{\hat{p}(t_{b}^{\prime },t_{p}^{\prime })}.
\end{equation*}%
The constraint is that global consumption of energy decline by the same
amount as supply:%
\begin{equation*}
g(t_{b}^{\prime },t_{p}^{\prime })=\frac{\hat{X}_{e}^{W}(t_{b}^{\prime
},t_{p}^{\prime })}{\hat{p}_{e}}=G,
\end{equation*}%
where $\hat{p}_{e}$ is given by (\ref{phat c star given G}). Our problem
reduces to maximizing the Lagrangian:%
\begin{equation*}
\mathcal{L}=\hat{W}(t_{b}^{\prime },t_{p}^{\prime })+\lambda \left[
G-g(t_{b}^{\prime },t_{p}^{\prime })\right] ,
\end{equation*}%
where $\lambda $ is the lagrange multiplier.

\subsubsection{Special Cases}

To get some insight into this problem we consider several extreme cases. We
focus on the determinants of the change in world demand for energy, as given
by (\ref{demand change partial btas}).

\paragraph{Only Direct Consumption of Energy}

The simplest case is $\eta =0$ so that manufactures are not consumed. Hence,
energy is consumed only directly by households, as in Simple Model. In this
case (\ref{demand change partial btas}) simplifies to:%
\begin{equation*}
\hat{X}_{e}^{W}=\frac{1}{1+t_{p}^{\prime }}\frac{\omega _{e}}{\pi _{e}}\pi
_{c}^{\prime }\hat{Y}+\frac{\omega _{e}^{\ast }}{\pi _{e}^{\ast }}\pi
_{c}^{\ast \prime }\hat{Y}^{\ast }.
\end{equation*}%
How is this expression altered by a shift in demand? Inspection of (\ref{c
share prime}) shows that $\pi _{c}^{\prime }$ is increasing in $\pi _{c}$
and of (\ref{c share star prime}) that $\pi _{c}^{\ast \prime }$ is
increasing in $\pi _{c}^{\ast }$. Thus, a shift in preference for energy
from $\mathcal{F}$ to $\mathcal{H}$ (an increase in $\pi _{c}$ together with
a decrease in $\pi _{c}^{\ast }$) will increase the tax base in $\mathcal{H}$
so that a \emph{lower} production tax rate in $\mathcal{H}$ will acheive the
same reduction in world energy consumption, given $\hat{Y}$.

Taking account of the change in income in $\mathcal{H}$ strengthens this
effect. Note that (\ref{Y hat partial btas}) reduces to:%
\begin{equation*}
\hat{Y}=\frac{\pi _{L}+\left( 1-\beta \right) \pi _{e}\hat{p}%
_{e}^{1/(1-\beta )}}{1-\frac{t_{p}^{\prime }}{1+t_{p}^{\prime }}\pi
_{c}^{\prime }},
\end{equation*}%
which is increasing in $\pi _{c}^{\prime }$.

Another case delivering essentially the same result is if $\eta >0$ but $%
\gamma =1$. In this case the manufacturing sector is active, but only labor
is used to produce manufactures. Then (\ref{demand change partial btas})
simplifies to:%
\begin{equation*}
\hat{X}_{e}^{W}=\left( 1-\eta \right) \left( \frac{1}{1+t_{p}^{\prime }}%
\frac{\omega _{e}}{\pi _{e}}\pi _{c}^{\prime }\hat{Y}+\frac{\omega
_{e}^{\ast }}{\pi _{e}^{\ast }}\pi _{c}^{\ast \prime }\hat{Y}^{\ast }\right)
.
\end{equation*}%
The only difference is the multiplicative term, which represents direct
spending on energy as a share of spending on the $c$-good. As above, income
in $\mathcal{H}$ rises with an increase in $\pi _{c}^{\prime }$, now
according to:%
\begin{equation*}
\hat{Y}=\frac{\pi _{L}+\left( 1-\beta \right) \pi _{e}\hat{p}%
_{e}^{1/(1-\beta )}}{1-\frac{t_{p}^{\prime }}{1+t_{p}^{\prime }}\left(
1-\eta \right) \pi _{c}^{\prime }}.
\end{equation*}

\paragraph{Only Indirect Consumption of Energy}

Now consider $\eta =1$, so that there is no direct spending on energy.
Energy is demanded only indirectly as it is embodied in manufactures. In
this case (\ref{demand change partial btas}) reduces to:%
\begin{equation*}
\hat{X}_{e}^{W}=\left( 1-\gamma \right) \left( \frac{1+t_{b}^{\prime }}{%
1+t_{p}^{\prime }}\bar{j}^{\prime }+1-\bar{j}^{\prime }\right) \left( \frac{1%
}{1+t_{b}^{\prime }}\frac{\omega _{e}}{\pi _{e}}\pi _{c}^{\prime }\hat{Y}+%
\frac{\omega _{e}^{\ast }}{\pi _{e}^{\ast }}\pi _{c}^{\ast \prime }\hat{Y}%
^{\ast }\right) .
\end{equation*}%
In this case, consider an improvement in $\mathcal{H}$'s comparative
advantage in manufactures. In particular, raising $\bar{j}$ will, via (\ref%
{jbar prime}), lead to a rise in $\bar{j}^{\prime }$. The resulting increase
in the tax base will, given $t_{b}^{\prime }$, allow for a \emph{lower}
production tax rate to acheive the same reduction in world demand for
energy. In this case we can ignore $\hat{Y}$ and $\hat{Y}^{\ast }$ since
they don't vary with $\bar{j}$.

We can now return to the analysis of a shift in demand, imposing initial
symmetry across countries to keep things simple ($\pi _{e}=\pi _{e}^{\ast }$
and $\omega _{e}=$ $\omega _{e}^{\ast }$, so that $Y=Y^{\ast }$):

\begin{equation*}
\hat{X}_{e}^{W}=\left( 1-\gamma \right) \left( \frac{1}{1+t_{p}^{\prime }}%
\bar{j}^{\prime }+1-\bar{j}^{\prime }\right) \left( \pi _{c}^{\prime }\hat{Y}%
+\pi _{c}^{\ast \prime }\hat{Y}^{\ast }\right) ,
\end{equation*}%
where we have assumed no border adjustments so that $t_{b}^{\prime }=0$. In
this case, it appears that a shift in preference for energy from $\mathcal{F}
$ to $\mathcal{H}$ will raise overall energy demand, requiring a \emph{higher%
} production tax rate to reduce energy demand by the same amount. To confirm
this intuition, consider the change in income (imposing $Y=Y^{\ast }$):%
\begin{equation*}
\hat{Y}=\frac{\pi _{L}+\left( 1-\beta \right) \pi _{e}\hat{p}%
_{e}^{1/(1-\beta )}+\left( 1-\gamma \right) \frac{t_{p}^{\prime }}{%
1+t_{p}^{\prime }}\bar{j}^{\prime }\pi _{c}^{\ast \prime }\hat{Y}^{\ast }}{%
1-\left( 1-\gamma \right) \frac{t_{p}^{\prime }}{1+t_{p}^{\prime }}\bar{j}%
^{\prime }\pi _{c}^{\prime }}.
\end{equation*}%
Consider how the income change is affected by changes in $\pi _{c}^{\prime }$
and $\pi _{c}^{\ast \prime }$:%
\begin{equation*}
\frac{\partial \hat{Y}}{\partial \pi _{c}^{\prime }}=\frac{(1-\gamma )\frac{%
t_{p}^{\prime }}{1+t_{p}^{\prime }}\bar{j}^{\prime }}{\left( 1-(1-\gamma )%
\frac{t_{p}^{\prime }}{1+t_{p}^{\prime }}\bar{j}^{\prime }\pi _{c}^{\prime
}\right) ^{2}}
\end{equation*}%
\begin{equation*}
\frac{\partial \hat{Y}}{\partial \pi _{c}^{\ast \prime }}=\frac{(1-\gamma )%
\frac{t_{p}^{\prime }}{1+t_{p}^{\prime }}\bar{j}^{\prime }\hat{Y}^{\ast }}{%
1-(1-\gamma )\frac{t_{p}^{\prime }}{1+t_{p}^{\prime }}\bar{j}^{\prime }\pi
_{c}^{\prime }}
\end{equation*}%
Noting that $1-(1-\gamma )\frac{t_{p}^{\prime }}{1+t_{p}^{\prime }}\bar{j}%
^{\prime }\pi _{c}^{\prime }<1$ and $\hat{Y}^{\ast }<1$, it follows that $%
\frac{\partial \hat{Y}}{\partial \pi _{c}^{\prime }}>\frac{\partial \hat{Y}}{%
\partial \pi _{c}^{\ast \prime }}$. Thus, a shift in energy preference from $%
\mathcal{F}$ to $\mathcal{H}$ tends to increase $\hat{Y}$. Taking into
account that $\hat{Y}^{\ast }$ is invariable to energy preference, it must
then be that this shift increases $\left( \pi _{c}^{\prime }\hat{Y}+\pi
_{c}^{\ast \prime }\hat{Y}^{\ast }\right) $. To maintain the same goal for
emissions reduction, $\left( \frac{1}{1+t_{p}^{\prime }}\bar{j}^{\prime }+1-%
\bar{j}^{\prime }\right) $ must fall to compensate. How should $%
t_{p}^{\prime }$ change to achieve this decrease? To determine this,
substitute (\ref{jbar prime}) into the expression. This yields 
\begin{eqnarray*}
\left( \frac{1}{1+t_{p}^{\prime }}\bar{j}^{\prime }+1-\bar{j}^{\prime
}\right) &=&1-\frac{t_{p}^{\prime }}{1+t_{p}^{\prime }}\left( \frac{\bar{j}%
(1+t_{p}^{\prime })^{-\theta (1-\gamma )}}{\bar{j}(1+t_{p}^{\prime
})^{-\theta (1-\gamma )}+1-\bar{j}}\right) \\
&=&1-\frac{t_{p}^{\prime }}{(1+t_{p}^{\prime })\left( 1+\bar{j}(1-\bar{j}%
)(1+t_{p}^{\prime })^{\theta (1-\gamma )}\right) }
\end{eqnarray*}%
Now take a derivative with respect to $t_{p}^{\prime }$ to get (after
rearranging):%
\begin{equation*}
\bar{j}(1-\bar{j})(1+t_{p}^{\prime })^{\theta (1-\gamma )}\left( \theta
(1-\gamma )t_{p}^{\prime }-1\right) -1
\end{equation*}%
When this derivative is positive, a \textit{decrease} in $t_{p}^{\prime }$
is needed to balance the increase in the consumption term. When it is
negative, an \textit{increase} in $t_{p}^{\prime }$ is needed. An exact
solution must be determined numerically after asserting values for
parameters, but immediately one can see that if $t_{p}^{\prime }\leq \frac{1%
}{\theta (1-\gamma )}$, which for reasonable values of $\theta $ and $\gamma 
$ is true for tax rates of up to and greater than 100\%, an \textit{increase}
in $t_{p}^{\prime }$ is unequivocally needed to maintain the same level of
emissions reduction as energy preference shifts from $\mathcal{F}$ to $%
\mathcal{H}$.

\subsection{Leakage Under Partial BTA's}

Imposition of carbon taxes typically results in production and consumption
leakage; $\mathcal{F}$ increases production and consumption of energy,
somewhat offsetting the reductions in $\mathcal{H}$. Here we derive formulas
for leakage resulting from introducing partial BTA's, continuing to assume
baseline taxes are zero.

We start by summarizing results from above that are the key ingredients for
leakage calculations. There are six ways in which energy is used throughout
the world: for production of manufactures in either country for delivery to
either country and directly for consumption in either country.

After imposing partial BTA's, the value of the energy used in $\mathcal{H}$
to supply the domestic market is:

\begin{equation*}
p_{e}^{\prime }M_{e}^{HH\prime }=\left( 1-\gamma \right) \bar{j}^{\prime
}\eta \frac{\pi _{c}^{\prime }Y^{\prime }}{1+t_{p}^{\prime }}
\end{equation*}%
while for the foreign market:%
\begin{equation*}
p_{e}^{\prime }M_{e}^{FH\prime }=\left( 1-\gamma \right) \bar{j}^{\prime
}\eta \frac{\pi _{c}^{\ast \prime }Y^{\ast \prime }}{1+\tilde{t}_{p}^{\prime
}}.
\end{equation*}%
The value of the energy used in $\mathcal{F}$ to supply manufactures to $%
\mathcal{H}$ is:%
\begin{equation*}
p_{e}^{\prime }M_{e}^{HF\prime }=\left( 1-\gamma \right) \left( 1-\bar{j}%
^{\prime }\right) \eta \frac{\pi _{c}^{\prime }Y^{\prime }}{1+t_{b}^{\prime }%
},
\end{equation*}%
while in producing for $\mathcal{F}$:%
\begin{equation*}
p_{e}^{\prime }M_{e}^{FF\prime }=\left( 1-\gamma \right) \left( 1-\bar{j}%
^{\prime }\right) \eta \pi _{c}^{\ast \prime }Y^{\ast \prime }.
\end{equation*}%
Direct demand for energy by households in $\mathcal{H}$ is:%
\begin{equation*}
p_{e}^{\prime }C_{e}^{\prime }=\left( 1-\eta \right) \frac{\pi _{c}^{\prime
}Y^{\prime }}{1+t_{p}^{\prime }},
\end{equation*}%
while in $\mathcal{F}$:%
\begin{equation*}
p_{e}^{\prime }C_{e}^{\ast \prime }=\left( 1-\eta \right) \pi _{c}^{\ast
\prime }Y^{\ast \prime }.
\end{equation*}

We can exploit the equations above, together with (\ref{jbar prime}), to
derive expressions for the changes in the quantity of energy used for each
of these six purposes:%
\begin{equation}
\hat{M}_{e}^{HH}=\left( \frac{\left( 1+\tilde{t}_{p}^{\prime }\right)
^{-\theta \left( 1-\gamma \right) }}{\bar{j}\left( 1+\tilde{t}_{p}^{\prime
}\right) ^{-\theta \left( 1-\gamma \right) }+1-\bar{j}}\right) \frac{\hat{\pi%
}_{c}\hat{Y}}{\left( 1+t_{p}^{\prime }\right) \hat{p}_{e}},  \label{MhatH}
\end{equation}

\begin{equation}
\hat{M}_{e}^{FH}=\left( \frac{\left( 1+\tilde{t}_{p}^{\prime }\right)
^{-\theta \left( 1-\gamma \right) }}{\bar{j}\left( 1+\tilde{t}_{p}^{\prime
}\right) ^{-\theta \left( 1-\gamma \right) }+1-\bar{j}}\right) \frac{\hat{\pi%
}_{c}^{\ast }\hat{Y}^{\ast }}{\left( 1+\tilde{t}_{p}^{\prime }\right) \hat{p}%
_{e}},  \label{MhatF}
\end{equation}%
\begin{equation}
\hat{C}_{e}=\frac{\hat{\pi}_{c}\hat{Y}}{\left( 1+t_{p}^{\prime }\right) \hat{%
p}_{e}},  \label{Chate}
\end{equation}%
\begin{equation}
\hat{M}_{e}^{HF}=\left( \frac{1}{\bar{j}\left( 1+\tilde{t}_{p}^{\prime
}\right) ^{-\theta \left( 1-\gamma \right) }+1-\bar{j}}\right) \frac{\hat{\pi%
}_{c}\hat{Y}}{\left( 1+t_{b}^{\prime }\right) \hat{p}_{e}},
\label{MhatHstar}
\end{equation}%
\begin{equation}
\hat{M}_{e}^{FF}=\left( \frac{1}{\bar{j}\left( 1+\tilde{t}_{p}^{\prime
}\right) ^{-\theta \left( 1-\gamma \right) }+1-\bar{j}}\right) \frac{\hat{\pi%
}_{c}^{\ast }\hat{Y}^{\ast }}{\hat{p}_{e}},  \label{MhatFstar}
\end{equation}%
and%
\begin{equation}
\hat{C}_{e}^{\ast }=\frac{\hat{\pi}_{c}^{\ast }\hat{Y}^{\ast }}{\hat{p}_{e}}.
\label{Chatestar}
\end{equation}%
We now turn to formulas for production leakage and consumption leakage, in
turn.

\subsubsection{Production Leakage}

Consider production leakage first, where production includes the energy used
in manufacturing together with direct consumption of energy (which we
interpret as household production). Production leakage represents the
increase in energy used in production in the $\mathcal{F}$ divided by the
decline in energy used in production in $\mathcal{H}$:%
\begin{eqnarray*}
l_{P} &=&\frac{\left( M_{e}^{HF\prime }+M_{e}^{FF\prime }+C_{e}^{\ast \prime
}\right) -\left( M_{e}^{HF}+M_{e}^{FF}+C_{e}^{\ast }\right) }{\left(
M_{e}^{HH}+M_{e}^{FH}+C_{e}\right) -\left( M_{e}^{HH\prime }+M_{e}^{FH\prime
}+C_{e}^{\prime }\right) } \\
&=&\frac{p_{e}M_{e}^{HF}\left( \hat{M}_{e}^{HF}-1\right)
+p_{e}M_{e}^{FF}\left( \hat{M}_{e}^{FF}-1\right) +p_{e}C_{e}^{\ast }\left( 
\hat{C}_{e}^{\ast }-1\right) }{p_{e}M_{e}^{HH}\left( 1-\hat{M}%
_{e}^{HH}\right) +p_{e}M_{e}^{FH}\left( 1-\hat{M}_{e}^{FH}\right)
+p_{e}C_{e}\left( 1-\hat{C}_{e}\right) }.
\end{eqnarray*}%
Plugging in the expressions for the six forms of spending on energy above:%
\begin{equation*}
l_{P}=\frac{\left( 1-\gamma \right) \eta \left( 1-\bar{j}\right) \left[ \pi
_{c}Y\left( \hat{M}_{e}^{HF}-1\right) +\pi _{c}^{\ast }Y^{\ast }\left( \hat{M%
}_{e}^{FF}-1\right) \right] +\left( 1-\eta \right) \pi _{c}^{\ast }Y^{\ast
}\left( \hat{C}_{e}^{\ast }-1\right) }{\left( 1-\gamma \right) \eta \bar{j}%
\left[ \pi _{c}Y\left( 1-\hat{M}_{e}^{HH}\right) +\pi _{c}^{\ast }Y^{\ast
}\left( 1-\hat{M}_{e}^{FH}\right) \right] +\left( 1-\eta \right) \pi
_{c}Y\left( 1-\hat{C}_{e}\right) }.
\end{equation*}%
Defining the $\mathcal{H}$'s share of world spending on the $c$-good by:%
\begin{equation*}
\omega _{c}=\frac{\pi _{c}Y}{\pi _{c}Y+\pi _{c}^{\ast }Y^{\ast }},
\end{equation*}%
we can write:%
\begin{equation*}
l_{P}=\frac{\left( 1-\gamma \right) \eta \left( 1-\bar{j}\right) \left[
\left( \omega _{c}\hat{M}_{e}^{HF}+\left( 1-\omega _{c}\right) \hat{M}%
_{e}^{FF}\right) -1\right] +\left( 1-\eta \right) \left( 1-\omega
_{c}\right) \pi _{c}^{\ast }Y^{\ast }\left( \hat{C}_{e}^{\ast }-1\right) }{%
\left( 1-\gamma \right) \eta \bar{j}\left[ 1-\left( \omega _{c}\hat{M}%
_{e}^{HH}+\left( 1-\omega _{c}\right) \hat{M}_{e}^{FH}\right) \right]
+\left( 1-\eta \right) \omega _{c}\left( 1-\hat{C}_{e}\right) }
\end{equation*}%
Plugging in equation (\ref{MhatH}) to (\ref{Chatestar}) yields our formula
for production leakage.

\subsubsection{Consumption Leakage}

Now consider consumption leakage, where consumption includes both the energy
consumed directly by households as well as the energy embodied in
consumption of manufactures. This form of leakage is simply a rearrangement
of the terms in the formula for production leakage:%
\begin{equation*}
l_{C}=\frac{\left( M_{e}^{FH\prime }+M_{e}^{FF\prime }+C_{e}^{\ast \prime
}\right) -\left( M_{e}^{FH}+M_{e}^{FF}+C_{e}^{\ast }\right) }{\left(
M_{e}^{HH}+M_{e}^{HF}+C_{e}\right) -\left( M_{e}^{HH\prime }+M_{e}^{HF\prime
}+C_{e}^{\prime }\right) }.
\end{equation*}%
Plugging in the expressions for the six forms of spending on energy above:%
\begin{equation*}
l_{C}=\frac{\left( 1-\gamma \right) \eta \pi _{c}^{\ast }Y^{\ast }\left[ 
\bar{j}\left( \hat{M}_{e}^{FH}-1\right) +\left( 1-\bar{j}\right) \left( \hat{%
M}_{e}^{FF}-1\right) \right] +\left( 1-\eta \right) \pi _{c}^{\ast }Y^{\ast
}\left( \hat{C}_{e}^{\ast }-1\right) }{\left( 1-\gamma \right) \eta \pi _{c}Y%
\left[ \bar{j}\left( 1-\hat{M}_{e}^{HH}\right) +\left( 1-\bar{j}\right)
\left( 1-\hat{M}_{e}^{HF}\right) \right] +\left( 1-\eta \right) \pi
_{c}Y\left( 1-\hat{C}_{e}\right) }
\end{equation*}%
or%
\begin{equation*}
l_{C}=\frac{\left( 1-\gamma \right) \eta \left( 1-\omega _{c}\right) \left[
\left( \bar{j}\hat{M}_{e}^{FH}+\left( 1-\bar{j}\right) \hat{M}%
_{e}^{FF}\right) -1\right] +\left( 1-\eta \right) \left( 1-\omega
_{c}\right) \left( \hat{C}_{e}^{\ast }-1\right) }{\left( 1-\gamma \right)
\eta \omega _{c}\left[ 1-\left( \bar{j}\hat{M}_{e}^{HH}+\left( 1-\bar{j}%
\right) \hat{M}_{e}^{HF}\right) \right] +\left( 1-\eta \right) \omega
_{c}\left( 1-\hat{C}_{e}\right) }
\end{equation*}%
Plugging in equation (\ref{MhatH}) to (\ref{Chatestar}) yields our formula
for consumption leakage.

\subsubsection{A Modified Leakage Formula}

Suppose we redefine leakage to be the increase in emissions abroad relative
to the decline in \emph{global} emissions. This definition has the
convenient property that the denominator is the same for production leakage,
consumption leakage, or even extraction leakage. Another advantage is that
the denominator is always positive given a set of taxes that reduce global
emissions.

Consider this formulation, which we denote by $\tilde{l}$, as it relates to
production leakage:%
\begin{eqnarray*}
\tilde{l}_{P} &=&\frac{p_{e}M_{e}^{HF}\left( \hat{M}_{e}^{HF}-1\right)
+p_{e}M_{e}^{FF}\left( \hat{M}_{e}^{FF}-1\right) +p_{e}C_{e}^{\ast }\left( 
\hat{C}_{e}^{\ast }-1\right) }{p_{e}Q_{e}^{W}\left( 1-G\right) } \\
&=&\frac{\left( 1-\gamma \right) \eta \left( 1-\bar{j}\right) \left[ \pi
_{c}Y\left( \hat{M}_{e}^{HF}-1\right) +\pi _{c}^{\ast }Y^{\ast }\left( \hat{M%
}_{e}^{FF}-1\right) \right] +\left( 1-\eta \right) \pi _{c}^{\ast }Y^{\ast
}\left( \hat{C}_{e}^{\ast }-1\right) }{p_{e}C_{e}^{W}\left( 1-G\right) } \\
&=&\frac{\left( 1-\gamma \right) \eta \left( 1-\bar{j}\right) \left[ \left(
\omega _{c}\hat{M}_{e}^{HF}+\left( 1-\omega _{c}\right) \hat{M}%
_{e}^{FF}\right) -1\right] +\left( 1-\eta \right) \left( 1-\omega
_{c}\right) \pi _{c}^{\ast }Y^{\ast }\left( \hat{C}_{e}^{\ast }-1\right) }{%
\left( 1-\eta \gamma \right) \left( 1-G\right) }.
\end{eqnarray*}%
For consumption leakage:%
\begin{equation*}
\tilde{l}_{C}=\frac{\left( 1-\gamma \right) \eta \left( 1-\omega _{c}\right) %
\left[ \left( \bar{j}\hat{M}_{e}^{FH}+\left( 1-\bar{j}\right) \hat{M}%
_{e}^{FF}\right) -1\right] +\left( 1-\eta \right) \left( 1-\omega
_{c}\right) \left( \hat{C}_{e}^{\ast }-1\right) }{\left( 1-\eta \gamma
\right) \left( 1-G\right) }
\end{equation*}%
Modified leakage is related to the standard leakage expression $l$ via:%
\begin{equation}
\tilde{l}=\frac{l}{1-l},  \label{ltilde vs. l}
\end{equation}%
or looked at the other way around:%
\begin{equation*}
l=\frac{\tilde{l}}{1+\tilde{l}}.
\end{equation*}%
We now turn to the value of these various leakage measures in special cases
of the model.

\subsubsection{Analysis of Special Cases}

Consider the special case of $\eta =1$ so that energy is consumed only
indirectly. Production leakage simplifies to:%
\begin{equation*}
l_{P}=\left( \frac{1-\bar{j}}{\bar{j}}\right) \frac{\left( \omega _{c}\hat{M}%
_{e}^{HF}+\left( 1-\omega _{c}\right) \hat{M}_{e}^{FF}\right) -1}{1-\left(
\omega _{c}\hat{M}_{e}^{HH}+\left( 1-\omega _{c}\right) \hat{M}%
_{e}^{FH}\right) },
\end{equation*}%
and the modified formula is:%
\begin{equation*}
\tilde{l}_{P}=\left( 1-\bar{j}\right) \frac{\left( \omega _{c}\hat{M}%
_{e}^{HF}+\left( 1-\omega _{c}\right) \hat{M}_{e}^{FF}\right) -1}{1-G}.
\end{equation*}%
Consumption leakage simplifies to:%
\begin{equation*}
l_{C}=\left( \frac{1-\omega _{c}}{\omega _{c}}\right) \frac{\left( \bar{j}%
\hat{M}_{e}^{FH}+\left( 1-\bar{j}\right) \hat{M}_{e}^{FF}\right) -1}{%
1-\left( \bar{j}\hat{M}_{e}^{HH}+\left( 1-\bar{j}\right) \hat{M}%
_{e}^{HF}\right) }
\end{equation*}%
and%
\begin{equation*}
\tilde{l}_{C}=\left( 1-\omega _{c}\right) \frac{\left( \bar{j}\hat{M}%
_{e}^{FH}+\left( 1-\bar{j}\right) \hat{M}_{e}^{FF}\right) -1}{1-G}
\end{equation*}

\paragraph{Pure Consumption Tax}

Suppose we have a pure consumption tax, $t_{b}^{\prime }=t_{p}^{\prime }=%
\tilde{t}_{c}^{\prime }$ and $\tilde{t}_{p}^{\prime }=0$. In this case (\ref%
{MhatH}) simplifies to:%
\begin{equation*}
\hat{M}_{e}^{HH}=\frac{\hat{X}_{e}^{HH}}{\left( 1+\tilde{t}_{c}^{\prime
}\right) \hat{p}_{e}}=\frac{\hat{\pi}_{c}\hat{Y}}{\left( 1+\tilde{t}%
_{c}^{\prime }\right) \hat{p}_{e}},
\end{equation*}%
(\ref{MhatF}) to:

\begin{equation*}
\hat{M}_{e}^{FH}=\frac{\hat{X}_{e}^{FH}}{\hat{p}_{e}}=\frac{\hat{\pi}%
_{c}^{\ast }\hat{Y}^{\ast }}{\hat{p}_{e}},
\end{equation*}%
(\ref{MhatHstar}) to:%
\begin{equation*}
\hat{M}_{e}^{HF}=\frac{\hat{X}_{e}^{HF}}{\left( 1+\tilde{t}_{c}^{\prime
}\right) \hat{p}_{e}}=\frac{\hat{\pi}_{c}\hat{Y}}{\left( 1+\tilde{t}%
_{c}^{\prime }\right) \hat{p}_{e}},
\end{equation*}%
and (\ref{MhatFstar}) to:%
\begin{equation*}
\hat{M}_{e}^{FF}=\frac{\hat{X}_{e}^{FF}}{\hat{p}_{e}}=\frac{\hat{\pi}%
_{c}^{\ast }\hat{Y}^{\ast }}{\hat{p}_{e}}.
\end{equation*}%
Price changes are given by:%
\begin{equation*}
\hat{p}_{c}=\hat{p}_{e}^{1-\gamma }\left( \bar{j}\left( 1+\tilde{t}%
_{c}^{\prime }\right) ^{-\theta \left( 1-\gamma \right) }+\left( 1-\bar{j}%
\right) \left( 1+\tilde{t}_{c}^{\prime }\right) ^{-\theta \left( 1-\gamma
\right) }\right) ^{-1/\theta }=\hat{p}_{e}^{1-\gamma }\left( 1+\tilde{t}%
_{c}^{\prime }\right) ^{\left( 1-\gamma \right) }
\end{equation*}%
and%
\begin{equation*}
\hat{p}_{c}^{\ast }=\hat{p}_{e}^{1-\gamma }.
\end{equation*}%
The resulting changes in consumption shares are:%
\begin{equation*}
\hat{\pi}_{c}=\frac{\hat{p}_{c}^{-\left( \sigma -1\right) }}{\pi _{c}\hat{p}%
_{c}^{-\left( \sigma -1\right) }+1-\pi _{c}}
\end{equation*}%
and%
\begin{equation*}
\hat{\pi}_{c}^{\ast }=\frac{\hat{p}_{e}^{-\left( \sigma -1\right) \left(
1-\gamma \right) }}{\pi _{c}^{\ast }\hat{p}_{e}^{-\left( \sigma -1\right)
\left( 1-\gamma \right) }+1-\pi _{c}^{\ast }}.
\end{equation*}

Production leakage reduces to:%
\begin{equation*}
l_{P}=\left( \frac{1-\bar{j}}{\bar{j}}\right) \frac{\left( \omega _{c}\frac{%
\hat{\pi}_{c}\hat{Y}}{\left( 1+\tilde{t}_{c}^{\prime }\right) \hat{p}_{e}}%
+\left( 1-\omega _{c}\right) \frac{\hat{\pi}_{c}^{\ast }\hat{Y}^{\ast }}{%
\hat{p}_{e}}\right) -1}{1-\left( \omega _{c}\frac{\hat{\pi}_{c}\hat{Y}}{%
\left( 1+\tilde{t}_{c}^{\prime }\right) \hat{p}_{e}}+\left( 1-\omega
_{c}\right) \frac{\hat{\pi}_{c}^{\ast }\hat{Y}^{\ast }}{\hat{p}_{e}}\right) }%
=-\left( \frac{1-\bar{j}}{\bar{j}}\right) .
\end{equation*}%
Noting that 
\begin{eqnarray*}
\omega _{c}\frac{\hat{\pi}_{c}\hat{Y}}{\left( 1+\tilde{t}_{c}^{\prime
}\right) \hat{p}_{e}}+\left( 1-\omega _{c}\right) \frac{\hat{\pi}_{c}^{\ast }%
\hat{Y}^{\ast }}{\hat{p}_{e}} &=&\omega _{c}\hat{M}_{e}^{HW}+\left( 1-\omega
_{c}\right) \hat{M}_{e}^{FW}= \\
&=&\frac{p_{e}M_{e}^{HW}}{p_{e}M_{e}^{W}}\hat{M}_{e}^{HW}+\frac{%
p_{e}M_{e}^{FW}}{p_{e}M_{e}^{W}}\hat{M}_{e}^{FW} \\
&=&\frac{M_{e}^{HW^{\prime }}}{M_{e}^{W}}+\frac{M_{e}^{FW^{\prime }}}{%
M_{e}^{W}} \\
&=&\frac{M_{e}^{W^{\prime }}}{M_{e}^{W}}=G
\end{eqnarray*}%
modified production leakage is:%
\begin{equation*}
\tilde{l}_{P}=-\left( 1-\bar{j}\right) .
\end{equation*}%
Of course this result also follows from (\ref{ltilde vs. l}). With a pure
consumption tax, production leakage is always negative as $\mathcal{F}$
reduces its production of emissions along with $\mathcal{H}$. The magnitude
(of the reduction in $\mathcal{F}$'s emissions) depends (negatively) on $%
\mathcal{H}$'s initial share in the global production of manufactures. We
learn nothing new from the production leakage measure in this setting.

Consumption leakage is more relevant in this setting. Evaluating consumption
leakage:%
\begin{eqnarray*}
l_{C} &=&\left( \frac{1-\omega _{c}}{\omega _{c}}\right) \frac{\left( \bar{j}%
\frac{\hat{\pi}_{c}^{\ast }\hat{Y}^{\ast }}{\hat{p}_{e}}+\left( 1-\bar{j}%
\right) \frac{\hat{\pi}_{c}^{\ast }\hat{Y}^{\ast }}{\hat{p}_{e}}\right) -1}{%
1-\left( \bar{j}\frac{\hat{\pi}_{c}\hat{Y}}{\left( 1+\tilde{t}_{c}^{\prime
}\right) \hat{p}_{e}}+\left( 1-\bar{j}\right) \frac{\hat{\pi}_{c}\hat{Y}}{%
\left( 1+\tilde{t}_{c}^{\prime }\right) \hat{p}_{e}}\right) } \\
&=&\left( \frac{1-\omega _{c}}{\omega _{c}}\right) \frac{\frac{\hat{\pi}%
_{c}^{\ast }\hat{Y}^{\ast }}{\hat{p}_{e}}-1}{1-\frac{\hat{\pi}_{c}\hat{Y}}{%
\left( 1+\tilde{t}_{c}^{\prime }\right) \hat{p}_{e}}} \\
&=&\left( \frac{1-\omega _{c}}{\omega _{c}}\right) \frac{\frac{\hat{p}%
_{e}^{-\left( \sigma -1\right) \left( 1-\gamma \right) -1}}{\pi _{c}^{\ast }%
\hat{p}_{e}^{-\left( \sigma -1\right) \left( 1-\gamma \right) }+1-\pi
_{c}^{\ast }}\hat{Y}^{\ast }-1}{1-\frac{\left( \hat{p}_{e}\left( 1+\tilde{t}%
_{c}^{\prime }\right) \right) ^{-\left( \sigma -1\right) \left( 1-\gamma
\right) -1}}{\pi _{c}\left( \hat{p}_{e}\left( 1+\tilde{t}_{c}^{\prime
}\right) \right) ^{-\left( \sigma -1\right) \left( 1-\gamma \right) }+1-\pi
_{c}}\hat{Y}},
\end{eqnarray*}%
In this setting, $\hat{p}_{e}<1$ (as dictated by the global emissions
reduction goal) and $\hat{p}_{e}\left( 1+\tilde{t}_{c}^{\prime }\right) >1$
(as dictated by $\tilde{t}_{c}^{\prime }$ being sufficient to acheive the
global emissions goal through a reduction in demand for manufactures in $%
\mathcal{H}$). Its easier to see using the modified leakage formula:%
\begin{equation*}
\tilde{l}_{C}=\left( \frac{1-\omega _{c}}{1-G}\right) \left( \frac{\hat{p}%
_{e}^{-\left( \sigma -1\right) \left( 1-\gamma \right) -1}}{\pi _{c}^{\ast }%
\hat{p}_{e}^{-\left( \sigma -1\right) \left( 1-\gamma \right) }+1-\pi
_{c}^{\ast }}\hat{Y}^{\ast }-1\right) .
\end{equation*}%
Consumption leakage is determined by the energy price decline and the extent
to which that decline leads $\mathcal{F}$ to substitute away from the $l$%
-good into manufactures. Consumption leakage is also proportional to $%
\mathcal{F}$'s initial share $1-\omega _{c}$ of global spending on
manufactures. 

Substituting in the energy price decline dictated by the global emission
goal:%
\begin{equation}
\tilde{l}_{C}=\left( \frac{1-\omega _{c}}{1-G}\right) \left( \frac{%
G^{-\left( \sigma -\sigma \gamma +\gamma \right) \left( 1-\beta \right)
/\beta }}{\pi _{c}^{\ast }G^{-\left( \sigma -1\right) \left( 1-\gamma
\right) \left( 1-\beta \right) /\beta }+1-\pi _{c}^{\ast }}\left( \pi
_{L}^{\ast }+\left( 1-\beta \right) \pi _{e}^{\ast }G^{1/\beta }\right)
-1\right) .  \label{modified consumption leakage with full BTA}
\end{equation}%
A nice feature of this leakage formula is that the tax rate itself doesn't
enter, as it is subsumed in $G$. For a given $G$, the larger is $\left(
\sigma -\sigma \gamma +\gamma \right) \left( 1-\beta \right) /\beta $, the
greater the consumption leakage resulting from a pure consumption tax.
Consumption leakage is increasing in the demand elasticity $\sigma $ and
decreasing in the extraction elasticity $\beta /\left( 1-\beta \right) $. A
smaller extraction elasticity implies a larger price decline to acheive the
global goal, and a higher demand elasticity implies a greater increase in $%
\mathcal{F}$'s consumption of embodied energy for a given price decline.

\paragraph{Pure Production Tax}

Suppose we have a pure production tax, $t_{b}^{\prime }=0$ and $\tilde{t}%
_{p}^{\prime }=t_{p}^{\prime }$. In this case (\ref{MhatH}) simplifies to:%
\begin{equation*}
\hat{M}_{e}^{HH}=\left( \frac{\left( 1+t_{p}^{\prime }\right) ^{-\theta
\left( 1-\gamma \right) -1}}{\bar{j}\left( 1+t_{p}^{\prime }\right)
^{-\theta \left( 1-\gamma \right) }+1-\bar{j}}\right) \frac{\hat{\pi}_{c}%
\hat{Y}}{\hat{p}_{e}},
\end{equation*}%
(\ref{MhatF}) to:

\begin{equation*}
\hat{M}_{e}^{FH}=\left( \frac{\left( 1+t_{p}^{\prime }\right) ^{-\theta
\left( 1-\gamma \right) -1}}{\bar{j}\left( 1+t_{p}^{\prime }\right)
^{-\theta \left( 1-\gamma \right) }+1-\bar{j}}\right) \frac{\hat{\pi}%
_{c}^{\ast }\hat{Y}^{\ast }}{\hat{p}_{e}},
\end{equation*}%
(\ref{MhatHstar}) to:%
\begin{equation*}
\hat{M}_{e}^{HF}=\left( \frac{1}{\bar{j}\left( 1+t_{p}^{\prime }\right)
^{-\theta \left( 1-\gamma \right) }+1-\bar{j}}\right) \frac{\hat{\pi}_{c}%
\hat{Y}}{\hat{p}_{e}},
\end{equation*}%
and (\ref{MhatFstar}) to:%
\begin{equation*}
\hat{M}_{e}^{FF}=\left( \frac{1}{\bar{j}\left( 1+t_{p}^{\prime }\right)
^{-\theta \left( 1-\gamma \right) }+1-\bar{j}}\right) \frac{\hat{\pi}%
_{c}^{\ast }\hat{Y}^{\ast }}{\hat{p}_{e}}.
\end{equation*}%
Price changes are given by:%
\begin{equation*}
\hat{p}_{c}=\hat{p}_{c}^{\ast }=\hat{p}_{e}^{1-\gamma }\left( \bar{j}\left(
1+t_{p}^{\prime }\right) ^{-\theta \left( 1-\gamma \right) }+\left( 1-\bar{j}%
\right) \right) ^{-1/\theta }.
\end{equation*}%
\begin{equation*}
\hat{p}_{c}^{-\theta }=\hat{p}_{e}^{-\theta \left( 1-\gamma \right) }\left( 
\bar{j}\left( 1+t_{p}^{\prime }\right) ^{-\theta \left( 1-\gamma \right)
}+\left( 1-\bar{j}\right) \right) 
\end{equation*}

In this setting (unlike the setting of the pure consumption tax) production
leakage should be a useful measure. Modified production leakage is%
\begin{eqnarray*}
\tilde{l}_{P} &=&\left( \frac{1-\bar{j}}{1-G}\right) \left( \left( \frac{1}{%
\bar{j}\left( 1+t_{p}^{\prime }\right) ^{-\theta \left( 1-\gamma \right) }+1-%
\bar{j}}\right) \left( \omega _{c}\frac{\hat{\pi}_{c}\hat{Y}}{\hat{p}_{e}}%
+\left( 1-\omega _{c}\right) \frac{\hat{\pi}_{c}^{\ast }\hat{Y}^{\ast }}{%
\hat{p}_{e}}\right) -1\right)  \\
&=&\left( \frac{1-\bar{j}}{1-G}\right) \left( \frac{G^{-\left( 1-\beta
\right) /\beta }}{\bar{j}\left( 1+t_{p}^{\prime }\right) ^{-\theta \left(
1-\gamma \right) }+1-\bar{j}}\hat{X}_{c}^{W}-1\right) .
\end{eqnarray*}

\end{document}
