%2multibyte Version: 5.50.0.2960 CodePage: 65001
%\input{tcilatex}
%\input{tcilatex}
%\input{tcilatex}


\documentclass[notitlepage,12pt]{article}
%%%%%%%%%%%%%%%%%%%%%%%%%%%%%%%%%%%%%%%%%%%%%%%%%%%%%%%%%%%%%%%%%%%%%%%%%%%%%%%%%%%%%%%%%%%%%%%%%%%%%%%%%%%%%%%%%%%%%%%%%%%%%%%%%%%%%%%%%%%%%%%%%%%%%%%%%%%%%%%%%%%%%%%%%%%%%%%%%%%%%%%%%%%%%%%%%%%%%%%%%%%%%%%%%%%%%%%%%%%%%%%%%%%%%%%%%%%%%%%%%%%%%%%%%%%%
\usepackage{amsfonts}
\usepackage{amsmath}

\setcounter{MaxMatrixCols}{10}
%TCIDATA{OutputFilter=LATEX.DLL}
%TCIDATA{Version=5.00.0.2552}
%TCIDATA{Codepage=65001}
%TCIDATA{<META NAME="SaveForMode" CONTENT="1">}
%TCIDATA{Created=Tuesday, November 13, 2012 12:43:14}
%TCIDATA{LastRevised=Thursday, July 27, 2017 14:15:13}
%TCIDATA{<META NAME="GraphicsSave" CONTENT="32">}
%TCIDATA{<META NAME="DocumentShell" CONTENT="Standard LaTeX\Standard LaTeX Article">}
%TCIDATA{Language=American English}
%TCIDATA{CSTFile=40 LaTeX article.cst}

\newtheorem{theorem}{Theorem}
\newtheorem{acknowledgement}[theorem]{Acknowledgement}
\newtheorem{algorithm}[theorem]{Algorithm}
\newtheorem{axiom}[theorem]{Axiom}
\newtheorem{case}[theorem]{Case}
\newtheorem{claim}[theorem]{Claim}
\newtheorem{conclusion}[theorem]{Conclusion}
\newtheorem{condition}[theorem]{Condition}
\newtheorem{conjecture}[theorem]{Conjecture}
\newtheorem{corollary}[theorem]{Corollary}
\newtheorem{criterion}[theorem]{Criterion}
\newtheorem{definition}[theorem]{Definition}
\newtheorem{example}[theorem]{Example}
\newtheorem{exercise}[theorem]{Exercise}
\newtheorem{lemma}[theorem]{Lemma}
\newtheorem{notation}[theorem]{Notation}
\newtheorem{problem}[theorem]{Problem}
\newtheorem{proposition}[theorem]{Proposition}
\newtheorem{remark}[theorem]{Remark}
\newtheorem{solution}[theorem]{Solution}
\newtheorem{summary}[theorem]{Summary}
\newenvironment{proof}[1][Proof]{\noindent\textbf{#1.} }{\ \rule{0.5em}{0.5em}}
\input{tcilatex}

\begin{document}

\title{Competitiveness and Carbon Taxes}
\author{Kortum and Weisbach \\
%EndAName
Yale and University of Chicago}
\maketitle

\begin{abstract}
Blah blah blah
\end{abstract}

\section{Leakage}

We focus on the case of $\eta =1$, in which energy is consumed only
indirectly through consumption of manufactures. In that case emissions show
up in four terms: $M_{e}^{HH}$, $M_{e}^{FH}$, $M_{e}^{HF}$, and $M_{e}^{FF}$
(the first subscript is the point of consumption and the second the point of
production). The total is:%
\begin{equation*}
M_{e}=M_{e}^{HH}+M_{e}^{FH}+M_{e}^{HF}+M_{e}^{FF}
\end{equation*}%
In a baseline of no carbon taxes, the fraction of emissions due to \emph{%
production} in $\mathcal{F}$ is:%
\begin{eqnarray*}
\frac{M_{e}^{HF}+M_{e}^{FF}}{M_{e}} &=&\frac{p_{e}M_{e}^{HF}+p_{e}M_{e}^{FF}%
}{p_{e}M_{e}}=\frac{\left( 1-\gamma \right) \left( 1-\bar{j}\right) \left(
\pi _{c}Y+\pi _{c}^{\ast }Y^{\ast }\right) }{\left( 1-\gamma \right) \left(
\pi _{c}Y+\pi _{c}^{\ast }Y^{\ast }\right) } \\
&=&1-\bar{j},
\end{eqnarray*}%
where $\bar{j}$ is $\mathcal{H}$'s market share in tradable manufactures.
The fraction of emissions due to \emph{consumption} in $\mathcal{F}$ is:%
\begin{equation*}
\frac{M_{e}^{FH}+M_{e}^{FF}}{M_{e}}=\frac{\left( 1-\gamma \right) \pi
_{c}^{\ast }Y^{\ast }}{\left( 1-\gamma \right) \left( \pi _{c}Y+\pi
_{c}^{\ast }Y^{\ast }\right) }=1-\omega _{c},
\end{equation*}%
where%
\begin{equation*}
\omega _{c}=\frac{\pi _{c}Y}{\pi _{c}Y+\pi _{c}^{\ast }Y^{\ast }}
\end{equation*}%
is $\mathcal{H}$'s share of world spending on the $c$-good.

We consider carbon policy in $\mathcal{H}$ consisting of a combination of a
production tax $t_{p}^{\prime }$ and a border tax adjustment $t_{b}^{\prime
}\in \lbrack 0,t_{p}^{\prime }]$. With no border adjustments ($t_{b}^{\prime
}=0$) it is a pure production tax (at rate $t_{p}^{\prime }$) while with
full border adjustments ($t_{b}^{\prime }=t_{p}^{\prime }$) it is a pure
consumption tax (at rate $t_{b}^{\prime }$). It is often helpful to
parameterize carbon taxes in terms of the border adjustment $t_{b}^{\prime }$
and the effective production tax $\tilde{t}_{p}^{\prime }$, satisfying:%
\begin{equation*}
1+\tilde{t}_{p}^{\prime }=\frac{1+t_{p}^{\prime }}{1+t_{b}^{\prime }}.
\end{equation*}%
In this space, there are no constraints on the tax rates, other than $%
t_{b}^{\prime }\geq 1$ and $\tilde{t}_{p}^{\prime }\geq 1$.

The proportional reduction in global emissions $G<1$ under such policies is
acheived through reduced demand for energy, which drives down the price of
energy faced by the extraction sector, leading that sector to supply less
energy on the world market. This reduction in the price of energy $\hat{p}%
_{e}$, known as the \emph{fuel price effect}, is connected to the reduction
in global emissions via the energy supply curve:%
\begin{equation}
\hat{p}_{e}=G^{(1-\beta )/\beta }.  \label{fuel price effect}
\end{equation}

The effect of such policies on changes in the four types of energy use are
given by:%
\begin{equation*}
\hat{M}_{e}^{HH}=\frac{\bar{j}^{\prime }}{\bar{j}}\frac{\hat{\pi}_{c}\hat{Y}%
}{\left( 1+t_{p}^{\prime }\right) \hat{p}_{e}},
\end{equation*}

\begin{equation*}
\hat{M}_{e}^{FH}=\frac{\bar{j}^{\prime }}{\bar{j}}\frac{\hat{\pi}_{c}^{\ast }%
\hat{Y}^{\ast }}{\left( 1+\tilde{t}_{p}^{\prime }\right) \hat{p}_{e}},
\end{equation*}%
\begin{equation*}
\hat{M}_{e}^{HF}=\frac{1-\bar{j}^{\prime }}{1-\bar{j}}\frac{\hat{\pi}_{c}%
\hat{Y}}{\left( 1+t_{b}^{\prime }\right) \hat{p}_{e}}
\end{equation*}%
and%
\begin{equation*}
\hat{M}_{e}^{FF}=\frac{1-\bar{j}^{\prime }}{1-\bar{j}}\frac{\hat{\pi}%
_{c}^{\ast }\hat{Y}^{\ast }}{\hat{p}_{e}}.
\end{equation*}

The goal of the carbon tax policy in $\mathcal{H}$ is to reduce global
emissions, the source of climate change. We can express $G$ in terms of the
four terms above as:%
\begin{eqnarray}
G &=&\frac{M_{e}^{HH}\hat{M}_{e}^{HH}+M_{e}^{FH}\hat{M}_{e}^{FH}+M_{e}^{HF}%
\hat{M}_{e}^{HF}+M_{e}^{FF}\hat{M}_{e}^{FF}}{%
M_{e}^{HH}+M_{e}^{FH}+M_{e}^{HF}+M_{e}^{FF}}  \notag \\
&=&\omega _{c}\bar{j}\hat{M}_{e}^{HH}+\left( 1-\omega _{c}\right) \bar{j}%
\hat{M}_{e}^{FH}+\omega _{c}\left( 1-\bar{j}\right) \hat{M}_{e}^{HF}+\left(
1-\omega _{c}\right) \left( 1-\bar{j}\right) \hat{M}_{e}^{FF}  \notag \\
&=&\bar{j}^{\prime }\left( \omega _{c}\frac{\hat{\pi}_{c}\hat{Y}}{\left(
1+t_{p}^{\prime }\right) \hat{p}_{e}}+\left( 1-\omega _{c}\right) \frac{\hat{%
\pi}_{c}^{\ast }\hat{Y}^{\ast }}{\left( 1+\tilde{t}_{p}^{\prime }\right) 
\hat{p}_{e}}\right)   \notag \\
&&+\left( 1-\bar{j}^{\prime }\right) \left( \omega _{c}\frac{\hat{\pi}_{c}%
\hat{Y}}{\left( 1+t_{b}^{\prime }\right) \hat{p}_{e}}+\left( 1-\omega
_{c}\right) \frac{\hat{\pi}_{c}^{\ast }\hat{Y}^{\ast }}{\hat{p}_{e}}\right) 
\notag \\
&=&\left( \frac{\bar{j}^{\prime }}{1+\tilde{t}_{p}^{\prime }}+1-\bar{j}%
^{\prime }\right) \left( \omega _{c}\frac{\hat{\pi}_{c}\hat{Y}}{\left(
1+t_{b}^{\prime }\right) \hat{p}_{e}}+\left( 1-\omega _{c}\right) \frac{\hat{%
\pi}_{c}^{\ast }\hat{Y}^{\ast }}{\hat{p}_{e}}\right) .  \label{G equation}
\end{eqnarray}%
We will exploit this equation below to substitute out terms involving
changes in income.

We define \emph{modified leakage} $\tilde{l}_{P}$ as the increased emissions
in $\mathcal{F}$ resulting from a unilateral carbon tax in $\mathcal{H}$
relative to the resulting decline in global emissions. A value of $\tilde{l}%
_{P}>0$ means that $\mathcal{F}$ has increased its emissions even as global
emissions have declined due to the carbon tax in $\mathcal{H}$. Our leakage
formula is thus:%
\begin{eqnarray*}
\tilde{l}_{P} &=&\frac{\left( M_{e}^{HF^{\prime }}+M_{e}^{FF^{\prime
}}\right) -\left( M_{e}^{HF}+M_{e}^{FF}\right) }{\left(
M_{e}^{HH}+M_{e}^{FH}+M_{e}^{HF}+M_{e}^{FF}\right) \left( 1-G\right) } \\
&=&\frac{1}{1-G}\frac{M_{e}^{HF}\left( \hat{M}_{e}^{HF}-1\right)
+M_{e}^{FF}\left( \hat{M}_{e}^{FF}-1\right) }{%
M_{e}^{HH}+M_{e}^{FH}+M_{e}^{HF}+M_{e}^{FF}} \\
&=&\frac{1-\bar{j}}{1-G}\left( \omega _{c}\hat{M}_{e}^{HF}+\left( 1-\omega
_{c}\right) \hat{M}_{e}^{FF}-1\right) .
\end{eqnarray*}%
Leakage is driven by the proportional increase in $\mathcal{F}$'s use of
energy in manufactures produced for its export market and in manufactures
produced for its home market. Plugging these expressions into the leakage
formula:%
\begin{equation*}
\tilde{l}_{P}=\frac{1-\bar{j}}{1-G}\left( \frac{1-\bar{j}^{\prime }}{1-\bar{j%
}}\left( \omega _{c}\frac{\hat{\pi}_{c}\hat{Y}}{\left( 1+t_{b}^{\prime
}\right) \hat{p}_{e}}+\left( 1-\omega _{c}\right) \frac{\hat{\pi}_{c}^{\ast }%
\hat{Y}^{\ast }}{\hat{p}_{e}}\right) -1\right) .
\end{equation*}%
Plugging in (\ref{G equation}), we get:%
\begin{equation*}
\tilde{l}_{P}=\frac{1-\bar{j}}{1-G}\left( \frac{1-\bar{j}^{\prime }}{1-\bar{j%
}}\frac{G}{\frac{\bar{j}^{\prime }}{1+\tilde{t}_{p}^{\prime }}+1-\bar{j}%
^{\prime }}-1\right) .
\end{equation*}%
To simplify further, we can use:%
\begin{equation*}
\frac{1-\bar{j}^{\prime }}{1-\bar{j}}=\frac{1}{\bar{j}\left( 1+\tilde{t}%
_{p}^{\prime }\right) ^{-\theta \left( 1-\gamma \right) }+1-\bar{j}}
\end{equation*}%
and%
\begin{equation*}
\bar{j}^{\prime }=\frac{\bar{j}\left( 1+\tilde{t}_{p}^{\prime }\right)
^{-\theta \left( 1-\gamma \right) }}{\bar{j}\left( 1+\tilde{t}_{p}^{\prime
}\right) ^{-\theta \left( 1-\gamma \right) }+1-\bar{j}}
\end{equation*}%
Incorporating these expressions:%
\begin{eqnarray*}
\tilde{l}_{P} &=&\frac{1-\bar{j}}{1-G}\left( \frac{1}{\bar{j}\left( 1+\tilde{%
t}_{p}^{\prime }\right) ^{-\theta \left( 1-\gamma \right) }+1-\bar{j}}\frac{G%
}{1-\frac{\tilde{t}_{p}^{\prime }}{1+\tilde{t}_{p}^{\prime }}\frac{\bar{j}%
\left( 1+\tilde{t}_{p}^{\prime }\right) ^{-\theta \left( 1-\gamma \right) }}{%
\bar{j}\left( 1+\tilde{t}_{p}^{\prime }\right) ^{-\theta \left( 1-\gamma
\right) }+1-\bar{j}}}-1\right)  \\
&=&\frac{1-\bar{j}}{1-G}\left( \frac{G}{\bar{j}\left( 1+\tilde{t}%
_{p}^{\prime }\right) ^{-\theta \left( 1-\gamma \right) }\left( 1-\frac{%
\tilde{t}_{p}^{\prime }}{1+\tilde{t}_{p}^{\prime }}\right) +1-\bar{j}}%
-1\right)  \\
&=&\frac{1-\bar{j}}{1-G}\left( \frac{G}{\bar{j}\left( 1+\tilde{t}%
_{p}^{\prime }\right) ^{-\theta \left( 1-\gamma \right) -1}+1-\bar{j}}%
-1\right) .
\end{eqnarray*}%
One implication is that, given $G$, leakage depends only on $\tilde{t}%
_{p}^{\prime }$. Furthermore, leakage is monotonically increasing in $\tilde{%
t}_{p}^{\prime }$. At the extreme of $\tilde{t}_{p}^{\prime }=0$, as under a
full border tax adjustment, the leakage formula reduces to $-\left( 1-\bar{j}%
\right) $. Leakage is negative as $\mathcal{F}$ contibutes to reductions in
global emissions in proportion to its initial share of those emissions.

\section{Welfare}

The goal of a carbon tax policy is to reduce global emissions while
minimizing any reduction in the standard of living of individuals in the
country imposing the tax. In this model, a country's welfare is its spending
power - more precisely, its income divided by a price index of energy,
manufactures, and the l-good (see :

\begin{equation*}
W = \frac{Y}{p}
\end{equation*}
After taxes, welfare can change, and we denote this as before with a hat:

\begin{equation*}
\hat{W} = \frac{W^\prime}{W}
\end{equation*}
These expressions are analogous for Foreign with stars. There is one more
relevant measure which is world welfare. Since welfare can differ between
Home and Foreign, world welfare is expressed in expectation, weighted by the
sizes of the countries' respective labor forces:

\begin{equation}
W_{world} = \omega_L*W + \omega_L^\star*W^\star
\end{equation}
where $\omega_L$ and $\omega_L^\star$ denote the world share of labor
returns in Home and Foreign, respectively (recall that wages are equalized):

\begin{align}
&\omega_L = \frac{\pi_L*Y_{rel}}{\pi_L*Y_{rel} + \pi_L^\star} \\
&\omega_L^\star = \frac{\pi_L^\star}{\pi_L*Y_{rel} + \pi_L^\star}
\end{align}

\end{document}
