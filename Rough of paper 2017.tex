%2multibyte Version: 5.50.0.2960 CodePage: 65001
%\input{tcilatex}


\documentclass[notitlepage,12pt]{article}
%%%%%%%%%%%%%%%%%%%%%%%%%%%%%%%%%%%%%%%%%%%%%%%%%%%%%%%%%%%%%%%%%%%%%%%%%%%%%%%%%%%%%%%%%%%%%%%%%%%%%%%%%%%%%%%%%%%%%%%%%%%%%%%%%%%%%%%%%%%%%%%%%%%%%%%%%%%%%%%%%%%%%%%%%%%%%%%%%%%%%%%%%%%%%%%%%%%%%%%%%%%%%%%%%%%%%%%%%%%%%%%%%%%%%%%%%%%%%%%%%%%%%%%%%%%%
\usepackage{amsfonts}
\usepackage{amsmath}

\setcounter{MaxMatrixCols}{10}
%TCIDATA{OutputFilter=LATEX.DLL}
%TCIDATA{Version=5.50.0.2960}
%TCIDATA{Codepage=65001}
%TCIDATA{<META NAME="SaveForMode" CONTENT="1">}
%TCIDATA{BibliographyScheme=Manual}
%TCIDATA{Created=Tuesday, November 13, 2012 12:43:14}
%TCIDATA{LastRevised=Friday, August 11, 2017 09:45:24}
%TCIDATA{<META NAME="GraphicsSave" CONTENT="32">}
%TCIDATA{<META NAME="DocumentShell" CONTENT="Standard LaTeX\Standard LaTeX Article">}
%TCIDATA{Language=American English}
%TCIDATA{CSTFile=40 LaTeX article.cst}

\newtheorem{theorem}{Theorem}
\newtheorem{acknowledgement}[theorem]{Acknowledgement}
\newtheorem{algorithm}[theorem]{Algorithm}
\newtheorem{axiom}[theorem]{Axiom}
\newtheorem{case}[theorem]{Case}
\newtheorem{claim}[theorem]{Claim}
\newtheorem{conclusion}[theorem]{Conclusion}
\newtheorem{condition}[theorem]{Condition}
\newtheorem{conjecture}[theorem]{Conjecture}
\newtheorem{corollary}[theorem]{Corollary}
\newtheorem{criterion}[theorem]{Criterion}
\newtheorem{definition}[theorem]{Definition}
\newtheorem{example}[theorem]{Example}
\newtheorem{exercise}[theorem]{Exercise}
\newtheorem{lemma}[theorem]{Lemma}
\newtheorem{notation}[theorem]{Notation}
\newtheorem{problem}[theorem]{Problem}
\newtheorem{proposition}[theorem]{Proposition}
\newtheorem{remark}[theorem]{Remark}
\newtheorem{solution}[theorem]{Solution}
\newtheorem{summary}[theorem]{Summary}
\newenvironment{proof}[1][Proof]{\noindent\textbf{#1.} }{\ \rule{0.5em}{0.5em}}
\input{tcilatex}
\begin{document}

\title{Trade and Carbon Taxes: An Analytic General Equliibrium Model}
\author{Sam Kortum, Michael Wang, and David A. Weisbach \\
%EndAName
Yale University and the University of Chicago}
\maketitle

\begin{abstract}
Blah blah blah
\end{abstract}

\section{Introduction}

Leakage is a central concern to design of climate change policy.

Originally, concern about developed countries adopting carbon prices while
developing do not. Even without such extreme differentiation, still concern
about differential carbon prices (whether explicit, such as through a tax,
subsidies, or cap and trade, or implicit, such as through regulations).

As a result, large number of studies. Mostly CGE. Small handful of analytic
models.

CGE have the advantage of detailed representations of the economy, often
with gret detail about the energy sector. Disadvantage is that they are
non-transparent.

Basic idea is to develop an analytic model that captures drivers of leakage.
Use DFS structure to model specialization in manufacturing across countries.
See how taxes change patterns of extraction, production, and consumption of
energy and goods created with energy.

Derive a number of analytic results. In other cases, calibrate and
numerically solve model. In these cases, look for results that are robust to
choices of parameters.

\subsection{Prior literature}

\subsubsection{CGE approaches}

Key findings:

\begin{itemize}
\item Leakage rates most often in the range of 5\% to 20\%, with some
outliers.

\item The larger the taxing coalition, the lower the leakage.

\item BA's reduce leakage substantially

\item Most important variable in determining effects of unilateral carbon
price is the energy supply elasticity.

\item Distinguish two drivers of leakage: the fuel price effect (in which
lower demand for fossil fuels in the taxing region suppresses prices,
increasing demand on the non-taxing regions) and the competitiveness effect
[different name?] (in which increased costs for industry in the taxing
region causes a shift to the non-taxing region).
\end{itemize}

\subsubsection{Analytic approaches}

Small body of work has used analytic models to understand leakage. Markusen
(1975), which focuses on environmental harms more generally, is earliest
example.

Jakob, Marschinski and Hubler (2013) use a version of Markusen (1975),
modified to allow different sectors in the economy to have different
emissions intensities. Their key finding is that the relative intensities of
the exporting and non-exporting sectors in the non-taxing region affect
whether BA's reduce leakage.

Bohringer Lange and Rutherford (2014) use an analytic model to consider
differential carbon price, such as lower taxes on trade-exposed sectors.
They decompose the effects of differential taxes into leakage effects and
terms of trade effects. They incorporate this decomposition into a CGE model
to produce guidelines for carbon pricing.

Fischer and Fox (2012).

Fullerton, Karney and Baylis (2014) use an analytic general equilibrium
model to analyze how capital used in abatement of emissions effects leakage.
They find that if abatement is resource intensive (such as requiring
substantial capital), a domestic carbon tax can produce negative leakage
because it increases global resource costs and hence investment in
non-taxing regions.

\section{Model structure}

There are two countries or regions, Home ($\mathcal{H}$) and Foreign ($%
\mathcal{F}$). Each is endowed with energy deposits ($E$, and $E^{\ast }$,
in $H$ and $F$ respectively) and labor ($L$ and $L^{\ast }$, measured in
efficiency units). (Throughout, variables related to $\mathcal{F}$ are
denoted with a $^{\ast }$.) Representative individuals or firms in each
country produce goods, receive income from the use of their endowments in
production, and use the income to purchase and consume the goods. The
countries differ only in their endowments of deposits and labor, in the
parameters of the utility functions of individuals in that country, and in
their labor share of energy extraction.

[expand the above paragraph to give better description]

\subsection{Production}

Each country produces extracts deposits using labor to produce usable
energy. Each country also produces two types of consumption goods: a good
produced only with labor, the $l$-good or services, and manufactured goods,
the $m$-goods, which are produced using energy and labor. All goods are
costlessly traded (but labor and deposits are immobile). We parameterize
production as follows.

\subsubsection{Energy extraction}

Firms in $\mathcal{H}$ extract energy using labor $L_{e}$ and desposits $E$
in a Cobb-Douglas production function with labor share $\beta $:

\begin{equation*}
Q_{e}=\left( \frac{L_{e}}{\beta }\right) ^{\beta }E^{1-\beta }.
\end{equation*}%
(We derive this production function from more primitive assumptions about
extraction in the Appendix.). As a result, firms hire labor:%
\begin{equation*}
L_{e}=\beta \left( \frac{p_{e}}{w}\right) ^{1/(1-\beta )}E
\end{equation*}%
where $w$ is the wage rate in $H$, and $p_{e}$ is the global price of
energy. Firms extract a quantity:%
\begin{equation*}
Q_{e}=\left( \frac{p_{e}}{w}\right) ^{\beta /(1-\beta )}E.
\end{equation*}

The problem is the same in $\mathcal{F}$, replacing $\beta $ with $\beta
^{\ast }$, $w$ with $w^{\ast }$, $E$ with $E^{\ast }$. The solution in $%
\mathcal{F}$ is $L_{e}^{\ast }$ and $Q_{e}^{\ast }$. Differences in $\beta $
and $\beta ^{\ast }$ capture differences in the labor share of energy
extraction, such as between a country like Saudi Arabia where energy is
cheap to extract (low $\beta $) and Canada, where energy is costly to
extract (high $\beta $).

World supply of energy is the energy produced in $H$ and $F$:

\begin{equation*}
Q_{e}^{W}=Q_{e}+Q_{e}^{\ast }=\left( \frac{p_{e}}{w}\right) ^{\beta
/(1-\beta )}E+\left( \frac{p_{e}}{w^{\ast }}\right) ^{\beta ^{\ast
}/(1-\beta ^{\ast })}E^{\ast }.
\end{equation*}

\subsubsection{Manufacturing}

Firms in each country use energy and labor to produce manufactured goods or $%
m$-goods. To model specialization in manufacturing and trade, we use a
version of the Ricardian model introduced by Dornbusch, Fisher, and
Samuelson (1977). In particular, let there be a continuous variety of $m$%
-goods indexed by $j\in \lbrack 0,1]$. The production function for variety $%
j $ in $\mathcal{H}$ is:%
\begin{equation*}
Q_{m}(j)=A(j)\left( \frac{L_{m}(j)}{\gamma }\right) ^{\gamma }\left( \frac{%
M(j)}{1-\gamma }\right) ^{1-\gamma },
\end{equation*}%
where $A(j)$ is $\mathcal{H}$'s productivity in variety $j$, $M\left(
j\right) $ is energy, and $\gamma $ is labor's share. There are many
price-taking producers have access to the technology to produce each variety 
$j$. In $\mathcal{F}$ the production function has the same form, with
productivity $A^{\ast }(j)$.

Productivity for each variety is%
\begin{equation*}
A(j)=\frac{A}{j^{1/\theta }}
\end{equation*}%
in $\mathcal{H}$ and%
\begin{equation*}
A^{\ast }(j)=\frac{A^{\ast }}{\left( 1-j\right) ^{1/\theta }}
\end{equation*}%
in $\mathcal{F}$. The parameters $A$ and $A^{\ast }$ capture absolute
advantage in $\mathcal{H}$ and $\mathcal{F}$. The relative productivity of
the two countries in producing variety $j$ is:%
\begin{equation}
R(j)=\frac{A(j)}{A^{\ast }(j)}=\frac{A}{A^{\ast }}\left( \frac{j}{1-j}%
\right) ^{-1/\theta }.  \label{relative productivity}
\end{equation}%
For $j<j^{\prime }$, $\mathcal{H}$ has a comparative advantage in variety $j$
and $\mathcal{F}$ in $j^{\prime }$. The parameter $\theta $ captures
(inversely) the strength of comparative advantage. As $\theta \rightarrow
\infty $ relative productivity does not vary across varieties.

\paragraph{Specialization in $m$-goods}

Individual varieties of the $m$-good are costlessly traded, which means that
specialization in production of manufactured goods is detached from
country-level demand. Due to trade in energy, prices $p_{e}$ are the same in
each country. A bundle of inputs costs $w^{\gamma }p_{e}^{1-\gamma }$ in $%
\mathcal{H}$ and $w^{\ast \gamma }p_{e}^{1-\gamma }$ in $\mathcal{F}$. Hence 
$\mathcal{H}$ will produce varieties $j$ for which $R(j)\geq \left(
w/w^{\ast }\right) ^{\gamma }$ and $\mathcal{F}$ will produce the rest. In
other words, $\mathcal{H}$ produces all varieties in the interval $[0,\bar{j}%
]$ and $\mathcal{F}$ in the interval $(\bar{j},1]$ where:%
\begin{equation}
\bar{j}=\frac{A^{\theta }w^{-\gamma \theta }}{A^{\theta }w^{-\gamma \theta
}+A^{\ast \theta }w^{\ast -\gamma \theta }}.  \label{jbar}
\end{equation}

Given specialization, we can solve for the competitive price of each
manufactured variety. For $j\leq \bar{j}$, the variety is produced in $%
\mathcal{H}$, so that:%
\begin{equation}
p_{m}(j)=\frac{w^{\gamma }p_{e}^{1-\gamma }}{A(j)},  \label{pm(j)}
\end{equation}%
while for $j>\bar{j}$, the variety is produced in $\mathcal{F}$, with:%
\begin{equation*}
p_{m}(j)=\frac{w^{\ast \gamma }p_{e}^{1-\gamma }}{A^{\ast }(j)}.
\end{equation*}%
With no trade cost we have the law of one price; these variety-level prices
apply for consumers in either country.

\paragraph{Price index for manufactured goods}

The price index for the $m$-good is the average price of varieties of the $m$%
-good. We show in the online appendix that we can express this as:

\begin{equation*}
p_{m}=\phi p_{e}^{1-\gamma }\left( A^{\theta }w^{-\gamma \theta }+A^{\ast
\theta }w^{\ast -\gamma \theta }\right) ^{1/\theta },
\end{equation*}%
where

\begin{equation*}
\phi =\left( \frac{\theta }{\theta -\left( \rho -1\right) }\right)
^{-1/\left( \rho -1\right) }.
\end{equation*}%
Because the relevant average price is independent of where a variety is
produced, $\bar{j}$ given in (\ref{jbar}) is also the share of world
spending on the $m$-good devoted to producers in $\mathcal{H}$ and $1-\bar{j}
$ the share devoted to producers in $\mathcal{F}$. Note that this share does
not depend on on $\rho $ . The so-called trade elasticity, giving the
response of trade shares to factor costs, is $\theta $.

\paragraph{Energy use in manufacturing}

Firms in $H$ manufacture varieties $\left[ 0,\bar{j}\right] $ and for each
variety, use $M\left( j\right) $ of energy. Hence, total energy used by
firms in $H$ is:

\begin{equation*}
M_{W}=\int_{0}^{\bar{j}^{\prime }}M\left( j\right) dj,
\end{equation*}%
Similarly, firms in $F$ use $M_{W}^{\ast }$, integrating from $\left( 1-\bar{%
j}\right) $ to $1$. Total energy used in the manufacturing sector is: $%
M_{W}+M_{W}^{\ast }=M.$

\subsubsection{Services (the $l$-good)}

Production of the $l$-good in a given country is given by

\begin{equation*}
Q_{l}=L_{l}.
\end{equation*}%
Labor is measured efficiency units, so this formulation can capture
differences in $l$-sector productivity across countries.

\subsection{Consumption}

A representative individual in each country receives rental income from the
ownership of the energy deposits and receives wages $w$ and $w^{\ast }$ in
exchange for services. The individuals use this income to purchase
manufactured goods and services, maximizing a constant-elasticity of
substitution utility function.

\subsubsection{Income}

Income comes from labor and rents on energy deposits:

\begin{equation*}
Y=wL+rE=wL+\left( 1-\beta \right) p_{e}Q_{e}=wL+\left( 1-\beta \right)
p_{e}^{1/\left( 1-\beta \right) }E.
\end{equation*}%
If $H$ imposes taxes, income also includes net tax revenue, which is rebated
lump sum. A similar equation holds for $\mathcal{F}$ and world income is the
sum of the two: $Y^{W}=Y+Y^{\ast }$.

\subsubsection{Utility}

Preferences are represented by a CES utility function:

\begin{equation}
U\left( C_{c},C_{l}\right) =\left( \alpha ^{1/\sigma }C_{c}^{\left( \sigma
-1\right) /\sigma }+\left( 1-\alpha \right) ^{1/\sigma }C_{l}^{\left( \sigma
-1\right) /\sigma }\right) ^{\sigma /\left( \sigma -1\right) }.
\label{utility}
\end{equation}%
where the $c$-good is a composite of the direct consumption of energy $C_{e%
\text{ }}$and of the manufactured good $C_{m}$:

\begin{equation*}
C_{c}=\left( \frac{C_{m}}{\eta }\right) ^{\eta }\left( \frac{C_{e}}{1-\eta }%
\right) ^{1-\eta },
\end{equation*}%
and consumption nof the $m$-good is an aggregate of the individual varieties:%
\begin{equation*}
C_{m}=\left( \int_{0}^{1}C_{m}(j)^{\left( \rho -1\right) /\rho }dj\right)
^{\rho /\left( \rho -1\right) }.
\end{equation*}%
To simplify the presentation, in what follows, we set $\eta =0$, (in which
case, $C_{c}=C_{m}$, \ and we use $C_{m}$ as our convention). This
assumption means that that individuals do not directly consume energy. The
qualitative results do not change with this simplification.

Define the demand function for consumers in $H$ for the $m$-good:

\begin{equation*}
D\left( p_{m}\right) =\frac{\alpha p_{m}^{-\left( \sigma -1\right) }}{%
ap_{m}^{-\left( \sigma -1\right) }+\left( 1-\alpha \right) w^{-\left( \sigma
-1\right) }}
\end{equation*}%
With this utility function (and no taxes), spending in $H$ on the $m$-goods
is:

\begin{equation*}
p_{m}C_{m}=D\left( p_{m}\right) Y.
\end{equation*}%
A parallel result holds in $F$.

\subsection{Equilibrium without taxes}

An equilibrium is an allocation of labor in $H$ across the $l$-good, energy
extraction,and the composite $m$-good:

\begin{equation*}
L=L_{l}+L_{e}+\int_{0}^{1}L_{m}\left( j\right) dj,
\end{equation*}%
a parallel condition in $F$, and a price of energy that clears the market.
We set labor in $F$ as the numeraire and only consider equilibria where both
countries produce the $l$-good, so we have $w^{\ast }=w=1$.

Demand for energy is $\left( 1-\gamma \right) $ of the demand for $m$-goods, 
$\left( 1-\gamma \right) \left( D\left( p_{m}\right) Y+D^{\ast }\left(
p_{m}\right) Y^{\ast }\right) .$

With no taxes, the equilibrium is straighforward. Setting the supply of
energy of equal to that demand, we get:

\begin{equation}
p_{e}^{1/\left( 1-\beta \right) }E+p_{e}^{1/\left( 1-\beta ^{\ast }\right)
}E^{\ast }=\left( 1-\gamma \right) \left( D\left( p_{m}\right) Y+D^{\ast
}\left( p_{m}\right) Y^{\ast }\right) .  \label{EQ condition}
\end{equation}%
Solving for the price of energy determines the labor used in extraction. In $%
H$, we get:

\begin{equation*}
L_{e}=\beta p_{e}^{1/\left( 1-\beta \right) }E,
\end{equation*}%
and likewise for $F$.\ The price of energy also determines the demand for $m$%
-goods and, because of Cobb-Douglas production, we know the labor share. A
fraction $\bar{j}$ is produced in $H$, giving us:

\begin{equation*}
L_{m}=\ \bar{j}\gamma \left( D(p_{c})Y+D^{\ast }(p_{c})Y^{\ast }\right)
\end{equation*}%
A similar expression holds for $F$, except using $\left( 1-\bar{j}\right) $.
Labor used to produce the $l$-good is the residual.

\section{Taxes}

Our goal is to consider how this equilibrium changes with taxes. We allow
taxes to be imposed at each level of production. In particular, we consider
taxes (1) on the extraction of energy, (2) on the use of energy in
production, and (3) on the energy content of consumption. We call these
three taxes an extraction tax, $t_{e}$, a production tax, $t_{p}$, and a
consumption tax $t_{c}$. Taxes can be imposed in both $H$ and $F$ or just in 
$H$. All taxes are ad valorem. We also allow $H$ to impose border tax
adjustments (defined below).

Without loss of generality, we assume that emissions are one-for-one with
energy use, which means that taxes on emissions are simply taxes on energy.
We do not have any non-polluting sources of energy in the model, which means
we cut off that source of substitution in response to taxation. Taxes on
energy will, however, cause poducers to substitute toward non-polluting
labor, so a similar effect is seen in the model if we think of labor as the
set of non-polluting inputs.

We first define each type of tax and border tax adjustments. Then we
consider the effects of each tax on the location of activities, on leakage
rates, and on welfare. In the next section, we provide simulations that
allow us to compute the optimal set of taxes and border adjustments to
achieve a given emisisons goal.

[Notation: we need to sort this out. Should we use a prime for pretty much
everything below to indicate that prices and quantities are tax-adjusted?
Without some difference in notation between no-tax and tax, I get confused.
But carrying primes around everywhere might not be ideal.]

\subsection{Extraction tax}

An extraction tax is an ad valorem tax on the extraction of energy. If the
market price of energy is $p_{e}$, energy extraction firms receive only $%
p_{e}/\left( 1+t_{e}\right) $. Firms in $H$ extract:

\begin{equation*}
Q_{e}=\left( \frac{p_{e}}{1+t_{e}}\right) ^{\beta /\left( 1-\beta \right) }E.
\end{equation*}%
Similar expressions hold for $F$ if $F$ imposes an extraction tax. After-tax
revenue in the extraction sector is

\begin{equation*}
R_{e}=\frac{p_{e}}{1+t_{e}}Q_{e}=\left( \frac{p_{e}}{1+t_{e}}\right)
^{1/\left( 1-\beta \right) }E,
\end{equation*}%
and tax revenue is $T_{e}=t_{e}R_{e}.$

\subsection{Production tax}

A production tax is a tax on energy used in manufacturing. Energy used in
manufacturing now costs $p_{e}\left( 1+t_{p}\right) $. As a result,
producers shift away from energy and toward labor. The cost of manufactured
goods rises by the factor $\left( 1+t_{p}\right) ^{1-\gamma }$ .

\subsection{Consumption tax}

A consumption tax is a tax imposed on the direct consumption of energy and
on the consumption of a good produced using energy based on the energy (or
emissions) from the production of the good. For example, a consumption tax
on an automobile is a tax on the energy used to produce the steel in the
automobile as well as on any fuel used to drive the automobile. It is a tax
on the total energy used to obtain transportation services.

A consumption tax will affect how goods are manufactured because it raises
the relative price of energy inputs. Therefore, to derive an expression for
a consumption tax, we must determine how manufacturers will respond to the
tax in equilibrium.

Consider a firm's sales to customers in $H$. The firm's problem is to choose
inputs of labor $l$ and energy $e$ to maximize revenue per unit, taking as
given the competitive price fo the good $p\left( j\right) $ in $H$ and $%
p^{\ast }\left( j\right) $ in $F$, inclusive of ad valorem taxes $t_{c}$ and 
$t_{c^{\ast }}$ on consumption: 
\begin{equation*}
\max_{l,e}\left\{ \left( p_{m}(j)-t_{c}p_{e}e\right) -wl-p_{e}e\right\}
\end{equation*}%
subject to:%
\begin{equation*}
A(j)\left( \frac{l}{\gamma }\right) ^{\gamma }\left( \frac{e}{1-\gamma }%
\right) ^{1-\gamma }=1.
\end{equation*}%
Solving this problem gives a price of:%
\begin{equation*}
p_{m}(j)=\left( 1+t_{c}\right) ^{1-\gamma }\frac{p_{e}^{1-\gamma }}{A(j)}.
\end{equation*}%
(setting $w=1$). This looks like a simple ad valorem tax because the
after-tax price of the $m$-good is the pre-tax price multipled by $\left(
1+t_{c}\right) ^{1-\gamma }$. Note, however, that this feature is a result
of our assumption of Cobb-Douglas production, and is not general.

[I find this notation confusing because there is nothing to tell the reader
that this is an after-tax price. Can we say something like $p_{m}^{\prime
}\left( j\right) =\left( 1+t_{c}\right) p_{m}\left( j\right) ,$ where the
prime indicates an after-tax price?]

For the firm's sales to consumers in $\mathcal{F}$ the same derivation
applies [where $t_{c}^{\ast }$ is the consumption tax, if any in $F$], so
that:%
\begin{equation*}
p_{m}^{\ast }(j)=\left( 1+t_{c}^{\ast }\right) ^{1-\gamma }\frac{w^{\gamma
}p_{e}^{1-\gamma }}{A(j)}.
\end{equation*}%
Note that $p_{m}\left( j\right) $ may not equal $p_{m}^{\ast }\left(
j\right) $ because the consumption tax rates in $H$ and $F$ may not be the
same (and with a unilateral consumption tax in $H$, $t_{c}^{\ast }$ would be 
$0$). A similar derivation gives the consumption tax for producers in $F$
for sales to consumers in $H$ and $F$.

\subsubsection{Border adjustments}

Border adjustments are (1) a tax (at rate $t_{b}$) on the value of energy
embodied in $H^{\prime }$s imports of the $m$-good and (2) a tax rebate
(also at rate $t_{b}$) on the value of energy embodied in $H$'s exports of
the $m$-good. The border tax adjustment tax rate need not be the same as the
tax rate on production, Instead, we allow $t_{b}^{\prime }\in \lbrack
0,t_{p}^{\prime }]$. [We do, however, restrict the tax rate on imports to be
the same as the rate of rebate on exports.]

With no border adjustments ($t_{b}^{\prime }=0$), the tax is a pure
production tax (at rate $t_{p}^{\prime }$). We refer to the case where $%
t_{b}^{\prime }=t_{p}^{\prime }$ as full border taxes and $t_{b}^{\prime
}\in \left( 0,t_{p}^{\prime }\right) $ as partial border taxes. Below, we
show that a production tax with a full border tax is the same as a pure
consumption tax (at rate $t_{b}^{\prime }$).

It is often helpful to parameterize taxes in terms of the border adjustment $%
t_{b}^{\prime }$ and the effective production tax $\tilde{t}_{p}^{\prime }$,
satisfying:%
\begin{equation*}
1+\tilde{t}_{p}=\frac{1+t_{p}}{1+t_{b}}.
\end{equation*}%
In this space, there are no constraints on the tax rates, other than $%
t_{b}\geq 0$ and $\tilde{t}_{p}\geq 0.$

Something about BAs under an extraction tax? Can tax extraction and border
adjust for imports and exports of energy to produce tax on energy used in $H$%
, so a producdtion tax. Or could border adjust for energy content of the $m$%
-goods, to produce a consumption tax. 

\section{Effects of global taxes}

Our goal is to understand the effects of taxes in $H$. Before turning to
that analysis, we first consider global taxes. Deriviations of these
propositions are in the Appendix.

Suppose that considering as a base case, a globallly harmonized carbon tax,
which means that both countries impose a tax at the same level of production
and at the same rate.

\begin{proposition}
If the parameters of the utility function, $\alpha $ and $\sigma $ are the
same in $H$ and $F$, a global production tax at rate $t_{p}$ has the same
effect on the price of energy (and, therefore, emissions) as a global
consumption tax and as a global extraction tax at the same rates $\left(
t_{c}=t_{p}=t_{e}\right) .$ These taxes differ in the location of tax
revenue, with the tax revenue tracking the location of the taxed activity.
With a global tax, each country prefers the tax to be imposed where it has
relatively more of the activity.
\end{proposition}

This result arises because all energy that is extracted is used in
production (in manufacturing in the basic model or directly consumed, and
therefore, in home production in the more general model) and all production
is consumed. The difference in the three taxes when imposed globally is
where in the chain of prodution they are imposed.

Identical demand parameters are needed because the location of the tax
revenue creates an income effect: without identical demand parameters, the
income effect would alter consumption choices.Note that this means that with
global taxes, the place that taxes are imposed, can, in theory, be chosen
based entirely on adminsitrative considerations, with offsetting revenue
transfers made between countries determining the distributive effects.

\begin{proposition}
The equilibrium with global taxes is the the same as specified by (\ref{EQ
condition}) substituting $p_{e}/\left( 1+t_{e}\right) $ for $p_{e}$.In
particular, with a global tax, $\bar{j}^{\prime }=\bar{j}$. This follows
immediately from the equivalence of the three taxes, the expression for
energy supply with an extracton tax, and the equilibrium conditions.
\end{proposition}

Define a global emissions reduction goal $G=\frac{Q_{e}^{w\prime }}{Q_{e}^{w}%
}<1$, [where $Q_{e}^{w}$ $=Q_{e}+Q_{e}^{\ast }$ is global energy production
under a baseline policy and $Q_{e}^{w\prime }$ is global energy production
under a proposed tax; also on notation, are we going to use $G$ or $\hat{Q}%
_{e}$? We use hats for all other changes].

\begin{proposition}
\lbrack need to assume common $\beta $ across countries for this, right?]
Energy use, and, therefore, emissions reductions, $G$, depends only on the
price of energy, $p_{e}$. With a production or consumption tax, the change
in energy price $\hat{p}_{e}=\frac{p_{e}^{\prime }}{p_{e}}$ to meet an
global emissions goal must satisfy:
\end{proposition}

\begin{equation*}
\hat{p}_{e}=G^{\left( 1-\beta \right) /\beta }
\end{equation*}

[With an extraction tax, we need $\frac{\hat{p}_{e}}{1+t_{e}}=G^{\left(
1-\beta \right) /\beta }$, right?] [Better to express this allowing the $%
\beta $'s to be different?]

This result arises because, in the model, energy is costlessly traded, so
there is a single global price of energy regardless of whether there are
global taxes or only taxes in $H$. In addition, given the production
function for energy, extraction depends only on $p_{e}$ and $w$. Because we
are assuming that $w=w^{\ast }=1$, the price of energy determines
extraction. The nominal price of energy, $p_{e}$, differs with an extraction
tax than with a production or consumption tax because with an extraction
tax, the price includes taxes paid by extractors while with production or
consumption taxes it does not.

This result (which also holds for taxes imposed only by $H$) means that for
any tax, we can determine its effects on emissions by determining the
equilibrium price of energy under the tax. Moreover, all taxes operate to
reduce emissions by lowering the global price of energy. With production and
consumption taxes, the nominal price goes down. With an extraction tax, the
after-tax price received by extractors goes down in exactly the same amount.

[Probably not worth mentioning, but global taxes have to be coordinated on
where in the chain of production they are imposed as well as the tax rate.
if $H$ imposes an extraction tax and $F$ imposes a production tax, we don't
get the same answer as when both impose one or the other.]

\section{Effects of taxes only in $H$}

\subsection{Preliminaries}

Although we now examine taxes unilaterally imposed by $H$, we compare taxes
that have the same effect on global emissions. Reason is that purpose of the
tax is to reduce emisisons and the resulting harms. Because temperature
changes are the same regardless of which country pollutes, we assume that $H$
cares about global emissions.

This assumption may only partially capture $H$'s incentives. While $H$ will
care about global emissions because the harms to $H$ from climate change are
the same regardless of where the emissions come from, climate treaties often
focus on emissions from each country. If $H$ is imposing the tax primarily
to comply with a treaty, $H$ will care about domestic emissions.

With unilateral tax, extraction still depends on $p_{e}$, so the level of
emissions still depends only on $p_{e}$ (taking into account that with
extraction taxes, extractors only keep an after-tax amount). Therefore,
Proposition x still holds.

\subsection{Location of activities}

Central concern with unilateral carbon prices is shift in where activities
occur. Note that this is not the same as leakage which is normally defined
based on changes in emissions. For example, consumption tax (we show) has
the potential to reduce consumption in $H$ which is partially offset by an
increase in consumption in $F$. Does not mean any shift in where production,
and, therefore, emissions occur. Even if no leakage, might be a concern.
Same for location of extraction.

[Is there a general statement we can make about location: each of the taxes
only directly affects the location of the taxed activity (extraction,
production or consumption). All other shifts occur either because of the
change in $p_{e}$ or an income effect due to tax revenue?]

\subsubsection{Extraction tax in $H$}

With extraction tax in $H$, we get

\begin{equation*}
Q_{e}=\left( \frac{p_{e}}{1+t_{e}}\right) ^{\beta /\left( 1-\beta \right) }E
\end{equation*}%
while $Q_{e}^{\ast }$ is unchanged. For a given price of energy, we see
relatively lower extraction in $H$ than previously. Should push up the price
of energy (less supply), which means that in equilibrium, we see higher
extraction in $F$, partially offsetting reduction in $H$.

[Worth specifying this - compare $\hat{Q}_{e}$ to $\hat{Q}_{e}^{\ast }$?]

Given an equilibrium price of energy, an extraction tax in $H$ does not
affect the location of production ($\bar{j}$ does not change) because energy
is costlessly traded so the location of the extraction of energy has no
effect on its use in manufacturing. The tax affects consumption only through
income effects.

The value of world supply of energy with an extraction tax in $H$ is:

\begin{equation}
p_{e}Q_{e}^{W}=\left( 1+t_{e}\right) ^{-\beta /\left( 1-\beta \right)
}p_{e}^{1/\left( 1-\beta \right) }E+p_{e}^{1/\left( 1-\beta \right) }E^{\ast
}  \label{Supply of energy with extraction tax in H}
\end{equation}

\subsubsection{Production tax in $H$}

No change in location of extraction (other than because $p_{e}$ changes).

Shift in j-bar:

\begin{equation*}
\bar{j}^{\prime }=\frac{A^{\theta }\left( 1+t_{p}\right) ^{-\theta \left(
1-\gamma \right) }}{A^{\theta }\left( 1+t_{p}\right) ^{-\theta \left(
1-\gamma \right) }+A^{\ast \theta }}.
\end{equation*}

[Notation - it would be helpful to specify when $\bar{j}$ is with or without
tax. I added the "prime" to do this.]

Changes $p_{m}$.

\begin{equation*}
p_{m}=\phi p_{e}^{1-\gamma }\left( A^{\theta }\left( 1+t_{p}\right)
^{-\theta \left( 1-\gamma \right) }+A^{\ast \theta }\right) ^{-1/\theta }.
\end{equation*}

Changes $Y$. Given this, consumption is as above.

Could write demand equation here and note that it equals supply in
equilibrium.

\subsubsection{Consumption tax in $H$}

Price of the $m$-goods is now different in $H$ and $F$. In $H$, we get

\begin{equation*}
p_{m}=\phi \left( \left( 1+t_{c}\right) p_{e}\right) ^{1-\gamma }\left(
A^{\theta }+A^{\ast \theta }\right) ^{-1/\theta }.
\end{equation*}%
In $F$, it is unchanged from above. We get corresponding changes in demand
for the $m$-good (adjusting the income in $H$ for taxes.) But for any level
of emissions reduction and hence $p_{e}$, the location of extraction and of
manufacturing ( $\bar{j}$) are unchanged.

\subsection{Relationship between taxes}

[Define two taxes as equal if they (1) produce the same equilibrium price of
energy and allocation of labor in each country and (2) the same utility for
all individuals in each country.]

\begin{proposition}
A consumption tax in $H$ at rate $t_{c}$ is equal to a production tax in $H$
at rate $t_{p}=t_{c}$ plus full border adjustments.
\end{proposition}

To see why this result arises, consider a consumption tax imposed only in $H$%
. Under this tax, consumers in $H$ pay a tax on all manufactured goods that
they purchase regardless of where they were produced. Compare that to a
production tax in $H$ with border adjustments. Under a production tax,
producers in $H$ must pay a tax on their production. If they sell the good
to consumers in $H$, there is no border adjustment, and the tax remains. If
they sell the good to consumers in $F$, the tax is removed, so of the goods
produced in $H$, only goods consumed in $H$ continue to bear a tax.
Similarly, producers in $F$ do not initially pay a tax when they produce a
good, but if, and only if, they sell it to consumers in $H$, border
adjustments impose a tax. Therefore, under a production tax with border
adjustments, all goods consumed in $H$ and only those goods bear a tax.

[I thought we had a similar result for extraction taxes but I\ don't see it
in the current notes.]

\subsection{Leakage}

Conventional definition of leakage is $-\Delta F/\Delta H$. This can be hard
to interpret because if we hold global emissions fixed, the both the
numerator and the denominator change when we change the type of tax.If $%
\Delta F$ changes, then $\Delta H$ must change too if emissions are held
fixed. Often, leakage measures are compared holding the nominal tax rate
fixed, which makes them very hard to compare.

Propose an alternative measure of leakage, which we call modified leakage: $%
-\Delta F/\left( \Delta H+\Delta F\right) $. With this definition, the
denominator is the decline in global emissions, so if we compare two taxes
set to be equally effective, we know that the denominator is the same. Can
easily be converted into the conventional measure if desired.

Also leads to simplification because the denominator is determined by $p_{e}$%
: all leakage measures will assume the same assumptions about price of
energy.

Leakage often said to include two effects: a fuel price effect and a
location effect. Fuel price effect arises because carbon tax in $H$ lowers
the global price of energy, leading to increased consumption in $F$. With
modified leakage, the change in fuel price is held constant across all
scenarios. Depending on tax, consumers in $F$ see its effect differently, so
behavior of those consumers is not held fixed. \ E.g., with production tax,
consumers in $F$ see a tax on the goods that they purchase from $H$ while
with a consumption tax they do not. But because price of energy is held
constant in all scenarios, we do not try to isolate a fuel price effect.

Emissions in our model come only from manufacturing. Convenient to have
terms for emissions from manufacturing. Define the spending in $H$ on energy
used during production in $H$, and similary for production in $F$ and
spending in $F$

\begin{eqnarray}
p_{e}M_{H} &=&\left( 1-\gamma \right) \bar{j}D\left( p_{m}\right) Y  \notag
\\
p_{e}M_{F} &=&\left( 1-\gamma \right) \left( 1-\bar{j}\right) D^{\ast
}\left( p_{m}^{\ast }\right) Y^{\ast }  \notag \\
p_{e}M_{H}^{\ast } &=&\left( 1-\gamma \right) \bar{j}D\left( p_{m}\right) Y 
\notag \\
p_{e}M_{F}^{\ast } &=&\left( 1-\gamma \right) \left( 1-\bar{j}\right)
D^{\ast }\left( p_{m}^{\ast }\right) Y^{\ast }
\end{eqnarray}

We then have $M_{H}+M_{F}=M_{W}$, and $M_{H}^{\ast }+M_{F}^{\ast }=M_{W}$,
and $M_{W}+M_{F}^{\ast }=M_{W}=Q_{e}^{W}$.

Modified leakage for a given global reduction in emissions $G$ is:\ 

\begin{equation*}
\tilde{l}_{P}=\frac{\left( M_{e}^{HF^{\prime }}+M_{e}^{FF^{\prime }}\right)
-\left( M_{e}^{HF}+M_{e}^{FF}\right) }{\left(
M_{e}^{HH}+M_{e}^{FH}+M_{e}^{HF}+M_{e}^{FF}\right) \left( 1-G\right) }
\end{equation*}

\begin{proposition}
Leakage with a production or consumption tax is:
\end{proposition}

\begin{equation*}
\tilde{l}_{P}=\frac{1-\bar{j}}{1-G}\left( \frac{G}{\bar{j}\left( 1+\tilde{t}%
_{p}^{\prime }\right) ^{-\theta \left( 1-\gamma \right) -1}+1-\bar{j}}%
-1\right)
\end{equation*}

Recalling that $\tilde{t}_{p}^{\prime }$ is the residual production tax if
there are partial border tax adjustments, we had previously that if $\tilde{t%
}_{p}^{\prime }=0$, the tax is a pure consumption tax. We therefore get:

\begin{corollary}
With a pure consumption tax, leakage is equal to $-\left( 1-\bar{j}\right)
<0 $.
\end{corollary}

Need to explain why we get negative leakage when nobody else does (other
than DF who gets it for entirely different reasons). Basic idea here: with a
consumption tax, producers in both $H$ and $F$ can sell fewer $m$-goods to
consumers in $H$. Energy prices go down, which means producers in both
countries sell more to consumers in $F$ but to generate a net reduction,
producers have to produce overall less. Producers in $F$ will be affected
the same way producers in $H$ will. Home bias could offset this somewhat and
most other models have home bias built in, although it is not clear that
this is the full explanation for the difference.

\subsection{Welfare\newline
}

Design of a carbon tax, including choice to impose border adjustments must
be based on welfare, not on a measure of leakage or of the location of
activities.

Define welfare as effective consumption: $\frac{Y}{p}$, where $p$ is the
price index for the $m$-good and the $l$-good: $p=\left( \alpha
p_{m}^{-\left( \sigma -1\right) }+\left( 1-\alpha \right) \right)
^{-1/\left( 1-\sigma \right) }$.

\subsubsection{$F$'s welfare}

\begin{proposition}
$F$ is better off with a consumption tax in $H$. That is, if $H$ is going to
impose a unialteral production tax that acheives a desired reduction in
global emissions, $F$ is bette off if $H$ imposes border adjustments as well.
\end{proposition}

The reasoning is straightforward, Holding emissions constant means holding $%
p_{e}$ fixed and by construction, the price of the $l$-good is fixed at 1.
Because income in $F$ is from wages and rents from energy extraction, income
in $F$ is fixed. This leaves only the price of the $m$-good to vary. With a
production tax, any varieties of the $m$-good produced in $H$ that consumers
in $F$ purchase will bear the tax while with a border adjustment, the tax is
removed. Therefore, $F$ is better off with border adjustments.

\subsubsection{$H$'s welfare}

\begin{proposition}
If $H$ sufficiently dominates manufacturing, $H$ is better off without
border adjustments.
\end{proposition}

Reasoning:

\section{Simulations}

Possible simulations:

\begin{itemize}
\item Beta v. leakage under a production tax (and possibly, under a
consumption tax).

\item leakage under a production tax with variable border adjustments

\item $H$'s welfare under production tax plus variable border adjustments
(i.e., BA's on x axis, welfare on y axis).

\item leakage v. relative size of $H$

\item Welfare gains v. relative size of $H$ (under production tax?): tells
us the benefits of increasing the size of the taxing coalition.
\end{itemize}

\section{Comparisons to CGE results}

Want to compare our results to the summary of the CGE results given above;

\begin{itemize}
\item Leakage rates most often in the range of 5\% to 20\%, with some
outliers.

\item The larger the taxing coalition, the lower the leakage.

\item BA's reduce leakage substantially

\item Most important variable in determining effects of unilateral carbon
price is the energy supply elasticity.

\item Distinguish two drivers of leakage: the fuel price effect (in which
lower demand for fossil fuels in the taxing region suppresses prices,
increasing demand on the non-taxing regions) and the competitiveness effect
[different name?] (in which increased costs for industry in the taxing
region causes a shift to the non-taxing region).
\end{itemize}

\end{document}
