%2multibyte Version: 5.50.0.2960 CodePage: 65001
%\input{tcilatex}
%\input{tcilatex}


\documentclass[notitlepage,12pt]{article}
%%%%%%%%%%%%%%%%%%%%%%%%%%%%%%%%%%%%%%%%%%%%%%%%%%%%%%%%%%%%%%%%%%%%%%%%%%%%%%%%%%%%%%%%%%%%%%%%%%%%%%%%%%%%%%%%%%%%%%%%%%%%%%%%%%%%%%%%%%%%%%%%%%%%%%%%%%%%%%%%%%%%%%%%%%%%%%%%%%%%%%%%%%%%%%%%%%%%%%%%%%%%%%%%%%%%%%%%%%%%%%%%%%%%%%%%%%%%%%%%%%%%%%%%%%%%
\usepackage{amsfonts}
\usepackage{amsmath}

\setcounter{MaxMatrixCols}{10}
%TCIDATA{OutputFilter=LATEX.DLL}
%TCIDATA{Version=5.00.0.2552}
%TCIDATA{Codepage=65001}
%TCIDATA{<META NAME="SaveForMode" CONTENT="1">}
%TCIDATA{Created=Tuesday, November 13, 2012 12:43:14}
%TCIDATA{LastRevised=Wednesday, July 26, 2017 14:52:51}
%TCIDATA{<META NAME="GraphicsSave" CONTENT="32">}
%TCIDATA{<META NAME="DocumentShell" CONTENT="Standard LaTeX\Standard LaTeX Article">}
%TCIDATA{Language=American English}
%TCIDATA{CSTFile=40 LaTeX article.cst}

\newtheorem{theorem}{Theorem}
\newtheorem{acknowledgement}[theorem]{Acknowledgement}
\newtheorem{algorithm}[theorem]{Algorithm}
\newtheorem{axiom}[theorem]{Axiom}
\newtheorem{case}[theorem]{Case}
\newtheorem{claim}[theorem]{Claim}
\newtheorem{conclusion}[theorem]{Conclusion}
\newtheorem{condition}[theorem]{Condition}
\newtheorem{conjecture}[theorem]{Conjecture}
\newtheorem{corollary}[theorem]{Corollary}
\newtheorem{criterion}[theorem]{Criterion}
\newtheorem{definition}[theorem]{Definition}
\newtheorem{example}[theorem]{Example}
\newtheorem{exercise}[theorem]{Exercise}
\newtheorem{lemma}[theorem]{Lemma}
\newtheorem{notation}[theorem]{Notation}
\newtheorem{problem}[theorem]{Problem}
\newtheorem{proposition}[theorem]{Proposition}
\newtheorem{remark}[theorem]{Remark}
\newtheorem{solution}[theorem]{Solution}
\newtheorem{summary}[theorem]{Summary}
\newenvironment{proof}[1][Proof]{\noindent\textbf{#1.} }{\ \rule{0.5em}{0.5em}}
\input{tcilatex}

\begin{document}

\title{Competitiveness and Carbon Taxes}
\author{Kortum and Weisbach \\
%EndAName
Yale and University of Chicago}
\maketitle

\begin{abstract}
Blah blah blah
\end{abstract}

\section{Leakage}

We focus on the case of $\eta =1$, in which energy is consumed only
indirectly through consumption of manufactures. In that case emissions show
up in four terms: $M_{e}^{HH}$, $M_{e}^{FH}$, $M_{e}^{HF}$, and $M_{e}^{FF}$
(the first subscript is the point of consumption and the second the point of
production). In a baseline of no carbon taxes, the fraction of emissions due
to \emph{production} in $\mathcal{F}$ is:%
\begin{eqnarray*}
\frac{M_{e}^{HF}+M_{e}^{FF}}{M_{e}^{HH}+M_{e}^{FH}+M_{e}^{HF}+M_{e}^{FF}} &=&%
\frac{p_{e}M_{e}^{HF}+p_{e}M_{e}^{FF}}{%
p_{e}M_{e}^{HH}+p_{e}M_{e}^{FH}+p_{e}M_{e}^{HF}+p_{e}M_{e}^{FF}} \\
&=&\frac{\left( 1-\gamma \right) \left( 1-\bar{j}\right) \left( \pi
_{c}Y+\pi _{c}^{\ast }Y^{\ast }\right) }{\left( 1-\gamma \right) \left( \pi
_{c}Y+\pi _{c}^{\ast }Y^{\ast }\right) } \\
&=&1-\bar{j},
\end{eqnarray*}%
where $\bar{j}$ is $\mathcal{H}$'s market share in tradable manufactures.
The fraction of emissions due to \emph{consumption} in $\mathcal{F}$ is:%
\begin{equation*}
\frac{M_{e}^{FH}+M_{e}^{FF}}{M_{e}^{HH}+M_{e}^{FH}+M_{e}^{HF}+M_{e}^{FF}}=%
\frac{\left( 1-\gamma \right) \pi _{c}^{\ast }Y^{\ast }}{\left( 1-\gamma
\right) \left( \pi _{c}Y+\pi _{c}^{\ast }Y^{\ast }\right) }=1-\omega _{c},
\end{equation*}%
where%
\begin{equation*}
\omega _{c}=\frac{\pi _{c}Y}{\pi _{c}Y+\pi _{c}^{\ast }Y^{\ast }}
\end{equation*}%
is $\mathcal{H}$'s share of world spending on the $c$-good.

We define \emph{modified leakage} $\tilde{l}_{P}$ as the increased emissions
in $\mathcal{F}$ resulting from a unilateral carbon tax in $\mathcal{H}$
relative to the resulting decline in global emissions. A value of $\tilde{l}%
_{P}>0$ means that $\mathcal{F}$ has increased its emissions even as global
emissions have declined due to the carbon tax in $\mathcal{H}$. Recall that
the proportional change in global emissions is denoted by $G$ (so that for
any carbon tax worth considering, we can take $G<1$). Our leakage formula is
thus:%
\begin{eqnarray*}
\tilde{l}_{P} &=&\frac{\left( M_{e}^{HF^{\prime }}+M_{e}^{FF^{\prime
}}\right) -\left( M_{e}^{HF}+M_{e}^{FF}\right) }{\left(
M_{e}^{HH}+M_{e}^{FH}+M_{e}^{HF}+M_{e}^{FF}\right) \left( 1-G\right) } \\
&=&\frac{1}{1-G}\frac{M_{e}^{HF}\left( \hat{M}_{e}^{HF}-1\right)
+M_{e}^{FF}\left( \hat{M}_{e}^{FF}-1\right) }{%
M_{e}^{HH}+M_{e}^{FH}+M_{e}^{HF}+M_{e}^{FF}} \\
&=&\frac{1-\bar{j}}{1-G}\left( \omega _{c}\hat{M}_{e}^{HF}+\left( 1-\omega
_{c}\right) \hat{M}_{e}^{FF}-1\right) .
\end{eqnarray*}%
Leakage is driven by the proportional increase in $\mathcal{F}$'s use of
energy in manufactures produced for its export market and in manufactures
produced for its home market. To derive expressions for these changes, we
need to take a stand on the specific carbon taxes being considered.

We treat the carbon tax in $\mathcal{H}$ as a combination of a production
tax $t_{p}^{\prime }$ and a border tax adjustment $t_{b}^{\prime }\in
\lbrack 0,t_{p}^{\prime }]$. With no border adjustments ($t_{b}^{\prime }=0$%
) it is a pure production tax (at rate $t_{p}^{\prime }$) while with full
border adjustments ($t_{b}^{\prime }=t_{p}^{\prime }$) it is a pure
consumption tax (at rate $t_{b}^{\prime }$). The reduction in global
emissions $G$ under such policies is acheived through reduced demand for
energy, which drives down the price of energy faced by the extraction
sector, leading that sector to supply less energy on the world market. This
reduction in the price of energy $\hat{p}_{e}$, known as the \emph{fuel
price effect}, is connected to the reduction in global emissions via the
energy supply curve:%
\begin{equation}
\hat{p}_{e}=G^{(1-\beta )/\beta }.  \label{fuel price effect}
\end{equation}

The effect of such policies on changes in $\mathcal{F}$'s energy use are
given by:%
\begin{equation*}
\hat{M}_{e}^{HF}=\frac{1-\bar{j}^{\prime }}{1-\bar{j}}\frac{\hat{\pi}_{c}%
\hat{Y}}{\left( 1+t_{b}^{\prime }\right) \hat{p}_{e}}
\end{equation*}%
and%
\begin{equation*}
\hat{M}_{e}^{FF}=\frac{1-\bar{j}^{\prime }}{1-\bar{j}}\frac{\hat{\pi}%
_{c}^{\ast }\hat{Y}^{\ast }}{\hat{p}_{e}}.
\end{equation*}%
Plugging these expressions into the leakage formula:%
\begin{equation*}
\tilde{l}_{P}=\frac{1-\bar{j}}{1-G}\left( \frac{1-\bar{j}^{\prime }}{1-\bar{j%
}}\left[ \omega _{c}\hat{\pi}_{c}\hat{Y}/\left( 1+t_{b}^{\prime }\right)
+\left( 1-\omega _{c}\right) \hat{\pi}_{c}^{\ast }\hat{Y}^{\ast }\right] 
\frac{1}{\hat{p}_{e}}-1\right) 
\end{equation*}%
Leakage depends on three basic factors: (i) the trade share effect is the
change in $\mathcal{F}$'s market share in manufacturing $\left( 1-\bar{j}%
^{\prime }\right) /\left( 1-\bar{j}\right) $, (ii) the fuel price effect is
the inverse of the proportional decline in the global energy price $1/\hat{p}%
_{e}$, and (iii) the spending effect is the change in world spending on
manufactured goods (taking account of any border tax adjustment) $\omega _{c}%
\hat{\pi}_{c}\hat{Y}/\left( 1+t_{b}^{\prime }\right) +\left( 1-\omega
_{c}\right) \hat{\pi}_{c}^{\ast }\hat{Y}^{\ast }$. We can consider these
factor separately.

The trade share effect is given by:%
\begin{equation*}
\frac{1-\bar{j}^{\prime }}{1-\bar{j}}=\frac{1}{\bar{j}\left( \frac{%
1+t_{p}^{\prime }}{1+t_{b}^{\prime }}\right) ^{-\theta \left( 1-\gamma
\right) }+1-\bar{j}}.
\end{equation*}%
This factor is increasing in the effective production tax rate $\tilde{t}%
_{p}^{\prime }$, defined by $1+\tilde{t}_{p}^{\prime }=(1+t_{p}^{\prime
})/(1+t_{b}^{\prime })$, which shifts production from $\mathcal{H}$ to $%
\mathcal{F}$. With full border tax adjustments this factor reduces to $1$
and thus no longer contributes to leakage.

The second factor is the fuel price effect. If a carbon tax leads to a
reduction in global emissions, equation (\ref{fuel price effect}) tells us
there will be a reduction in the energy price. Since this change is in the
denominator, the fuel price effect always contributes positively to leakage.
For a given change in spending on manufactures (and hence on energy), it
implies a greater increase in energy use. But, unlike the trade share
effect, the fuel price effect leads to greater energy use in both countries.
In fact, there would be a fuel price effect even with no trade in
manufactures.

The third factor is the most nuanced. Income falls in both countries due to
reduced rents from energy deposits. But, income may rise in $\mathcal{H}$
due to new tax revenue. There are also changes in the share of income spent
on manufactures, due to price changes from both taxes and the fuel price
effect. The change in the share of income spent on the manufactured good in $%
\mathcal{H}$ is:%
\begin{equation*}
\hat{\pi}_{c}=\frac{\hat{p}_{c}^{-\left( \sigma -1\right) }}{\pi _{c}\hat{p}%
_{c}^{-\left( \sigma -1\right) }+1-\pi _{c}}
\end{equation*}%
and likewise for $\mathcal{F}$. The change in the price of the manufactured
good in $\mathcal{H}$ is given by:%
\begin{eqnarray*}
\hat{p}_{c} &=&\hat{p}_{m}=\hat{p}_{e}^{1-\gamma }\left( \bar{j}\left(
1+t_{p}^{\prime }\right) ^{-\theta \left( 1-\gamma \right) }+\left( 1-\bar{j}%
\right) \left( 1+t_{b}^{\prime }\right) ^{-\theta \left( 1-\gamma \right)
}\right) ^{-1/\theta } \\
&=&\left( 1+t_{b}^{\prime }\right) ^{\left( 1-\gamma \right) }\hat{p}%
_{e}^{1-\gamma }\left( \bar{j}\left( \frac{1+t_{p}^{\prime }}{%
1+t_{b}^{\prime }}\right) ^{-\theta \left( 1-\gamma \right) }+\left( 1-\bar{j%
}\right) \right) ^{-1/\theta },
\end{eqnarray*}%
while in $H$:%
\begin{equation*}
\hat{p}_{c}^{\ast }=\hat{p}_{e}^{1-\gamma }\left( \bar{j}\left( \frac{%
1+t_{p}^{\prime }}{1+t_{b}^{\prime }}\right) ^{-\theta \left( 1-\gamma
\right) }+\left( 1-\bar{j}\right) \right) ^{-1/\theta }.
\end{equation*}%
Prices fall due to the fuel price effect, but taxes that shift trade shares
push prices up. In general, these shifts work in parallel in either country.
The border tax adjustment, however causes prices to rise more in $\mathcal{H}
$ than in $\mathcal{F}$:%
\begin{equation*}
\frac{\hat{p}_{c}}{\hat{p}_{c}^{\ast }}=\left( 1+t_{b}^{\prime }\right)
^{\left( 1-\gamma \right) }.
\end{equation*}

\end{document}
