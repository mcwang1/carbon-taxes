%2multibyte Version: 5.50.0.2960 CodePage: 65001
%\input{tcilatex}
%\input{tcilatex}


\documentclass[notitlepage,12pt]{article}
%%%%%%%%%%%%%%%%%%%%%%%%%%%%%%%%%%%%%%%%%%%%%%%%%%%%%%%%%%%%%%%%%%%%%%%%%%%%%%%%%%%%%%%%%%%%%%%%%%%%%%%%%%%%%%%%%%%%%%%%%%%%%%%%%%%%%%%%%%%%%%%%%%%%%%%%%%%%%%%%%%%%%%%%%%%%%%%%%%%%%%%%%%%%%%%%%%%%%%%%%%%%%%%%%%%%%%%%%%%%%%%%%%%%%%%%%%%%%%%%%%%%%%%%%%%%
\usepackage{amsfonts}
\usepackage{amsmath}

\setcounter{MaxMatrixCols}{10}
%TCIDATA{OutputFilter=LATEX.DLL}
%TCIDATA{Version=5.00.0.2552}
%TCIDATA{Codepage=65001}
%TCIDATA{<META NAME="SaveForMode" CONTENT="1">}
%TCIDATA{Created=Tuesday, November 13, 2012 12:43:14}
%TCIDATA{LastRevised=Tuesday, July 25, 2017 14:45:49}
%TCIDATA{<META NAME="GraphicsSave" CONTENT="32">}
%TCIDATA{<META NAME="DocumentShell" CONTENT="Standard LaTeX\Standard LaTeX Article">}
%TCIDATA{Language=American English}
%TCIDATA{CSTFile=40 LaTeX article.cst}

\newtheorem{theorem}{Theorem}
\newtheorem{acknowledgement}[theorem]{Acknowledgement}
\newtheorem{algorithm}[theorem]{Algorithm}
\newtheorem{axiom}[theorem]{Axiom}
\newtheorem{case}[theorem]{Case}
\newtheorem{claim}[theorem]{Claim}
\newtheorem{conclusion}[theorem]{Conclusion}
\newtheorem{condition}[theorem]{Condition}
\newtheorem{conjecture}[theorem]{Conjecture}
\newtheorem{corollary}[theorem]{Corollary}
\newtheorem{criterion}[theorem]{Criterion}
\newtheorem{definition}[theorem]{Definition}
\newtheorem{example}[theorem]{Example}
\newtheorem{exercise}[theorem]{Exercise}
\newtheorem{lemma}[theorem]{Lemma}
\newtheorem{notation}[theorem]{Notation}
\newtheorem{problem}[theorem]{Problem}
\newtheorem{proposition}[theorem]{Proposition}
\newtheorem{remark}[theorem]{Remark}
\newtheorem{solution}[theorem]{Solution}
\newtheorem{summary}[theorem]{Summary}
\newenvironment{proof}[1][Proof]{\noindent\textbf{#1.} }{\ \rule{0.5em}{0.5em}}
%\input{tcilatex}

\begin{document}

\title{Competitiveness and Carbon Taxes}
\author{Kortum and Weisbach \\
%EndAName
Yale and University of Chicago}
\maketitle

\begin{abstract}
Blah blah blah
\end{abstract}

\section{Leakage}

We define \emph{proportional leakage} $l_{P}$ as the proportional change in
emissions in $\mathcal{F}$ (brought about by a unilateral carbon tax in $%
\mathcal{H}$) relative to the proportional change in global emissions $G$
(for any unilateral carbon tax worth considering we can presume $G<1$). A
value of $l_{P}=1$ means that emissions in both countries fall in parallel
while a value of $1/G$ means that emissions in $\mathcal{F}$ are unchanged.
Values of $l_{P}$ in excess of $1/G$ indicate a classic case of leakage, in
which the policy in $\mathcal{H}$ to reduce global emissions leads $\mathcal{%
F}$ to actually increase its emissions.

The tax in $\mathcal{H}$ is a combination of a production tax $t_{p}^{\prime
}$ and a border tax adjustment $t_{b}^{\prime }\in \lbrack 0,t_{p}^{\prime
}] $. With no border adjustments ($t_{b}^{\prime }=0$) it is a pure
production tax (at rate $t_{p}^{\prime }$) while with full border
adjustments ($t_{b}^{\prime }=t_{p}^{\prime }$) it is a pure consumption tax
(at rate $t_{b}^{\prime }$). The reduction in global emissions $G$ under
such policies is acheived by driving down the price of energy faced by the
extraction sector, leading that sector to supply less energy on the world
market. This reduction in the price of energy $\hat{p}_{e}$, known as the 
\emph{fuel price effect}, is connected to the reduction in global emissions
via the energy supply curve:%
\begin{equation*}
\hat{p}_{e}=G^{(1-\beta )/\beta }.
\end{equation*}

To simplify a bit, we focus on the case of $\eta =1$, in which energy is
consumed only indirectly through consumption of manufactures. Leakage can
then be decomposed into (i) the change in $\mathcal{F}$'s market share in
manufacturing multiplied by (ii) the change in world spending on the
manufactured good (which equals the change in spending on energy, since
energy has a fixed share in the value of manufactured goods) all divided by
(iii) the change in the energy price and the reduction in global emissions:%
\begin{equation*}
l_{P}=\frac{\left( 1-\bar{j}^{\prime }\right) \left( \pi _{c}^{\prime
}Y^{\prime }+\pi _{c}^{\ast \prime }Y^{\ast \prime }\right) }{\left( 1-\bar{j%
}\right) \left( \pi _{c}Y+\pi _{c}^{\ast }Y^{\ast }\right) }\frac{1}{\hat{p}%
_{e}G}.
\end{equation*}%
Defining $\mathcal{H}$'s share of world spending on the $c$-good by%
\begin{equation*}
\omega _{c}=\frac{\pi _{c}Y}{\pi _{c}Y+\pi _{c}^{\ast }Y^{\ast }},
\end{equation*}%
we get:%
\begin{equation*}
l_{P}=\frac{1-\bar{j}^{\prime }}{1-\bar{j}}\left( \omega _{c}\hat{\pi}_{c}%
\hat{Y}+\left( 1-\omega _{c}\right) \hat{\pi}_{c}^{\ast }\hat{Y}^{\ast
}\right) G^{-1/\beta },
\end{equation*}%
which brings out the role of changes in spending shares on manufactures in
each country.

Consider these parts separately. The change in $\mathcal{F}$'s market share
in manufactures (which is the change in its trade share, given costless
trade) is:%
\begin{equation*}
\frac{1-\bar{j}^{\prime }}{1-\bar{j}}=\frac{1}{\bar{j}\left( \frac{%
1+t_{p}^{\prime }}{1+t_{b}^{\prime }}\right) ^{-\theta \left( 1-\gamma
\right) }+1-\bar{j}}.
\end{equation*}%
This change is increasing in the effective production tax rate $\tilde{t}%
_{p}^{\prime }$ defined by $1+\tilde{t}_{p}^{\prime }=(1+t_{p}^{\prime
})/(1+t_{b}^{\prime })$. It is the component of leakage that is independent
of the fuel price effect.

The other components of leakage arise from the fuel price effect. The change
in the share of income spent on the manufactured good in $\mathcal{H}$ is:%
\begin{equation*}
\hat{\pi}_{c}=\frac{\hat{p}_{c}^{-\left( \sigma -1\right) }}{\pi _{c}\hat{p}%
_{c}^{-\left( \sigma -1\right) }+1-\pi _{c}}
\end{equation*}%
and likewise for $\mathcal{F}$. Changes in these shares depend on changes in
the price of the manufactured good. In $\mathcal{H}$ that change is given by:%
\begin{eqnarray*}
\hat{p}_{c} &=&\hat{p}_{m}=\hat{p}_{e}^{1-\gamma }\left( \bar{j}\left(
1+t_{p}^{\prime }\right) ^{-\theta \left( 1-\gamma \right) }+\left( 1-\bar{j}%
\right) \left( 1+t_{b}^{\prime }\right) ^{-\theta \left( 1-\gamma \right)
}\right) ^{-1/\theta } \\
&=&\left( 1+t_{b}^{\prime }\right) ^{\left( 1-\gamma \right) }\hat{p}%
_{e}^{1-\gamma }\left( \bar{j}\left( \frac{1+t_{p}^{\prime }}{%
1+t_{b}^{\prime }}\right) ^{-\theta \left( 1-\gamma \right) }+\left( 1-\bar{j%
}\right) \right) ^{-1/\theta },
\end{eqnarray*}%
while in $H$:%
\begin{equation*}
\hat{p}_{c}^{\ast }=\hat{p}_{e}^{1-\gamma }\left( \bar{j}\left( \frac{%
1+t_{p}^{\prime }}{1+t_{b}^{\prime }}\right) ^{-\theta \left( 1-\gamma
\right) }+\left( 1-\bar{j}\right) \right) ^{-1/\theta }.
\end{equation*}%
The last term in each expression is the proportional increase in the price
of the manufactured good caused by the effective production tax, which
raises costs in $H$ and leads to a change in the goods that each country
specializes in. The fuel price effect partially offsets this price increase.
Only the border tax adjustment causes prices to rise more in $\mathcal{H}$
than in $\mathcal{F}$:%
\begin{equation*}
\frac{\hat{p}_{c}}{\hat{p}_{c}^{\ast }}=\left( 1+t_{b}^{\prime }\right)
^{\left( 1-\gamma \right) }.
\end{equation*}

\subsection{Pure Production Tax}

In the case of a pure production tax ($t_{b}^{\prime }=0$), these price
changes would be the same. If the initial shares were the same ($\pi
_{c}=\pi _{c}^{\ast }$) then changes in shares would be the same $\hat{\pi}%
_{c}=\hat{\pi}_{c}^{\ast }$. In this case leakage is simply:%
\begin{eqnarray*}
l_{P} &=&\hat{\pi}_{c}\frac{\omega _{c}\hat{Y}+\left( 1-\omega _{c}\right) 
\hat{Y}^{\ast }}{\bar{j}\left( \frac{1+t_{p}^{\prime }}{1+t_{b}^{\prime }}%
\right) ^{-\theta \left( 1-\gamma \right) }+1-\bar{j}}G^{-1/\beta } \\
&=&
\end{eqnarray*}

\subsection{Pure Consumption Tax}

In the case of a pure consumption tax ($t_{b}^{\prime }=t_{p}^{\prime }$),
there is no change in $F$'s trade share:%
\begin{equation*}
\frac{1-\bar{j}^{\prime }}{1-\bar{j}}=\frac{1}{\bar{j}+1-\bar{j}}=1.
\end{equation*}%
Thus leakage is due only to the increase in foreign consumption brou%
\begin{equation*}
l_{P}=\left( \omega _{c}\hat{\pi}_{c}\hat{Y}+\left( 1-\omega _{c}\right) 
\hat{\pi}_{c}^{\ast }\hat{Y}^{\ast }\right) G^{-1/\beta }
\end{equation*}

\section{Welfare}

The ultimate goal of any carbon taxing scheme is to maximize welfare conditional on meeting emissions reduction criteria. In this model, a country's welfare is its spending power - more precisely, its income divided by a price index of energy, manufactures, and the l-good (see :

\begin{equation*}
W = \frac{Y}{p}
\end{equation*}
After taxes, welfare can change, and we denote this as before with a hat:

\begin{equation*}
\hat{W} = \frac{W^\prime}{W}
\end{equation*}
These expressions are analogous for Foreign with stars. There is one more relevant measure which is world welfare. Since welfare can differ between Home and Foreign, world welfare is expressed in expectation, weighted by the sizes of the countries' respective labor forces:

\begin{equation}
W_{world} = \omega_L*W + \omega_L^\star*W^\star
\end{equation}
where $\omega_L$ and $\omega_L^\star$ denote the world share of labor returns in Home and Foreign, respectively (recall that wages are equalized):

\begin{align}
&\omega_L = \frac{\pi_L*Y_{rel}}{\pi_L*Y_{rel} + \pi_L^\star}\\
&\omega_L^\star = \frac{\pi_L^\star}{\pi_L*Y_{rel} + \pi_L^\star}
\end{align}

\end{document}
