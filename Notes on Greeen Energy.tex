%2multibyte Version: 5.50.0.2960 CodePage: 65001
%\input{tcilatex}
%\input{tcilatex}
%\input{tcilatex}


\documentclass[notitlepage,12pt]{article}
%%%%%%%%%%%%%%%%%%%%%%%%%%%%%%%%%%%%%%%%%%%%%%%%%%%%%%%%%%%%%%%%%%%%%%%%%%%%%%%%%%%%%%%%%%%%%%%%%%%%%%%%%%%%%%%%%%%%%%%%%%%%%%%%%%%%%%%%%%%%%%%%%%%%%%%%%%%%%%%%%%%%%%%%%%%%%%%%%%%%%%%%%%%%%%%%%%%%%%%%%%%%%%%%%%%%%%%%%%%%%%%%%%%%%%%%%%%%%%%%%%%%%%%%%%%%
\usepackage{amsfonts}
\usepackage{amsmath}

\setcounter{MaxMatrixCols}{10}
%TCIDATA{OutputFilter=LATEX.DLL}
%TCIDATA{Version=5.00.0.2552}
%TCIDATA{Codepage=65001}
%TCIDATA{<META NAME="SaveForMode" CONTENT="1">}
%TCIDATA{Created=Tuesday, November 13, 2012 12:43:14}
%TCIDATA{LastRevised=Thursday, August 10, 2017 19:02:16}
%TCIDATA{<META NAME="GraphicsSave" CONTENT="32">}
%TCIDATA{<META NAME="DocumentShell" CONTENT="Standard LaTeX\Standard LaTeX Article">}
%TCIDATA{Language=American English}
%TCIDATA{CSTFile=40 LaTeX article.cst}

\newtheorem{theorem}{Theorem}
\newtheorem{acknowledgement}[theorem]{Acknowledgement}
\newtheorem{algorithm}[theorem]{Algorithm}
\newtheorem{axiom}[theorem]{Axiom}
\newtheorem{case}[theorem]{Case}
\newtheorem{claim}[theorem]{Claim}
\newtheorem{conclusion}[theorem]{Conclusion}
\newtheorem{condition}[theorem]{Condition}
\newtheorem{conjecture}[theorem]{Conjecture}
\newtheorem{corollary}[theorem]{Corollary}
\newtheorem{criterion}[theorem]{Criterion}
\newtheorem{definition}[theorem]{Definition}
\newtheorem{example}[theorem]{Example}
\newtheorem{exercise}[theorem]{Exercise}
\newtheorem{lemma}[theorem]{Lemma}
\newtheorem{notation}[theorem]{Notation}
\newtheorem{problem}[theorem]{Problem}
\newtheorem{proposition}[theorem]{Proposition}
\newtheorem{remark}[theorem]{Remark}
\newtheorem{solution}[theorem]{Solution}
\newtheorem{summary}[theorem]{Summary}
\newenvironment{proof}[1][Proof]{\noindent\textbf{#1.} }{\ \rule{0.5em}{0.5em}}
\input{tcilatex}

\begin{document}

\title{Incorporating Green Energy Supply}
\author{Kortum and Weisbach \\
%EndAName
Yale and University of Chicago}
\maketitle

\begin{abstract}
\end{abstract}

\section{Energy Supply}

Our starting point is a production function for extraction of a quantity $%
Q_{e}$ of energy:%
\begin{equation}
Q_{e}=\left( L_{e}/\beta \right) ^{\beta }E^{1-\beta },  \label{e production}
\end{equation}%
where $L_{e}$ is labor input, $E$ is an energy endowment, and $\beta $ is
labor's share. (We derive this production function from more primitive
assumptions about extraction in the Appendix.) The energy supply curve is:%
\begin{equation*}
Q_{e}=\left( \frac{p_{e}}{w}\right) ^{\beta /(1-\beta )}E,
\end{equation*}%
with energy supply elasticity $\varepsilon _{S}=\beta /(1-\beta )$.

The problem is the same in $\mathcal{F}$. We'll assume $\beta $ with $\beta
^{\ast }$ for simplicity, $w=w^{\ast }=1$ due to costless trade in services
and $p_{e}=p_{e}^{\ast }$ due to costless trade in energy. In that setting
the global supply of energy is:%
\begin{equation*}
Q_{e}^{W}=Q_{e}+Q_{e}^{\ast }=p_{e}^{\beta /(1-\beta )}E^{W},
\end{equation*}%
where $E^{W}=E+E^{\ast }$. Consider changes over time, given endowments and
no extraction taxes. Let $G=$ $\hat{Q}_{e}^{W}$ be the change in global
emissions, i.e. emissions are proportional to energy use. With $\hat{p}_{e}$
the change in the energy price, we have:%
\begin{equation*}
G=\hat{p}_{e}^{\beta /(1-\beta )}.
\end{equation*}%
Global emissions are reduced if and only if the pre-tax price of energy
falls. Now consider how the picture changes if we introduce renewables, so
that global emissions are no longer proportional to energy use.

\subsection{Fossil Fuels and Renewables}

We distinguish between renewable\ energy, that involves no carbon emissions,
and fossil fuels. Let $Q_{f}$ be the quantity of fossil fuels and $Q_{r}$
the quantity of renewables, so that $Q_{e}=Q_{f}+Q_{r}$. It is convenient to
model these energy sources in a parallel manner:%
\begin{equation*}
Q_{f}=\left( L_{f}/\beta \right) ^{\beta }E_{f}^{1-\beta }
\end{equation*}%
and%
\begin{equation*}
Q_{r}=\left( L_{r}/\beta _{r}\right) ^{\beta _{r}}E_{r}^{1-\beta _{r}},
\end{equation*}%
where we cn start with the simple case $\beta _{r}=\beta $. Since these two
forms of energy are perfect substitutes in production, their price may
differ only due to their tax treatment.

Suppose there is a carbon tax on production at rate $t_{p}$, raising the
after-tax price of fossil fuels from $p_{f}$ to $(1+t_{p})p_{f}$. Since
renewables do not generate carbon emissions, they would naturally be exempt
from such a tax. In an equilibrium in which both fossil fuels and renewables
are used, the price of renewables will be:%
\begin{equation*}
p_{r}=(1+t_{p})p_{f}.
\end{equation*}%
In this case, total energy supply in $\mathcal{H}$ is:%
\begin{eqnarray*}
Q_{e} &=&p_{f}^{\beta /(1-\beta )}E_{f}+p_{r}^{\beta /(1-\beta )}E_{r} \\
&=&p_{e}^{\beta /(1-\beta )}\left( E_{f}+\left( 1+t_{p}\right) ^{\beta
/(1-\beta )}E_{r}\right) .
\end{eqnarray*}%
Notice how the production tax raises the supply of energy, given the price
of energy $p_{e}$. The mechanism is that the production tax makes marginal
forms of renewable energy cost efficient.

Suppose fossil fuels are traded at no cost, so that their world price is $%
p_{f}=p_{e}$. It's more realistic to assume that renewables are not traded.
Consider a border adjustment $t_{b}$ layered on top of the production tax $%
t_{p}$. Thus producers in $\mathcal{F}$ face an effective fuel price of $%
(1+t_{b})p_{f}$ if they use fossil fuels in producing for the export market.
If they use renewables they are exempt from the border adjustment. If, in
equilibrium, renewables are not sufficient to cover energy demand in
producing for the export market, the price of renewables in foreign will be:%
\begin{equation*}
p_{r}^{\ast }=(1+t_{b})p_{f}.
\end{equation*}%
The border tax in $\mathcal{H}$ segments the energy market in $\mathcal{F}$,
raising the price of renewables. While there will be fuel switching to serve
the export market, the higher price will bring new sources of renewables
into production. This requires that producers in $\mathcal{F}$ certify to $%
\mathcal{H}$ that they are using renewables since otherwise the equilibrium
price of renewables in $\mathcal{F}$ would remain at $p_{f}$ (and so
production of renewables would not be stimulated).

Although renewables add to world energy supply, they don't change the simple
link between changes in global emissions and the change in price of energy:%
\begin{equation*}
G=\frac{Q_{f}^{W\prime }}{Q_{f}^{W}}=\frac{p_{e}^{\prime \beta /(1-\beta
)}E_{f}^{W}}{p_{e}^{\beta /(1-\beta )}E_{f}^{W}}=\hat{p}_{e}^{\beta
/(1-\beta )}.
\end{equation*}%
Thus, as before, a goal for reduction in global emissions nails down the
necessary reduction in the price of energy. But, note that we no longer have 
$G=\hat{Q}_{e}^{W}$.

\end{document}
