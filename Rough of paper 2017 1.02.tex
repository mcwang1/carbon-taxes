%2multibyte Version: 5.50.0.2960 CodePage: 65001
%\input{tcilatex}


\documentclass[notitlepage,12pt]{article}
%%%%%%%%%%%%%%%%%%%%%%%%%%%%%%%%%%%%%%%%%%%%%%%%%%%%%%%%%%%%%%%%%%%%%%%%%%%%%%%%%%%%%%%%%%%%%%%%%%%%%%%%%%%%%%%%%%%%%%%%%%%%%%%%%%%%%%%%%%%%%%%%%%%%%%%%%%%%%%%%%%%%%%%%%%%%%%%%%%%%%%%%%%%%%%%%%%%%%%%%%%%%%%%%%%%%%%%%%%%%%%%%%%%%%%%%%%%%%%%%%%%%%%%%%%%%
\usepackage{amsfonts}
\usepackage{amsmath}

\setcounter{MaxMatrixCols}{10}
%TCIDATA{OutputFilter=LATEX.DLL}
%TCIDATA{Version=5.50.0.2960}
%TCIDATA{Codepage=65001}
%TCIDATA{<META NAME="SaveForMode" CONTENT="1">}
%TCIDATA{BibliographyScheme=Manual}
%TCIDATA{Created=Tuesday, November 13, 2012 12:43:14}
%TCIDATA{LastRevised=Monday, August 28, 2017 16:42:37}
%TCIDATA{<META NAME="GraphicsSave" CONTENT="32">}
%TCIDATA{<META NAME="DocumentShell" CONTENT="Standard LaTeX\Standard LaTeX Article">}
%TCIDATA{Language=American English}
%TCIDATA{CSTFile=40 LaTeX article.cst}

\newtheorem{theorem}{Theorem}
\newtheorem{acknowledgement}[theorem]{Acknowledgement}
\newtheorem{algorithm}[theorem]{Algorithm}
\newtheorem{axiom}[theorem]{Axiom}
\newtheorem{case}[theorem]{Case}
\newtheorem{claim}[theorem]{Claim}
\newtheorem{conclusion}[theorem]{Conclusion}
\newtheorem{condition}[theorem]{Condition}
\newtheorem{conjecture}[theorem]{Conjecture}
\newtheorem{corollary}[theorem]{Corollary}
\newtheorem{criterion}[theorem]{Criterion}
\newtheorem{definition}[theorem]{Definition}
\newtheorem{example}[theorem]{Example}
\newtheorem{exercise}[theorem]{Exercise}
\newtheorem{lemma}[theorem]{Lemma}
\newtheorem{notation}[theorem]{Notation}
\newtheorem{problem}[theorem]{Problem}
\newtheorem{proposition}[theorem]{Proposition}
\newtheorem{remark}[theorem]{Remark}
\newtheorem{solution}[theorem]{Solution}
\newtheorem{summary}[theorem]{Summary}
\newenvironment{proof}[1][Proof]{\noindent\textbf{#1.} }{\ \rule{0.5em}{0.5em}}
\input{tcilatex}
\begin{document}

\title{Unilateral Carbon Taxes and International Trade: An Analytic General
Equlibrium Model}
\author{Sam Kortum, Michael Wang, and David A. Weisbach \\
%EndAName
Yale, Northwestern, and the University of Chicago}
\maketitle

\begin{abstract}
Blah blah blah
\end{abstract}

\section{Introduction}

Leakage, defined roughly as an increase in emissions abroad in response to a
carbon tax at home, is a central concern for domestic climate change policy.
Originally, in response to the dichotomy introduced by the Kyoto Protocol,
leakage was a concern because developed countries were to adopt carbon
prices while developing were not. The concern was that in response to this
dichotomy, heavy industry would flee developed countries, leading to
inefficiencies in the location of production and making carbon prices
ineffective. Even without such extreme differentiation, leakage is still
concern because nations will inevitably have different carbon prices
(whether explicit, such as through a tax, subsidies, or tradable permits, or
implicit, such as through regulations). For example, under the Paris
Agreement, nations have different emissions reductions goals, implying
different shadow carbon prices, generating the possibility of leakage.

Because of its centrality to climate change policy, there have been a large
number of studies of leakage. Most studies use computable general
equilbirium models to study the issue. CGE models have the advantage of
detailed representations of the economy, often including particularly
detailed representations of the energy sector. Their disadvantage is that
they are non-transparent. Moreover, validiation of CGE results is not
straightforward, which means that the additional detail these models offer
may not produce meaningful benefits.

There are also a smaller number of analytic approaches to studying leakage
(described below). [What should we say about these?]

We study unilateral carbon prices using a stylized general equilibrium model
of international trade. The model allows nations to impose carbon taxes at
different stages of production and on consumption and allows nations to
impose border adjustments. Border adjustments are taxes on the importation
of goods produced using fossil fuels and rebates of any prior carbon taxes
paid on the export of goods. They are the primary policy tool suggested to
address the distortions from unilateral pricing. We derive a number of
analytic results. In other cases, calibrate and numerically solve model. In
these cases, look for results that are robust to choices of parameters.

The policy tradeoffs are surprisingly subtle in this context. While leakage
is touted as a prominent indicator of distortions from unilateral carbon
pricing, we find that leakage has little relationship to welfare. A policy
minimizing leakage is rarely optimal (for a given objective of global carbon
reduction). [We also show that full border adjustments are unlikely to be
optimal, even leaving aside their considerable administrative costs (for a
discussion of these costs, see Kortum and Weisbach 2017))

\subsection{Prior literature}

\subsubsection{CGE approaches}

The CGE literature is too large to summarize in detail. The major studies
are cited in the notes.\footnote{%
Work includes Alexeeva-Talebi et al. (2012), Babiker (2005), Balistreri and
Rutherford (2012), Bednar-Friedl, Schinko, and Steininger (2012), Boeters
and Bollen (2012), B\"{o}hringer, Balistreri, and Rutherford (2012), B\"{o}%
hringer et al. (2012), B\"{o}hringer, Carbone, and Rutherford (2012), B\"{o}%
hringer et al. (2012),Branger and Quirion (2014a), Branger and Quirion
(2014a), Caron (2012), Dong and Walley (2012), Dr\"{o}ge (2009), Elliott et
al. (2010), Felder and Rutherford (1993), Jakob, Marschinski, and H\"{u}bler
(2013), Jakob, Steckel, and Edenhofer (2014), Lockwood and Whalley (2010),
Richels, Blanford, and Rutherford (2009), van Asselt and Brewer (2010),
Weitzel, H\"{u}bler, and Peterson (2012), and Winchester, Paltsev, and
Reilly (2011), Kuik and Gerlagh (2003), Kuik and Hofkes (2010), Kuik and
Verbruggen (2002), Monjon and Quirion (2011), Helm, Hepburn, and Ruta
(2012), de Cendra (2006), Fischer and Fox (2011), Fischer and Fox (2012a),
Fischer and Fox (2012b), and Elliott et al. (2013).} There are five findings
that seem to be robust across most of the CGE models. They are:

\begin{itemize}
\item Leakage rates are most often in the range of 5\% to 20\% (with leakage
defined as the increase in emissions in non-taxing regines as a percent of
the reduction in emisisons in the taxing region). There are some outliers.

\item The larger the taxing coalition, the lower the leakage.

\item Border adjustments reduce leakage substantially, with leakage rates
under border adjustments ranging from \_\_ to \_\_.

\item The most important variable in determining effects of unilateral
carbon price is the elasticity of energy supply.

\item The models distinguish two drivers of leakage: the fuel price effect
(in which lower demand for fossil fuels in the taxing region suppresses
prices, increasing demand on the non-taxing regions) and the competitiveness
effect (in which increased costs for industry in the taxing region causes a
shift to the non-taxing region).
\end{itemize}

\subsubsection{Analytic approaches}

Small body of work has used analytic models to understand leakage. Markusen
(1975), which focuses on environmental harms more generally, is earliest
example.

Jakob, Marschinski and Hubler (2013) use a version of Markusen (1975),
modified to allow different sectors in the economy to have different
emissions intensities. Their key finding is that the relative intensities of
the exporting and non-exporting sectors in the non-taxing region affect
whether BA's reduce leakage.

Bohringer Lange and Rutherford (2014) use an analytic model to consider
differential carbon price, such as lower taxes on trade-exposed sectors.
They decompose the effects of differential taxes into leakage effects and
terms of trade effects. They incorporate this decomposition into a CGE model
to produce guidelines for carbon pricing.

Fischer and Fox (2012).

Fullerton, Karney and Baylis (2014) use an analytic general equilibrium
model to analyze how capital used in abatement of emissions effects leakage.
They find that if abatement is resource intensive (such as requiring
substantial capital), a domestic carbon tax can produce negative leakage
because it increases global resource costs and hence investment in
non-taxing regions.

\section{Model structure}

Our goal is to develop a relatively simple and transparent general
equilibrium model that captures the effects of carbon taxes on trade. We
purposefully suppress many elements commonly found in CGE\ models so our
model is analytically tractable.

In particular, we assume that the world is divided into only two countries
or regions, which we call Home and Foreign, and that there are only two
factors of production, labor and deposits of energy both of which are
immobile. Each country has a fixed endowment of each factor, and the
countries may differ in their endowments.

Production takes place in two stages. First, firms in each country use labor
to extract deposits, producing usable energy, which is costlessly traded.
Although they use the same extraction technology, the countries may differ
in their cost of extraction (as measured in units of labor). In our base
model, there is only one type of energy, fossil fuel energy, which produces
carbon dioxide when used. In an extension, we add renewable energy.

In the second stage of production, firms in each country use energy
(creating emissions) and labor to produce manufactured goods. One of our
goals is to investigate how taxes affect the location of production. To
allow for these affects, we assume that each country has a comparative
advantage in particular varieties of manufactured goods, determined by a
version of the Ricardian model introduced by Dornbush, Fisher and Samuelson
(1977).

Each country also produces services using only labor. Therefore, labor in
each country is used in three ways: to extract deposits, to work in
manufacturing, and to produce services.

Services play an important role in the model. We assume trade balances,
which means that if taxes shift where extraction or production takes place
(which is a central concern with unilateral carbon taxes), the labor used to
produce services effectively acts as a residual. For example, if the Home
country imposes a tax on the use of energy in production and, because of the
resulting change in comparative advantage, some production moves offshore to
the Foreign country, labor in the Home country shifts to producing services
(and likewise, labor in the Foreign country shifts away from producing
services). Taxes, in this case, effectively shift the energy intensity of
the two countries by shifting where manufacturing takes place and where the
production of services takes place.

Consumers in each country earn income by working in one of the three sectors
and by receiving rents from deposits. Consumers use their income to purchase
manufactured goods and services to maximize utility, closing the model.
Consumers can purchase goods produced in either country without trade costs.

Our goal is to consider how taxes affect elements of the model, including
where extraction, production, and consumption take place, how emissions
change in each of the two countries (including a measure of leakage), and
how taxes affect welfare. To do this, we consider a very general set of
taxes: each country can impose taxes on energy at each stage of production
or on the consumption of goods created using energy. In particular, each
country can impose a tax on the extraction of energy (an extraction tax), on
the use of energy in production (a production tax), and on the purchase of
goods that were produced using energy (a consumption tax).

We also allow the Home country to impose border adjustments, which are taxes
on the energy used in the production of imported goods and rebates of taxes
paid for the use of energy on exported goods. In most models, border
adjustments are imposed when a country has a tax on production, and the
border adjustment tax rate is equal to the production tax rate. One of our
goals is to determine whether and the extent to which border adjustments are
optimal, so we allow border adjustments to be imposed at any rate between
zero (i.e., no border adjustments) and the production tax rate. In general,
we find that a corner solution is not optimal.

To make the model tractable, we assume relatively simply functional forms:
Cobb-Douglas production functions and a CES utility function. In particular,
we parameterize the model as follows.

\subsection{Consumption}

Consumers in the two countries, Home ($H$) and Foreign ($F$) consume
services (the $l$-good) and a composite manufactured good (the $m$-good).
These goods enter utility via a CES utility function with an elasticity of
substitution $\sigma $:

\begin{equation}
U\left( C_{m},C_{l}\right) =\left( \alpha ^{1/\sigma }C_{m}{}^{\left( \sigma
-1\right) /\sigma }+\left( 1-\alpha \right) ^{1/\sigma }C_{l}^{\left( \sigma
-1\right) /\sigma }\right) ^{\sigma /\left( \sigma -1\right) }.
\label{Utility}
\end{equation}%
The $m$-good is itself a Dixit-Stiglitz aggregate over a unit continuum of
individual manufactured goods\footnote{%
We could allow for an arbitrary elasticity of substitution within the Dixit
Stiglitz aggregator but since its value plays no role in the model, we let
it be $\sigma $ without loss of generality.}:%
\begin{equation*}
C_{m}=\left( \int_{0}^{1}C_{m}(j)^{\left( \sigma -1\right) /\sigma
}dj\right) ^{\sigma /\left( \sigma -1\right) }.
\end{equation*}

[In our simulations, we will allow the two countries to have different
parameters for their utility functions ($\alpha $, and $\sigma $) but for
our analytic model, we constrain them to be the same.]

[somewhere in the notes, we say that the parameters of the utility function
can be different across countries. Are we allowing this? It doesn't seem
like it in the analytic section but maybe in the simulations?] 

Consumer optimization gives a demand function for the $m$-good of:

\begin{equation*}
D\left( p_{m}\right) =\frac{\alpha p_{m}^{-\left( \sigma -1\right) }}{%
ap_{m}^{-\left( \sigma -1\right) }+\left( 1-\alpha \right) w^{-\left( \sigma
-1\right) }},
\end{equation*}%
where $p_{m}$ is the aggregate price of the $m$-good and $w$ is the wage
rate. Spending in $H$ on the $m$-good is:

\begin{equation*}
p_{m}C_{m}=D\left( p_{m}\right) Y.
\end{equation*}%
A parallel result holds in $F$. We denote all quantities in $F$ using an
asterisk, giving us $p_{m}^{\ast }C_{m}^{\ast }=D\left( p_{k}^{\ast }\right)
Y^{\ast }$.

\subsubsection{Consumption taxes}

Each country can impose an ad valorem tax, $t_{c}$ and $t_{c}^{\ast }\,$, on
the energy content of the $m$-good (and, in our extended model in the
simulations, on the direct consumption of energy, such as for home heating).
The energy content of a good is the total energy used in production of the
good. For example, a consumption tax on an automobile includes a tax on the
energy used to produce the steel in the automobile even if the consumer does
not directly use that energy or produce the resulting emissions.

Firms that sell $m$-goods will have an incentive to shift their mix of
inputs because the use of energy as an input is taxed while the use of labor
is not. As a result, the price of a good that bears a consumption tax on its
energy content must be determined based in equilibrium behavior of firms. To
derive this expression we must first specify the production technology, so
we defer this to Part \_.

\subsection{Production}

\ Upstream, the two final goods are produced as follows.

\subsubsection{Services (the $l$-good)}

Production of the $l$-good in a $H$ is

\begin{equation*}
Q_{l}=L_{l}.
\end{equation*}%
and similarly for $F:Q_{l}^{\ast }=L_{l}^{\ast }$. Labor is measured
efficiency units, so this formulation can capture differences in labor
productivity across countries. We only consider equilibria where each
country supplies services, and services are freely traded. We set the $l$%
-good in $F$ to be the numeraire, which allows us to set $w=w^{\ast }=1.$

\subsubsection{The $m$-good}

As noted, production of the $m$-good takes place in two stages.

\paragraph{Extraction}

First, firms extract energy using labor $L_{e}$ and desposits $E$ in a
Cobb-Douglas production function with labor share $\beta $. In $H$, we have:

\begin{equation*}
Q_{e}=\left( \frac{L_{e}}{\beta }\right) ^{\beta }E^{1-\beta }.
\end{equation*}%
(We derive this production function from more primitive assumptions about
extraction in the Appendix.) Extraction of deposits in $F$ uses the same
production function, replacing $\beta $ with $\beta ^{\ast }$, $E$ with $%
E^{\ast }$. Differences in $\beta $ and $\beta ^{\ast }$ capture differences
in the labor share of energy extraction, such as between a country like
Saudi Arabia where energy is cheap to extract (low $\beta $) and Canada,
where energy is costly to extract (high $\beta $).

Each country can impose an ad valorem tax on extraction, $t_{e}$ and $%
t_{e}^{\ast }$. If the market price of energy is $p_{e}$, firms in $H$
engaging in extraction receive only $p_{e}/\left( 1+t_{e}\right) $. With
this tax and recalling that $w=1$, we get an energy supply curve in $H$ of 
\begin{equation*}
Q_{e}=\left( \frac{p_{e}}{1+t_{e}}\right) ^{\beta /(1-\beta )}E.
\end{equation*}%
The supply curve for energy in $F$ is similar, givng us output of $%
Q_{e}^{\ast }.$

World supply of energy is the sum of the energy produced in $H$ and $F$:

\begin{equation}
Q_{e}^{W}=Q_{e}+Q_{e}^{\ast }=\left( \frac{p_{e}}{1+t_{e}}\right) ^{\beta
/(1-\beta )}E+\left( \frac{p_{e}}{1+t_{e}^{\ast }}\right) ^{\beta ^{\ast
}/(1-\beta ^{\ast })}E^{\ast }.  \label{Global supply of energy}
\end{equation}

\paragraph{Production}

Firms in each country use energy and labor to produce varieties of the $m$%
-good with Cobb-Douglas technology and labor share $\gamma $. To model
specialization in manufacturing and trade, we use a version of the Ricardian
model introduced by Dornbusch, Fisher, and Samuelson (1977). In particular,
let there be a continuous variety of $m$-goods indexed by $j\in \lbrack 0,1]$%
. The production function for variety $j$ in $\mathcal{H}$ is:%
\begin{equation*}
Q_{m}(j)=A\left( j\right) \left( \frac{L_{m}(j)}{\gamma }\right) ^{\gamma
}\left( \frac{M(j)}{1-\gamma }\right) ^{1-\gamma },
\end{equation*}%
where $A(j)$ is $\mathcal{H}$'s productivity in variety $j$, $M\left(
j\right) $ is energy, and $\gamma $ is labor's share. There are many
price-taking producers have access to the technology to produce each variety 
$j$. In $\mathcal{F}$ the production function has the same form, with
productivity $A^{\ast }(j)$.

Productivity for each variety is%
\begin{equation*}
A(j)=\frac{A}{j^{1/\theta }}
\end{equation*}%
in $\mathcal{H}$ and%
\begin{equation*}
A^{\ast }(j)=\frac{A^{\ast }}{\left( 1-j\right) ^{1/\theta }}
\end{equation*}%
in $\mathcal{F}$. The parameters $A$ and $A^{\ast }$ capture absolute
advantage in $\mathcal{H}$ and $\mathcal{F}$. The relative productivity of
the two countries in producing variety $j$ is:%
\begin{equation}
R(j)=\frac{A(j)}{A^{\ast }(j)}=\frac{A}{A^{\ast }}\left( \frac{j}{1-j}%
\right) ^{-1/\theta }.  \label{relative productivity}
\end{equation}%
For $j<j^{\prime }$, $\mathcal{H}$ has a comparative advantage in variety $j$
and $\mathcal{F}$ in $j^{\prime }$. The parameter $\theta $ captures
(inversely) the strength of comparative advantage. As $\theta \rightarrow
\infty $ relative productivity does not vary across varieties.

\paragraph{Production tax}

Each country can, if it chooses, impose an ad valorem tax on the use of
energy in production, raising the price from $p_{e}$ to $\left(
1+t_{p}\right) p_{e}$ in $H$ and $\left( 1+t_{p}^{\ast }\right) p_{e}$ in $F$%
. Because the energy share of production is $\left( 1-\gamma \right) $, this
raises the cost of production by $\left( 1+t_{e}\right) ^{1-\gamma }$ in $H$
and correspondingly in $F$ if $F$ imposes a production tax.

\paragraph{Specialization in $m$-goods}

Individual varieties of the $m$-good are costlessly traded, which means that
specialization in production of manufactured goods is detached from
country-level demand. Due to trade in energy, prices $p_{e}$ are the same in
each country. A bundle of inputs costs $\left( 1+t_{p}\right) ^{1-\gamma
}p_{e}^{1-\gamma }$ in $\mathcal{H}$ and $\left( 1+t_{p}^{\ast }\right)
^{1-\gamma }p_{e}^{1-\gamma }$ in $\mathcal{F}$. Hence $\mathcal{H}$ will
produce varieties $j$ for which $R(j)\geq \frac{\left( 1+t_{p}\right)
^{1-\gamma }}{\left( 1+t_{p}^{\ast }\right) ^{1-\gamma }}$ and $\mathcal{F}$
will produce the rest. This gives us an expression for the good produced in $%
\mathcal{H}$: $\ j\in \left[ 0,\bar{j}\right] $\ where:%
\begin{equation}
\bar{j}=\frac{A^{\theta }\left( 1+t_{p}\right) ^{-\theta \left( 1-\gamma
\right) }}{A^{\theta }\left( 1+t_{p}\right) ^{-\theta \left( 1-\gamma
\right) }+A^{\ast \theta }\left( 1+t_{p}^{\ast }\right) ^{-\theta \left(
1-\gamma \right) }}.  \label{jbar}
\end{equation}%
with goods $j\in \left( \bar{j},1\right] $ produced in $F$.

\paragraph{Price index for manufactured goods}

The price index for the $m$-good is the average price of varieties of the $m$%
-good. We show in the appendix that we can express this as:

\begin{equation*}
p_{m}=\phi p_{e}^{1-\gamma }\left( A^{\theta }\left( 1+t_{p}\right)
^{-\theta \left( 1-\gamma \right) }+A^{\ast \theta }\left( 1+t_{p}^{\ast
}\right) ^{-\theta \left( 1-\gamma \right) }\right) ^{1/\theta },
\end{equation*}%
where

\begin{equation*}
\phi =\left( \frac{\theta }{\theta -\left( \sigma -1\right) }\right)
^{-1/\left( \sigma -1\right) }.
\end{equation*}%
Because the relevant average price is independent of where a variety is
produced, $\bar{j}$ given in (\ref{jbar}) is also the share of world
spending on the $m$-good devoted to producers in $\mathcal{H}$ and $1-\bar{j}
$ the share devoted to producers in $\mathcal{F}$. Note that this share does
not depend on on $\sigma $ . The so-called trade elasticity, giving the
response of trade shares to factor costs, is $\theta $.

\paragraph{Consumption tax}

With this statement of production technology, we can now give an expression
for a consumption tax. A firm producing variety $j$ and facing a consumption
tax on sale, will choose an input mix to maximize revenue, taking prices as
fixed. Solving this problem gives a final price$p_{m}\left( j\right) $ of:%
\begin{equation*}
p_{m}(j)=\left( 1+t_{c}\right) ^{1-\gamma }\frac{p_{e}^{1-\gamma }}{A(j)}.
\end{equation*}%
(recalling that $w=1$). Although this looks like a simple ad valorem tax
because the after-tax price of the $m$-good is the pre-tax price multipled
by $\left( 1+t_{c}\right) ^{1-\gamma },$. this a result of our assumption of
Cobb-Douglas production, and is not general.

[I find this notation confusing because there is nothing to tell the reader
that this is an after-tax price. Can we say something like $p_{m}^{\prime
}\left( j\right) =\left( 1+t_{c}\right) p_{m}\left( j\right) ,$ where the
prime indicates an after-tax price?]

For the $H$ firm's sales to consumers in $\mathcal{F}$, the expression is
the same, substituting the consumption tax, if any, charged to consumers in $%
F$, $t_{c}^{\ast }$:%
\begin{equation*}
p_{m}^{\ast }(j)=\left( 1+t_{c}^{\ast }\right) ^{1-\gamma }\frac{%
p_{e}^{1-\gamma }}{A(j)}.
\end{equation*}%
Note that $p_{m}\left( j\right) $ may not equal $p_{m}^{\ast }\left(
j\right) $ because the consumption tax rates in $H$ and $F$ may not be the
same (for example, with a unilateral consumption tax in $H$, $t_{c}^{\ast }$
would be $0$). A similar derivation gives the consumption tax for firms in $%
F $ for sales to consumers in $H$ and in $F$.

\subsection{Income}

Income comes from labor, rents on energy deposits, and tax revenue:

\begin{equation*}
Y=wL+rE+T_{e}+T_{c}+T_{p}.
\end{equation*}%
where $T_{e}$, $T_{c}$, and $T_{p}\,$, are tax revenues from extraction,
consumption, and production taxes respectively. A similar equation holds for 
$\mathcal{F}$ and world income is the sum of the two: $Y^{W}=Y+Y^{\ast }$.

\subsection{Equilibrium}

Equilibrium consists of a price of energy $p_{e}$ and an allocation of labor
across the three sectors in each country so that markets clear. In
particular, we need the global demand for energy $\left( 1-\gamma \right)
\left( D\left( p_{m}\right) Y+D^{\ast }\left( p_{m}\right) Y^{\ast }\right) $
[make sure this is okay with taxes] to equal global supply as given in
expression (\ref{Global supply of energy}).

It will be convenient in calibrating the model to express the equilibrium in
terms of spending shares. To do this, we define the following shares. The
share of (after-tax) spending on the $m$-good is%
\begin{equation*}
\pi _{m}=\frac{\left( 1+t_{c}\right) p_{m}C_{m}}{Y}=D\left( p_{m}\right) 
\end{equation*}

The share of spending on energy is:%
\begin{equation*}
\pi _{e}=\frac{p_{e}Q_{e}}{Y}
\end{equation*}

And the share of income attributable to labor is 

\begin{equation*}
\pi _{L}=\frac{L}{Y}
\end{equation*}

The equilibrium will determine 

\bigskip 

In many cases below and particularly in our simulations, we are interested
in how the equilibrium changes when we change taxes. We use the following
notation.

Initial position\ (which may include taxes) uses notation above.

New taxes, always has a prime, so might start with a production tax $t_{p}$
and change to a tax of $t_{p}^{^{\prime }}$. Note that this includes the
case where we start with no taxes because we can let $t_{p}=0$.

Change notation for any variable is $\hat{x}=x^{^{\prime }}/x$.

\section{Effects of global taxes}

Our goal is to understand the effects of taxes in $H$. Before turning to
that analysis, we first consider global taxes. Deriviations of these
propositions are in the Appendix.

Start by considering as a base case, a globally harmonized carbon tax, which
means that both countries impose the same tax at the same level of
production.

\begin{proposition}
If the parameters of the utility function, $\alpha $ and $\sigma $ are the
same in $H$ and $F$, a global production tax at rate $t_{p}$ has the same
effect on the price of energy (and, therefore, emissions) as a global
consumption tax and as a global extraction tax at the same rates $\left(
t_{c}=t_{p}=t_{e}\right) .$ These taxes differ in the location of tax
revenue, with the tax revenue tracking the location of the taxed activity.
With a global tax, each country prefers the tax to be imposed where it has
relatively more of the activity.
\end{proposition}

This result arises because all energy that is extracted is used in
production (in manufacturing in the basic model or directly consumed, and
therefore, in home production in the more general model) and all production
is consumed. The difference in the three taxes when imposed globally is
where in the chain of prodution they are imposed.

Identical demand parameters are needed because the location of the tax
revenue creates an income effect: without identical demand parameters, the
income effect would alter consumption choices.Note that this means that with
global taxes, the place that taxes are imposed, can, in theory, be chosen
based entirely on adminsitrative considerations, with offsetting revenue
transfers made between countries determining the distributive effects.

Note that with any of the global taxes, we have $\bar{j}^{`}=\bar{j}\,,$
which can be see by examining the expression for $\bar{j}$ (expression \ref%
{jbar}). This means that there is no change in the location of production.

Define a global emissions reduction goal $G=\frac{Q_{e}^{w\prime }}{Q_{e}^{w}%
}<1$, [where $Q_{e}^{w}$ $=Q_{e}+Q_{e}^{\ast }$ is global energy production
under a baseline policy and $Q_{e}^{w\prime }$ is global energy production
under a proposed tax; also on notation, are we going to use $G$ or $\hat{Q}%
_{e}$? We use hats for all other changes].

\begin{proposition}
Energy use, and, therefore, emissions reductions, $G$, depends only on the
price of energy, $p_{e}$. With a production or consumption tax, the change
in energy price $\hat{p}_{e}=\frac{p_{e}^{\prime }}{p_{e}}$ to meet an
global emissions goal must satisfy
\end{proposition}

\begin{equation*}
G=\frac{\left( p_{e}^{\prime }\right) ^{\beta /\left( 1-\beta \right)
}E+\left( p_{e}^{\prime }\right) ^{\beta ^{\ast }/\left( 1-\beta ^{\ast
}\right) }E^{\ast }}{p_{e}^{\beta /\left( 1-\beta \right) }E+p_{e}^{\beta
^{\ast }/\left( 1-\beta ^{\ast }\right) }E^{\ast }}\left( 1+\hat{t}%
_{e}\right) ^{-\beta /\left( 1-\beta \right) }.
\end{equation*}%
where$1+$ $\hat{t}_{e}=\frac{1+t_{e}^{^{\prime }}}{1+t_{e}}$. In the special
case where $H$ and $F$ have the same labor share of extraction $\beta $, we
get

\begin{equation*}
\frac{\hat{p}_{e}}{1+\hat{t}_{e}}=G^{\left( 1-\beta \right) /\beta }
\end{equation*}

This result follows from the global supply curve \ in expression (\ref%
{Global supply of energy}), the definition of $\hat{Q}_{e}$ and because , in
the model, energy is costlessly traded, so there is a single global price of
energy.

This result (which also holds for taxes imposed only by $H$) means that for
any tax, we can determine its effects on emissions by determining the
equilibrium price of energy under the tax. Moreover, all taxes operate to
reduce emissions by lowering the global price of energy. With production and
consumption taxes, the nominal price goes down. With an extraction tax, the
after-tax price received by extractors goes down in exactly the same amount.

[Probably not worth mentioning, but global taxes have to be coordinated on
where in the chain of production they are imposed as well as the tax rate.
if $H$ imposes an extraction tax and $F$ imposes a production tax, we don't
get the same answer as when both impose one or the other.]

\section{Effects of unlateral taxes}

\subsection{Choice of tax rate}

Although we now examine taxes unilaterally imposed by $H$, we compare taxes
that have the same effect on global emissions. Reason is that purpose of the
tax is to reduce emisisons and the resulting harms. Because temperature
changes are the same regardless of which country pollutes, we assume that $H$
cares about global emissions.

This assumption may only partially capture $H$'s incentives. While $H$ will
care about global emissions because the harms to $H$ from climate change are
the same regardless of where the emissions come from, climate treaties often
focus on emissions from each country. If $H$ is imposing the tax primarily
to comply with a treaty, $H$ will care about domestic emissions.

With unilateral tax, extraction still depends on $p_{e}$, so the level of
emissions still depends only on $p_{e}$ (taking into account that with
extraction taxes, extractors only keep an after-tax amount).

\subsection{Location of activities}

Central concern with unilateral carbon prices is shift in where activities
occur. Note that this is not the same as leakage which is normally defined
based on changes in emissions. For example, consumption tax (we show) has
the potential to reduce consumption in $H$ which is partially offset by an
increase in consumption in $F$. This shift in the location of cousmption
does not necessarily impliy that the location of production will change.
Because emissions arise from production, there may be no leakage under a
consumption tax. But even if no leakage, this shift in consumption might be
a concern. Same for location of extraction.

[Is there a general statement we can make about location: each of the taxes
only directly affects the location of the taxed activity (extraction,
production or consumption). All other shifts occur either because of the
change in $p_{e}$ or an income effect due to tax revenue?]

[the below seems kind of obvious given the expressions above. Leave this out
and just refer to the expressions already given?]

\subsubsection{Extraction tax in $H$}

With extraction tax in $H$, we get

\begin{equation*}
Q_{e}=\left( \frac{p_{e}}{1+t_{e}}\right) ^{\beta /\left( 1-\beta \right) }E
\end{equation*}

[Or in hat notation]

\begin{equation*}
\hat{Q}_{e}=\left( \frac{\hat{p}_{e}}{1+\hat{t}_{e}}\right) ^{\beta /\left(
1-\beta \right) }.
\end{equation*}%
$Q_{e}^{\ast }$ is unchanged. For a given price of energy, we see relatively
lower extraction in $H$ than previously. Should push up the price of energy
(less supply), which means that in equilibrium, we see higher extraction in $%
F$, partially offsetting reduction in $H$.

[Worth specifying this - compare $\hat{Q}_{e}$ to $\hat{Q}_{e}^{\ast }$?]

Given an equilibrium price of energy, an extraction tax in $H$ does not
affect the location of production ($\bar{j}$ does not change) because energy
is costlessly traded so the location of the extraction of energy has no
effect on its use in manufacturing. The tax affects consumption only through
income effects.

The value of world supply of energy with an extraction tax in $H$ is:

\begin{equation}
p_{e}Q_{e}^{W}=\left( 1+t_{e}\right) ^{-\beta /\left( 1-\beta \right)
}p_{e}^{1/\left( 1-\beta \right) }E+p_{e}^{1/\left( 1-\beta \right) }E^{\ast
}  \label{Supply of energy with extraction tax in H}
\end{equation}

[also put this in hat notation - doesn't simplify so we get a fraction
similar to the global tax case but without the extraction tax in $H$.]

\subsubsection{Production tax in $H$}

No change in location of extraction (other than because $p_{e}$ changes).

$\bar{j}$ becomes:

\begin{equation*}
\bar{j}^{\prime }=\frac{A^{\theta }\left( 1+t_{p}^{\prime }\right) ^{-\theta
\left( 1-\gamma \right) }}{A^{\theta }\left( 1+t_{p}^{\prime }\right)
^{-\theta \left( 1-\gamma \right) }+A^{\ast \theta }}.
\end{equation*}

[Notation - it would be helpful to specify when $\bar{j}$ is with or without
tax. I added the "prime" to do this.]

Changes $p_{m}$.

\begin{equation*}
p_{m}=\phi p_{e}^{1-\gamma }\left( A^{\theta }\left( 1+t_{p}\right)
^{-\theta \left( 1-\gamma \right) }+A^{\ast \theta }\right) ^{-1/\theta }.
\end{equation*}

Changes $Y$. Given this, consumption is as above.

Could write demand equation here and note that it equals supply in
equilibrium.

\subsubsection{Consumption tax in $H$}

Price of the $m$-goods is now different in $H$ and $F$. In $H$, we get

\begin{equation*}
p_{m}=\phi \left( \left( 1+t_{c}\right) p_{e}\right) ^{1-\gamma }\left(
A^{\theta }+A^{\ast \theta }\right) ^{-1/\theta }.
\end{equation*}%
In $F$, it is unchanged from above. We get corresponding changes in demand
for the $m$-good (adjusting the income in $H$ for taxes.) But for any level
of emissions reduction and hence $p_{e}$, the location of extraction and of
manufacturing ( $\bar{j}$) are unchanged.

\subsection{Relationship between taxes}

A key policy proposal to accompany uinilateral carbon taxes is border
adjustments. In this section, we seek to understand the relationship between
border adjustments and the three types of taxes considered so far.

Border adjustments are (1) a tax (at rate $t_{b}$) on the value of energy
embodied in $H^{\prime }$s imports of the $m$-good and (2) a tax rebate
(also at rate $t_{b}$) on the value of energy embodied in $H$'s exports of
the $m$-good. The border tax adjustment tax rate need not be the same as the
tax rate on production, Instead, we allow $t_{b}\in \lbrack 0,t_{p}]$. [We
do, however, restrict the tax rate on imports to be the same as the rate of
rebate on exports.]

With no border adjustments ($t_{b}=0$), the tax is a pure production tax (at
rate $t_{p}$). We refer to the case where $t_{b}=t_{p}$ as full border taxes
and $t_{b}\in \left( 0,t_{p}\right) $ as partial border taxes.

Define the following:%
\begin{equation*}
1+\tilde{t}_{p}=\frac{1+t_{p}}{1+t_{b}}.
\end{equation*}

[Define two taxes as equal if they (1) produce the same equilibrium price of
energy and allocation of labor in each country and (2) the same utility for
all individuals in each country.]

We can characterize the relationship between production taxes, consumption
taxes, and border adjustments as follows:

\begin{proposition}
The combination of a production tax at rate $t_{p}$ and border adjustments
at rate $t_{b}$ is equivalent to a consumption tax at rate $t_{c}=t_{b}$ and
a residual production tax at rate $\tilde{t}_{p}$.
\end{proposition}

To see why this result arises, consider a consumption tax imposed only in $H$%
. Under this tax, consumers in $H$ pay a tax on all manufactured goods that
they purchase regardless of where they were produced. Compare that to a
production tax in $H$ with border adjustments. Under a production tax,
producers in $H$ must pay a tax on their production. If they sell the good
to consumers in $H$, there is no border adjustment, and the tax remains. If
they sell the good to consumers in $F$, the tax is removed, so of the goods
produced in $H$, only goods consumed in $H$ continue to bear a tax.
Similarly, producers in $F$ do not initially pay a tax when they produce a
good, but if, and only if, they sell it to consumers in $H$, border
adjustments impose a tax. Therefore, under a production tax with border
adjustments, all goods consumed in $H$ and only those goods bear a tax

\begin{proposition}
\begin{corollary}
A consumption tax in $H$ at rate $t_{c}$ is equal to a production tax in $H$
at rate $t_{p}=t_{c}$ plus full border adjustments (i.e., $t_{b}=t_{p}$).
\end{corollary}
\end{proposition}

This follows immediately from the proposition by setting $t_{b}=t_{p}$.

[I thought we had a similar result for extraction taxes but I\ don't see it
in the current notes.]

Put expressions for $R\left( j\right) $ and $\bar{j}$ here using this
notation (expression 39 in notes): these just replace $t_{p}$ with $\tilde{t}%
_{p}$ (and because the tax is unilateral, set $t_{p}^{\ast }=0$) but are
otherwise identical. Just say that?

Something about BAs under an extraction tax? Can tax extraction and border
adjust for imports and exports of energy to produce tax on energy used in $H$%
, so a produc tion tax. Or could border adjust for energy content of the $m$%
-goods, to produce a consumption tax.

\subsection{Leakage}

Conventional definition of leakage is $-\Delta F/\Delta H$. This can be hard
to interpret because if we hold global emissions fixed, the both the
numerator and the denominator change when we change the type of tax.If $%
\Delta F$ changes, then $\Delta H$ must change too if emissions are held
fixed. Often, leakage measures are compared holding the nominal tax rate
fixed, which makes them very hard to compare.

Propose an alternative measure of leakage, which we call modified leakage: $%
-\Delta F/\left( \Delta H+\Delta F\right) $. With this definition, the
denominator is the decline in global emissions, so if we compare two taxes
set to be equally effective, we know that the denominator is the same. Can
easily be converted into the conventional measure if desired.

Also leads to simplification because the denominator is determined by $p_{e}$%
: all leakage measures will assume the same assumptions about price of
energy.

Leakage often said to include two effects: a fuel price effect and a
location effect. Fuel price effect arises because carbon tax in $H$ lowers
the global price of energy, leading to increased consumption in $F$. With
modified leakage, the change in fuel price is held constant across all
scenarios. Depending on tax, consumers in $F$ see its effect differently, so
behavior of those consumers is not held fixed. \ E.g., with production tax,
consumers in $F$ see a tax on the goods that they purchase from $H$ while
with a consumption tax they do not. But because price of energy is held
constant in all scenarios, we do not try to isolate a fuel price effect.

Emissions in our model come only from manufacturing. Convenient to have
terms for emissions from manufacturing.

We want an expression for the total use of energy in $H$ for goods consumed
in $H$, which we denote $M_{H}$. We know from the demand function that firms
in $H$ receive $\bar{j}D\left( p_{m}\right) Y$ by selling $m$-goods in $H$.
The energy share is $\left( 1-\gamma \right) $, giving us $p_{e}M_{H}=\left(
1-\gamma \right) \bar{j}D\left( p_{m}\right) Y$. We can similar define the
following:

\begin{eqnarray}
p_{e}M_{H} &=&\left( 1-\gamma \right) \bar{j}D\left( p_{m}\right) Y  \notag
\\
p_{e}M_{F} &=&\left( 1-\gamma \right) \left( 1-\bar{j}\right) D^{\ast
}\left( p_{m}^{\ast }\right) Y^{\ast }  \notag \\
p_{e}M_{H}^{\ast } &=&\left( 1-\gamma \right) \bar{j}D\left( p_{m}\right) Y 
\notag \\
p_{e}M_{F}^{\ast } &=&\left( 1-\gamma \right) \left( 1-\bar{j}\right)
D^{\ast }\left( p_{m}^{\ast }\right) Y^{\ast }
\end{eqnarray}%
We then have $M_{H}+M_{F}=M_{W}$, and $M_{H}^{\ast }+M_{F}^{\ast }=M_{W}$,
and $M_{W}+M_{F}^{\ast }=M_{W}=Q_{e}^{W}$.

Modified leakage for a given global reduction in emissions $G$ is:\ 

\begin{equation*}
\tilde{l}_{P}=\frac{\left( M_{e}^{HF^{\prime }}+M_{e}^{FF^{\prime }}\right)
-\left( M_{e}^{HF}+M_{e}^{FF}\right) }{\left(
M_{e}^{HH}+M_{e}^{FH}+M_{e}^{HF}+M_{e}^{FF}\right) \left( 1-G\right) }
\end{equation*}

\begin{proposition}
Leakage with a production or consumption tax is:
\end{proposition}

\begin{equation*}
\tilde{l}_{P}=\frac{1-\bar{j}}{1-G}\left( \frac{G}{\bar{j}\left( 1+\tilde{t}%
_{p}^{\prime }\right) ^{-\theta \left( 1-\gamma \right) -1}+1-\bar{j}}%
-1\right)
\end{equation*}

Recalling that $\tilde{t}_{p}^{\prime }$ is the residual production tax if
there are partial border tax adjustments, we had previously that if $\tilde{t%
}_{p}^{\prime }=0$, the tax is a pure consumption tax. We therefore get:

[This proposition assumes $\beta =\beta ^{\ast }$, right?]

\begin{corollary}
With a pure consumption tax, leakage is equal to $-\left( 1-\bar{j}\right)
<0 $.
\end{corollary}

Need to explain why we get negative leakage when nobody else does (other
than DF who gets it for entirely different reasons). Basic idea here: with a
consumption tax, producers in both $H$ and $F$ can sell fewer $m$-goods to
consumers in $H$. Energy prices go down, which means producers in both
countries sell more to consumers in $F$ but to generate a net reduction,
producers have to produce overall less. Producers in $F$ will be affected
the same way producers in $H$ will. Home bias could offset this somewhat and
most other models have home bias built in, although it is not clear that
this is the full explanation for the difference.

\subsection{Welfare\newline
}

Design of a carbon tax, including choice to impose border adjustments must
be based on welfare, not on a measure of leakage or of the location of
activities.

Define welfare as effective consumption: $\frac{Y}{p}$, where $p$ is the
price index for the $m$-good and the $l$-good: $p=\left( \alpha
p_{m}^{-\left( \sigma -1\right) }+\left( 1-\alpha \right) \right)
^{-1/\left( 1-\sigma \right) }$.

\subsubsection{$F$'s welfare}

\begin{proposition}
$F$ is better off with a consumption tax in $H$. That is, if $H$ is going to
impose a unialteral production tax that acheives a desired reduction in
global emissions, $F$ is bette off if $H$ imposes border adjustments as well.
\end{proposition}

The reasoning is straightforward, Holding emissions constant means holding $%
p_{e}$ fixed and by construction, the price of the $l$-good is fixed at 1.
Because income in $F$ is from wages and rents from energy extraction, income
in $F$ is fixed. This leaves only the price of the $m$-good to vary. With a
production tax, any varieties of the $m$-good produced in $H$ that consumers
in $F$ purchase will bear the tax while with a border adjustment, the tax is
removed. Therefore, $F$ is better off with border adjustments.

\subsubsection{$H$'s welfare}

\begin{proposition}
If $H$ sufficiently dominates manufacturing, $H$ is better off without
border adjustments.
\end{proposition}

Reasoning:

\section{Simulations}

We only need the share notation (the $\pi $'s) here, because their use is
for calibration, right? No need to introduce earlier. A few derivations use
it but I think all can easily be restated without. But needed for
calibration of the simulations, so introduce here.

Possible simulations:

\begin{itemize}
\item Beta v. leakage under a production tax (and possibly, under a
consumption tax).

\item leakage under a production tax with variable border adjustments

\item $H$'s welfare under production tax plus variable border adjustments
(i.e., BA's on x axis, welfare on y axis).

\item leakage v. relative size of $H$

\item Welfare gains v. relative size of $H$ (under production tax?): tells
us the benefits of increasing the size of the taxing coalition.
\end{itemize}

\section{Comparisons to CGE results}

Want to compare our results to the summary of the CGE results given above;

\begin{itemize}
\item Leakage rates most often in the range of 5\% to 20\%, with some
outliers.

\item The larger the taxing coalition, the lower the leakage.

\item BA's reduce leakage substantially

\item Most important variable in determining effects of unilateral carbon
price is the energy supply elasticity.

\item Distinguish two drivers of leakage: the fuel price effect (in which
lower demand for fossil fuels in the taxing region suppresses prices,
increasing demand on the non-taxing regions) and the competitiveness effect
[different name?] (in which increased costs for industry in the taxing
region causes a shift to the non-taxing region).
\end{itemize}

\end{document}
